\documentclass[pdf]{beamer}
\mode<presentation>{}
\usetheme{Dresden}
\usepackage{apalike}
\usepackage{graphicx}
\usepackage{mwe,tikz}\usepackage[percent]{overpic}
%% preamble
\title{A Finite Element-Volume Method for the Serre Equations}
\author{Jordan Pitt, Stephen Roberts and Christopher Zoppou \\ Australian National University}
\newcommand\solidrule[1][0.25cm]{\rule[0.5ex]{#1}{1pt}}
\newcommand\dashedrule{\mbox{\solidrule[2mm]\hspace{2mm}\solidrule[2mm]}}
\newcommand{\dotrule}[1]{%
	\parbox[]{#1}{\dotfill}}

\setbeamertemplate{itemize item}[triangle]

\begin{document}
%% title frame
\begin{frame}
\titlepage
\end{frame}
%% normal frame

%Do a brief show pictures of water waves, hazards posed
%Move onto the equations, with a picture
%Method: in terms of polynomial representation, FEM, FVM
%Validation: Present results of linear analysis: stability and dispersion analysis, then numerical solution comparisons
 
\begin{frame}{Introduction}
\begin{itemize}
	\item Motivation
	\item Method
	\item Properties and Validation
\end{itemize}
\end{frame}
\section{Motivation}
\begin{frame}
\end{frame}
\subsection{Phenomena}
\begin{frame}
\end{frame}
\subsection{Equations}
\begin{frame}
\end{frame}

\section{Method}
\begin{frame}
\end{frame}
\subsection{Polynomial Representation}
\begin{frame}
\end{frame}
\subsection{Calculation of Velocity}
\begin{frame}
\end{frame}
\subsection{Calculation of Fluxes}
\begin{frame}
\end{frame}
\subsection{Finite Volume Method}
\begin{frame}
\end{frame}

\section{Properties and Validation}
\begin{frame}
\end{frame}
\subsection{Linear Analysis}
%Stability, pictures
%Dispersion, lowest order terms and how it looks as dx -> 0
\begin{frame}
\end{frame}
\subsection{Analytic Solution}
%Soliton
%L1 error (h,G) , report conservation of mass and G
\begin{frame}
\end{frame}
\subsection{Forced Solution}
%Dry bed bump
%L1 error (h,G)
\begin{frame}
\end{frame}
\subsection{Experimental Comparison}
%Synolakis (Roeber?)
\begin{frame}
\end{frame}
\end{document}