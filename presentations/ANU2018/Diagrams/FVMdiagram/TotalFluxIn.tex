\documentclass[border=10pt]{standalone}
\usepackage{pgfplots} 
%\usepgfplotslibrary{external} 
%\tikzexternalize 
\usepgfplotslibrary{fillbetween}
\usetikzlibrary{arrows, decorations.markings}
\usepackage{tikz} 
\usepackage{amsmath} 
\usepackage{pgfplots} 
\usetikzlibrary{calc} 
\pgfplotsset{compat = newest, every axis plot post/.style={line join=round}, label style={font=\Large},every tick label/.append style={font=\Large} }
\begin{document} 
	\begin{tikzpicture}
	\makeatletter
	\tikzset{
		nomorepostaction/.code=\makeatletter\let\tikz@postactions\pgfutil@empty, % From https://tex.stackexchange.com/questions/3184/applying-a-postaction-to-every-path-in-tikz/5354#5354
		my axis/.style={
			postaction={
				decoration={
					markings,
					mark=at position 1 with {
						\arrow[ultra thick]{latex}
					}
				},
				decorate,
				nomorepostaction
			},
			thin,
			-, % switch off other arrow tips
			every path/.append style=my axis % this is necessary so it works both with "axis lines=left" and "axis lines=middle"
		}
	}
	\makeatother
	
	\begin{axis}[ 
	axis lines = left, 
	axis line style={my axis},
	width=15cm,
	height = 7.5cm,
	xtick={1,3,5,7,9}, 
	clip=false, 
	ytick = {\empty},
	xticklabels ={$x_0$,$x_1$,$x_2$,$x_3$,$x_4$},
	yticklabels ={},
	extra x ticks={0,2,4,6,8,10},
	%extra x tick style={xticklabels=\empty},
	extra x tick labels={\normalsize$x_{-\frac{1}{2}}$,\normalsize$x_{\frac{1}{2}}$,\normalsize$x_{\frac{3}{2}}$,\normalsize$x_{\frac{5}{2}}$,\normalsize$x_{\frac{7}{2}}$,\normalsize$x_{\frac{9}{2}}$	},
	xmin=0, 
	xmax=10, 
	ymin =-0.25, 
	ymax = 1,
	xlabel=$x$, 
	ylabel=  {$z$} ]
	\addplot [blue,no markers, raw gnuplot,thick] gnuplot[id=poly2] {plot 'fileW1.dat' smooth  cspline; };
	\addplot [name path=a,blue,no markers, raw gnuplot,thick,restrict x to domain=4:6] gnuplot[id=poly2] {plot 'fileW1.dat' smooth  cspline; };
	
	\path[name path=axis] (axis cs:4,-0.25) -- (axis cs:6,-0.25);
	
	\node[] at (axis cs: 5,1){\Large\color{blue}$q(x,t_0)$};

	
%		\addplot [
%		thick,
%		color=brown!60!black,
%		fill=brown!60!black, 
%		fill opacity=0.3
%		] fill between[of=b and axis];
		
	\addplot [
	thick,
	color=blue,
	fill=blue, 
	fill opacity=0.3
	] fill between[of=a and axis];
	
	\draw[dashed,thick] (2, -0.25  ) -- (2,1);
	\draw[dashed,thick] (4, -0.25  ) -- (4,1);
	\draw[dashed,thick] (6, -0.25  ) -- (6,1);
	\draw[dashed,thick] (8, -0.25  ) -- (8,1);
	
%	\path[->,thick] (1.7, 0.65) edge node[sloped, anchor=south east]{\Large$F_{\frac{1}{2}}$}  (2.3, 0.65);
	\path[->,thick] (3.7, 0.65) edge node[sloped, anchor=south east]{\Large$F_{\frac{3}{2}}$}  (4.3, 0.65);
%	\path[->,thick] (5.7, 0.65) edge node[sloped, anchor=south east]{\Large$F_{\frac{5}{2}}$}  (6.3, 0.65);
%	\path[->,thick] (7.7, 0.65) edge node[sloped, anchor=south east]{\Large$F_{\frac{7}{2}}$}  (8.3, 0.65);

%	\node[label={90:{\Large$S_1$}},circle,fill,inner sep=2pt] at (axis cs:3,0.0) {};
%	\node[label={90:{\Large$S_0$}},circle,fill,inner sep=2pt] at (axis cs:1,0.0) {};
%	\node[label={90:{\Large$S_2$}},circle,fill,inner sep=2pt] at (axis cs:5,0.0) {};
%	\node[label={90:{\Large$S_3$}},circle,fill,inner sep=2pt] at (axis cs:7,0.0) {};
%	\node[label={90:{\Large$S_4$}},circle,fill,inner sep=2pt] at (axis cs:9,0.0) {};
%	
		\draw [decorate,thick,decoration={brace,amplitude=10pt}]	(2,-0.5) -- (0,-0.5);
		\draw [decorate,thick,decoration={brace,amplitude=10pt}]	(4,-0.5) -- (2,-0.5);
		\draw [decorate,thick,decoration={brace,amplitude=10pt}]	(6,-0.5) -- (4,-0.5);
		\draw [decorate,thick,decoration={brace,amplitude=10pt}]	(8,-0.5) -- (6,-0.5);
		\draw [decorate,thick,decoration={brace,amplitude=10pt}]	(10,-0.5) -- (8,-0.5);
	
		\node[] at (axis cs: 1,-0.65){\Large$C_0$};
		\node[] at (axis cs: 3,-0.65){\Large$C_1$};
		\node[] at (axis cs: 5,-0.65){\Large$C_2$};
		\node[] at (axis cs: 7,-0.65){\Large$C_3$};
		\node[] at (axis cs: 9,-0.65){\Large$C_4$};
		
		\node[] at (axis cs: 5,0.5){\LARGE$\bar{q}_2$};
	\end{axis} 
	
	
	
	\end{tikzpicture}
\end{document}