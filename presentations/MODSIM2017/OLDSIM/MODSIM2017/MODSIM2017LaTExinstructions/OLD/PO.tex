\documentclass[a4paper,fleqn]{article} %Options in documentclass should be set to a4paper and fleqn.
\usepackage{modsim}
\usepackage{times}
\usepackage{psfrag}
 \usepackage{pstricks}
 %\usepackage{auto-pst-pdf}
\usepackage{pstool}
\usepackage{natbib} %The three packages modsim, times and natbib are required.
\usepackage{amsmath, amssymb, amsthm} %Also recommend the standard AMS LaTeX maths packages.
\usepackage{changes}

\newcommand\solidrule[1][0.25cm]{\rule[0.5ex]{#1}{1pt}}
\newcommand\dashedrule{\mbox{%
		\solidrule[2mm]\hspace{2mm}\solidrule[2mm]}}
\pagestyle{MODSIMheadings} %Calling the MODSIM Headings format
\MODSIMhead{J. Pitt {\it et al.}, Importance of Dispersion for Shoaling Waves} %This is the content of the headings in all pages (except the first page). The format should be author, title of paper. If the title is too long, then use ... at the end. If more than two authors, then please use the "et al." format with et al. in italic, for example A. Author {\it et al.}, Title of the paper.

% Define any other command or required packages below:
%%%%%%%%%%%%%%%%%%%%%%%%%%%%%%%%%%%%%%%%%%%%%%%%%%%%%%%%%%%%%%%%%%%%%%%%%%%%%%%%%%%%%%%%%%%%%%%%%%%%%%%%%%%
\usepackage{rotating}
\usepackage{amsbsy,enumerate}
\usepackage{graphicx}
\usepackage{ccaption}
\newcommand{\ve}{\varepsilon}
\newcommand{\sigmah}{\hat{\sigma}}
\newcommand{\sigmab}{\bar{\sigma}}
\newcommand{\R}{\mathbb{R}}
\newcommand{\Z}{\mathbb{Z}}
\newcommand{\E}{\mathbb{E}}
\newcommand{\Tbar}{\bar{T}}
\newcommand{\Thetabf}{\mathbf{\Theta}}
\newcommand{\Xbf}{\mathbf{X}}
\newcommand{\xbf}{\mathbf{x}}
\newcommand{\Ybf}{\mathbf{Y}}
\newcommand{\ybf}{\mathbf{y}}
\newcommand{\Ubf}{\mathbf{U}}
\newcommand{\Fbf}{\mathbf{F}}
\newcommand{\fbf}{\mathbf{f}}
\newcommand{\Pbf}{\mathbf{P}}
\newcommand{\Pbfbar}{\bar{\mathbf{P}}}
\newcommand{\Sb}{\bar{S}^{\nu}}
\newcommand{\Snuib}{\tilde{S}^\nu _i}
\newcommand{\Snui}{S^\nu _i}
\newcommand{\sigmanui}{\sigma ^\nu_i}
\newtheorem{lemma}{Lemma}
\newtheorem{corollary}{Corollary}
\newtheorem{assumption}{Assumption}
\newtheorem{theorem}{Theorem}
\newtheorem{remark}{Remark}
\newtheorem{definition}{Definition}
\newtheorem{cce}{Assumption CCE}
\newtheorem{proposition}{Proposition}
%% The next four lines create the correct format for Figure and Table captions. They require the package "ccaption".
\makeatletter
\renewcommand{\fnum@figure}[1]{\textbf{\figurename~\thefigure}. }
\renewcommand{\fnum@table}[1]{\textbf{\tablename~\thetable}. }
\makeatother

%%%%%%%%%%%%%%%%%%%%%%%%%%%%%%%%%%%%%%%%%%%%%%%%%%%%%%%%%%%%%%%%%%%%%%%%%%%%%%%%%%%%%%%%%%%%%%%%%%%%%%%%%%%

\begin{document}
% Defining Front matter.

\title{Importance of Dispersion for Shoaling Waves}
\author{\underline{J. Pitt} \address[ANU]{\it{Mathematical Sciences Institute, Australian National University, Canberra, ACT 0200, Australia}}, C. Zoppou \addressmark[ANU], S.G Roberts \addressmark[ANU]}


\email{jordan.pitt@anu.edu.au} %Email address of the presenting author only.

\date{June 2017}

%Keywords should be separated by commas and listed in Sentence case (first keyword with capital first letter and remaining keywords in lower case).
\begin{keyword}
	Tsunamis, shoaling waves, dispersion, models

\end{keyword}

\begin{abstract}
A tsunami has four main stages of its evolution; in the first stage the tsunami is generated, most commonly by seismic activity near subduction zones. The second stage is the tsunamis propagation through relatively deep water compared to its amplitude over large horizontal distances. The third stage is the shoaling and interaction of the tsunami with bathymetery as it the coast. Finally the tsunami reaches the coast and begins inundating previously dry land. For our purposes the hydrodynamic models we are interested in deal with the final three stages of the evolution of a tsunami.

The propagation of a tsunami through relatively deep water is well understood, as the whole wave travels without much deformation at almost constant speed. This stage of a tsunami is well modelled by current industrial models based on the shallow water wave equations. Current research into tsunamis focuses around more complex approximations to the Euler equations for the third and fourth stages. In this paper we focused on the Serre equations as they are considered a very good model for fluid behaviour up to the shoreline. 

Because the Serre equations allow for more fluid behaviour they are more complex than the shallow water wave equations and therefore more computationally expensive to numerically solve. Therefore, to allow us to simulate tsunamis as efficiently as possible it is important to know when using the more complicated Serre equations leads to significantly different behaviour in the evolution of a tsunami. To investigate this we have numerically simulated an experiment observing the propagation of waves over a submerged bar, and the propagation of a small amplitude wave up a gradual linear slope using both the Serre equations and the shallow water wave equations.

The results of these simulations demonstrated that the Serre and shallow water wave equations produce similar results for shoaling waves when the wave height is small compared to water depth and the horizontal scales are larger than the vertical scales. This is not surprising as this is the regime under which the shallow water wave equations are derived. However, outside this regime the shallow water wave equations are a poor model for wave shoaling and propagation, poorly approximating the shape and maximum height of waves. These results suggest that for a tsunami it is sufficient to use the shallow water wave equations in stages two and some of stage three, even for large changes in bathymetery. Although a switch to dispersive equations such as the Serre equations is required to accurately capture fluid behaviour in stages three and four nearer to the coast line.


\end{abstract} %Abstract should NOT extend beyond the first page.

\maketitle

\section{INTRODUCTION}
%Problem
%Serre equations
%numerical methods

The interaction of waves with bathymetry and particularly the shoaling of waves plays a central role in modelling the inundation caused by a tsunami. Current industrial models for tsunami inundation such as ANUGA utilize the non-dispersive shallow water wave equations. Current research into tsunami inundation includes building numerical models based on dispersive equations, such as the Serre equations. Dispersive equations allow for more fluid behaviour such as dispersion, which is the process by which waves of different frequencies travel at different speeds. Because dispersive equations model more fluid behaviour they are a better choice for inundation modelling than the shallow water wave equations. The Serre equations are considered by \cite{Bonneton-etal-2011-589} to be the most appropriate approximation to the Euler equations near the shoreline. The one-dimensional Serre equations for a depth of water $h(x,t)$ over a bed profile defined by $b(x)$ travelling at velocity $u(x,t)$ are given by \cite{Zoppou-etal-2017} as 

\begin{subequations}\label{eq:Serre_nonconservative_form}
	\begin{gather}
	\dfrac{\partial h}{\partial t} + \dfrac{\partial (uh)}{\partial x} = 0
	\label{eq:Serre_continuity}
	\end{gather}
	and
	\begin{gather}
\frac{\partial{u} h }{\partial t} + \frac{\partial}{\partial x} \left({u}^2 h + \frac{gh^2}{2} + \frac{h^3}{3}  \Phi + \frac{h^2}{2}\Psi \right)   + \left(gh + \frac{h^2}{2}\Phi + h \Psi \right) \frac{\partial b}{\partial x} = 0
	\label{eq:Serre_momentum}
	\end{gather}
	where
	\begin{align*}
	\Phi &= \left(\frac{\partial {u}}{\partial x} \right)^2 - {u}\frac{\partial^2 {u}}{\partial x^2} - \frac{\partial^2 {u}}{\partial x \partial t} \\
	\Psi &= \frac{\partial {u}}{\partial t} \frac{\partial b}{\partial x} + {u}\frac{\partial {u}}{\partial x}\frac{\partial b}{\partial x} + {u}^2\frac{\partial^2 b}{\partial x^2}.
	\end{align*}	
\end{subequations}

Setting $\Phi = \Psi = 0$ in the Serre equations transforms them into the shallow water wave equations. The additional $\Phi$ and $\Psi$ terms make the Serre equations more complex and therefore more computationally difficult to solve. However, the additional behaviour they allow for is not always significant. In this paper we will investigate the impact of these extra terms on the numerical solutions of the Serre equations compared to the shallow water wave equations. To do this we will use a well-validated numerical method for the Serre equations by \cite{Zoppou-etal-2017} and the ANUGA software for the shallow water wave equations. We begin by simulating an experiment of \cite{Beji-Battjes-1994} which observed periodic waves travelling over a submerged bar, and then move to a more appropriate scenario for tsunamis, the propagation of a small amplitude solitary wave up a linear slope

%--------------------------------------------------------------------------------
\section{Periodic Wave Over a Submerged Bar}
\label{Oscillatory Wave Over a Submerged Bar}
%--------------------------------------------------------------------------------
\cite{Beji-Battjes-1994} conducted a series of experiments where waves travel over a submerged trapezoidal bar. Their wave tank was $37.7$m in length, $0.8$m wide and $0.75$m high with a constant water stage of $0.4$m. The bed had the following profile; $(x,b) = [$($0$m,$0$m), ($6$m,$0$m), ($12$m,$0.3$m), ($14$m,$0.3$m), ($17$m,$0$m), ($18.95$m,$0$m), ($28.95$m,$0.4$m)$]$. Waves were generated by a wave maker located at $x=0$m, and were absorbed downstream by a $3$m long gravel beach, with a slope of $1:25$. There were seven wave gauges that recorded the depth of water over time at the following locations; WG1: $5.7$m, WG2: $10.5$m, WG3: $12.5$m, WG4: $13.5$m, WG5: $14.5$m, WG6: $15.7$m and WG7: $17.3$m.
%[how do you do this, B.C]

In this paper we focus on the propagation of periodic non-breaking sinusoidal waves with high frequency over the bar. These waves had a period of $T = 2$s and an amplitude of $A = 0.01$m. These waves travelled without deformation towards the bar. The waves steepened as they travelled up the front slope of the bar due to shoaling. On the horizontal top of the bar higher harmonic waves are slowly released. This release of waves is then accelerated as the waves travel down slope. Although bars do occur throughout the worlds oceans, in the context of tsunami modelling we are more interested in the behaviour of waves on the front slope region and the horizontal top of the bar and so we focus on wave gauges $1$ through $4$. 

The numerical results for the Serre equations are presented in Figure \ref{fig:WGSerre} and the numerical results for the shallow water wave equations are presented in Figure \ref{fig:WGSWW}. In these experiments wave gauge $1$ is used as the left boundary condition to recreate the incoming wave, so both simulations reproduce it precisely, we include it here as a reference for the incoming wave profile. It can be seen that the Serre equations reproduce the experiment results well even though the Serre equations are only an approximation to the true fluid behaviour, and in fact altering the dispersion relation of the Serre equations results in an even more accurate reproduction of the experiment as shown by \cite{Lannes-2013}. The shallow water wave equations results are poor for both the shape and the speed of the waves. This is because the shallow water wave equations are appropriate when the wave height is very small compared to the depth, and the horizontal scales are far larger than the vertical scales, which is not the case in these experiments. This experiment is specifically designed to study dispersive effects, and so the failure of the shallow water wave equations is not surprising as they neglect  dispersion. This demonstrates that dispersion can be a significant factor in the shoaling of waves, with large differences between the simulations using the Serre and shallow water wave equations.

\begin{figure}
	\centering
	\begin{tabular}{ccc}
		\resizebox{5cm}{!}{\includegraphics{WG1.pdf}} & & \resizebox{5cm}{!}{\includegraphics{WG2Serre.pdf}}  \\
		\hspace{5 mm} WG1 & & \hspace{5 mm}WG2 \\
		\resizebox{5cm}{!}{\includegraphics{WG3Serre.pdf}} && 
		\resizebox{5cm}{!}{\includegraphics{WG4Serre.pdf}}  \\
		\hspace{5 mm} WG3 & & \hspace{5 mm} WG4 \\
		
		
	\end{tabular}
	\caption{Wave height readings from various wave gauges from the experiment of \cite{Beji-Battjes-1994} ({\color{blue} \solidrule}) compared to a simulation of the experiment using the Serre equations ({\color{red} $\circ$}).}
	\label{fig:WGSerre}
\end{figure}

\begin{figure}
	\centering
	\begin{tabular}{ccc}
		\resizebox{5cm}{!}{\includegraphics{WG1.pdf}} & & \resizebox{5cm}{!}{\includegraphics{WG2SWW.pdf}}  \\
		\hspace{5 mm} WG1 & & \hspace{5 mm}WG2 \\
		\resizebox{5cm}{!}{\includegraphics{WG3SWW.pdf}} && 
		\resizebox{5cm}{!}{\includegraphics{WG4SWW.pdf}}  \\
		\hspace{5 mm} WG3 & & \hspace{5 mm} WG4 \\
		
		
	\end{tabular}
	\caption{Wave height readings from various wave gauges from the experiment of \cite{Beji-Battjes-1994} ({\color{blue} \solidrule}) compared to a simulation of the experiment using the shallow water wave equations ({\color{green!60!black} $\square$}).}
	\label{fig:WGSWW}
\end{figure}
	
%--------------------------------------------------------------------------------
\section{Propagation of a Soliton over a Slope.}
\label{Soliton up a Slope}
%--------------------------------------------------------------------------------
The numerical results for the propagation of periodic waves over a submerged bar demonstrated large differences between the Serre and shallow water wave equations. These differences are pronounced due to the fact that the experimental set-up had a larger ratio of wave amplitude to water depth and vertical to horizontal scale, than appropriate for the shallow water wave equations. To rectify this another numerical experiment was performed simulating the propagation of a small amplitude wave over a linear slope.  

In this simulation the initial conditions are the soliton of the Serre equations from \cite{El-etal-2006} with an amplitude of $0.01m$ on top of quiescent water $1m$ deep. The stage is kept at a fixed $1m$ throughout the scenario. The bed profile is defined in the following way; $(x,b) = [$($-100$m,$0$m), ($100$m,$0$m), ($149.5$m,$0.99$m), ($250$m,$0.99$m)$]$. This scenario simulates the propagation of a relatively small amplitude wave over large horizontal distances, over a relatively shallow slope much like a tsunami. The initial conditions and the numerical solutions for both the Serre equations and the shallow water wave equations are plotted in Figure \ref{fig:allstructs}.

It can be seen that the behaviour of the numerical solutions of the Serre equations and the shallow water wave equations are very similar when the water depth is large compared to the wave amplitude, with the soliton travelling over a large distance compared to the water depth with very little deformation for both equations after $t=20s$. For the Serre equations the soliton travels without deformation due to a balance between non-linearity and dispersion, for the shallow water wave equations it travels with little deformation because the effects of non-linearity are small as the wave amplitude is small. As the soliton progresses up the slope there is also very little difference between the numerical solutions throughout the shoaling process as can be seen at $t=40s$. It is only when the bed becomes flat after the slope that differences begin to arise although both numerical solutions are quite similar at $t=60s$ and $100s$. The main difference between the Serre equations and the shallow water wave equations is the emergence of undulations on top of the bore that has formed. These are physical oscillations that have been observed in well validated numerical solutions of the Serre equations such as those of \cite{Mitsotakis-etal-2014} and even the Euler equations as shown by \cite{Mitsotakis-etal-2017}. These undulations oscillate around the bore of the shallow water wave equations, significantly increasing its maximum amplitude. It was demonstrated by \cite{Pitt-2017} that undular bores of the Serre equations travel slightly quicker than the bores of the shallow water wave equations. This suggests that the shallow water wave equations will slightly underestimate the arrival time and maximum amplitude of an incoming wave.

\begin{figure}
	\centering
	\resizebox{6cm}{!}{\includegraphics{solitonslopeSerre1.pdf}}
	\\
	\hspace{10 mm}$t= 0s$
	\\ 
	\begin{tabular}{ccc}
		\resizebox{5cm}{!}{\includegraphics{solitonslopeSerre2.pdf}} & & \resizebox{5cm}{!}{\includegraphics{solitonslopeSerre3.pdf}}  \\
		\hspace{5 mm} $t= 20s$ & & \hspace{5 mm}$t= 40s$ \\
		\resizebox{5cm}{!}{\includegraphics{solitonslopeSerre4.pdf}} && 
		\resizebox{5cm}{!}{\includegraphics{solitonslopeSerre5.pdf}}  \\
		\hspace{5 mm} $t= 60s$ & & \hspace{5 mm} $t= 100s$ \\
		
		
	\end{tabular}
	\caption{The stage from the numerical solution for the Serre equations ({\color{blue} \solidrule}) and the shallow water wave equations ({\color{red} \solidrule}) for the soliton travelling over a slope, at different times. The initial conditions for both numerical models are also presented in terms of stage ({\color{black} \solidrule}) and bed profile ({\color{green!60!black} \solidrule}).}
	\label{fig:allstructs}
\end{figure}


%--------------------------------------------------------------------------------
\section{Conclusions}
\label{Conclusions}
Well-validated numerical methods for the Serre and shallow water wave equations for two simulations were used to investigate the effect of dispersion on shoaling waves. It was demonstrated that when the ratios of wave height to water depth and vertical scale to horizontal scale are small, dispersion is negligible during shoaling. However, dispersion is significant during the propagation of the these shoaled waves over relatively thin water. As these ratios become larger dispersion plays a larger role in shoaling and the propagation of waves over flat bathymetery. This demonstrates that in the tsunami context, the shallow water wave equations are sufficient for modelling shoaling waves in relatively deep water over vast horizontal distances. However, a switch to dispersive equations is required near the shoreline to accurately model the true fluid behaviour.   

%--------------------------------------------------------------------------------
\newpage	
\bibliographystyle{chicago} %use chicago style of referencing.
\bibliography{DamBreak}
\end{document}
