\chapter{Linear Analysis Results}
\label{app:LinAnal}
In this appendix we provide additional details supplementing the linear analysis results reported in Chapter \ref{chp:AnalNumMethod}. We provide the evolution matrices $\matr{E}$ for all methods in the thesis, details on how to calculate the convergence and dispersion properties from the evolution matrix where appropriate and the consistency tables not already in Chapter \ref{chp:AnalNumMethod}.  

The linear analysis in Chapter \ref{chp:AnalNumMethod} studies the convergence and dispersion properties of the numerical methods solving the Serre equations with a horizontal bed \eqref{eqn:FullSerreConHorizBed} linearised around a mean water depth $H$ and a mean background flow velocity $U$. With a small perturbation in the water depth $\eta$ and a small perturbation in the velocity $\mu$. The linearised Serre equations with a horizontal bed \eqref{eqn:LinAll} are
\begin{subequations}
	\label{appC:Lin}
	\begin{align}
	&\frac{\partial  \eta}{\partial  t} + H\frac{\partial \mu}{\partial  x} + U\frac{\partial  \eta}{\partial  x}  = 0, \\
	&H\frac{\partial  \mu}{\partial  t} + gH\frac{\partial  \eta}{\partial  x} + UH\frac{\partial  \mu}{\partial  x} - \frac{H^3}{3}\left(U\frac{\partial^3  \mu}{\partial  x^3} + \frac{\partial^3  \mu}{\partial  x^2 \partial  t}  \right)  = 0
	\end{align}	
\end{subequations}
which can be written in conservation law form \eqref{eqn:LinSerreG} for $\eta$ and $G$ as
\begin{subequations}
	\label{appC:ConLawFormLin}
	\begin{align}
	&\frac{\partial  \eta}{\partial  t} +\frac{\partial}{\partial  x} \left(H\mu + U \eta\right) = 0, \\
	&\frac{\partial  G}{\partial  t} + \frac{\partial}{\partial  x}\left(UG + UH\mu + gH \eta\right) = 0
	\end{align}
	where
	\begin{equation}
	G = UH + U \eta + H \mu -\frac{H^3}{3} \frac{\partial^2 \mu }{\partial x^2}.
	\end{equation}
\end{subequations}
The linear analysis assumes that $\eta$ and $\mu$ are Fourier modes \eqref{eqn:FourierNode}. A generic quantity $q$ is a Fourier mode if
\begin{align}
q(x,t) = q(0,0) e^{i\left(\omega^\pm t + kx\right)}
\label{eqn:FourModeApp}
\end{align}
where $\omega^\pm$ is the frequency of the wave and $k$ is the wavenumber. Consequently, $G$ as well as the cell averages $\overline{\eta}$, $\overline{\mu}$ and $\overline{G}$ are also Fourier modes \eqref{eqn:FourModeApp}.

Since these quantities are Fourier modes for a fixed grid such as those used by all numerical methods in this thesis we can relate the quantities at different nodal values to one another. For a generic Fourier mode $q$ we have
\begin{equation}
q^{n + m}_{j + l} = q^n_j e^{ i \left(m \omega^\pm \Delta t + l k \Delta x\right)}
\label{eqn:fourfac}
\end{equation}
where $\Delta t$ is the length of time step and $\Delta x$ is the length of a cell. 


Applying the numerical method to the linearised Serre equations \eqref{appC:Lin} and using \eqref{eqn:fourfac} one can derive an expression relating the quantities at the current and earlier times to the quantities at the next time via the evolution matrix $\matr{E}$ of the numerical method. This evolution matrix can then be analysed to obtain the convergence and dispersion properties of the numerical method. 

This appendix will provide the results for the first-, second- and third-order Finite Difference Volume Methods (FDVM) which are named $\text{FDVM}_1$, $\text{FDVM}_2$ and $\text{FDVM}_3$ respectively. Descriptions of these methods were published by \citet{Zoppou-etal-2017}. Additionally, we provide the evolution matrices for the two second-order Finite Difference Methods (FDM) $\mathcal{D}$ and $\mathcal{W}$, whose descriptions were published by \citet{Pitt-2018-61}. 

The linear analysis for the second-order finite element volume method $\text{FEVM}_2$ was performed in Chapter \ref{chp:AnalNumMethod}. Given that work the evolution matrix for other finite volume based methods can be written in terms of the reconstruction operators and the velocity operators. We demonstrate this process and provide the expressions for these constituent operators. Allowing others to reproduce the evolution matrices for all finite volume based methods in this thesis.  

For $\mathcal{D}$ and $\mathcal{W}$ the evolution matrix itself is provided. Since these FDM use the quantities at previous time steps, the shape of the evolution matrix is different. Hence the process to obtain the convergence and dispersion properties from these particular evolution matrices will be explained. 

Finally, the consistency results for $\text{FDVM}_1$, $\text{FDVM}_2$, $\text{FDVM}_3$, $\mathcal{D}$ and $\mathcal{W}$ will be provided. The results for $\text{FEVM}_2$ were provided in Chapter \ref{chp:AnalNumMethod}.  


\section{Finite Difference Volume Methods}
For the FDVM the evolution matrix is a $2\times2$ matrix that gives the following relationship \eqref{eqn:linearanalaim}
\begin{equation}
\label{eqn:AppDEFDVM}
\begin{bmatrix}
\overline{\eta} \\\overline{G} 
\end{bmatrix}^{n+1}_j = \matr{E} \begin{bmatrix}
\overline{\eta} \\\overline{G}
\end{bmatrix}^{n}_j.
\end{equation}
The evolution matrix $\matr{E}$ is obtained by applying the numerical method and making use of \eqref{eqn:fourfac} for all quantities.

In Chapter \ref{chp:AnalNumMethod} the evolution matrix for $\text{FEVM}_2$ is given in terms of the flux matrix $\matr{F}$ obtained from the flux approximations of the method. This expression is based on the used SSP Runge-Kutta time stepping. These expressions are summarised in Table \ref{tab:RKstepfactor}. Since $\text{FDVM}_2$ and $\text{FEVM}_2$ both use second-order SSP Runge-Kutta time stepping, these expressions are the same.  

\begin{table}
	\centering
	\begin{tabular}{l  c}
		Method & $\matr{E}$  \T\B \\
		\hline 
		$\text{FDVM}_1$& $\matr{I} - \Delta t\matr{F} $  \T\B \\
		$\text{FDVM}_2$ and $\text{FEVM}_2$ & $ \matr{I}  -\Delta t\matr{F} + \dfrac{1}{2}\Delta t^2\matr{F}^2$  \T\B \\
		$\text{FDVM}_3$& $\matr{I} -\Delta t\matr{F} + \dfrac{1}{2}\Delta t^2 \matr{F}^2 - \dfrac{1}{6}\Delta t^3 \matr{F}^3 $  \T\B \\
		\hline
	\end{tabular}
	\caption{Formula for $\matr{E}$ given $\matr{F}$ determined by the SSP Runge-Kutta timestepping method.}
	\label{tab:RKstepfactor}
\end{table}

The flux matrix $\matr{F}$ from the finite volume methods approximation to the flux terms \eqref{eqn:singleEvolveStep} is
\begin{equation}
 \matr{F} = - \dfrac{\left(1 - e^{-ik\Delta x}\right)}{ \Delta x}\begin{bmatrix}
 \mathcal{F}^{\eta,\eta} & \mathcal{F}^{\eta,G} \\\mathcal{F}^{G,\eta} &\mathcal{F}^{G,G}
 \end{bmatrix}.
 \label{eqn:APPFlux}
\end{equation}
Where $\mathcal{F}^{\eta,\eta} $, $\mathcal{F}^{G,\eta} $, $\mathcal{F}^{G,G} $ depend on the Froude number $Fr = {U}/{\sqrt{gH}}$ and are written in terms of the constituent operators. These expressions written in terms of the constituent operators are the same for all finite volume based methods in the thesis.

The factor $\mathcal{F}^{\eta,G}$ in the flux matrix does not vary with the Froude number and is
\[\mathcal{F}^{\eta,G} = H \mathcal{G}^G. \]
The expressions for the other coefficients of $\matr{F}$ are summarised in Tables \ref{tab:Fnnfactor}-\ref{tab:FGGfactor} for all values of $Fr$ with $\mathcal{F}^{\eta,\eta}$ in Table \ref{tab:Fnnfactor}, $\mathcal{F}^{G,\eta}$ in Table \ref{tab:FGnfactor} and $\mathcal{F}^{G,G}$ in Table \ref{tab:FGGfactor}.



\begin{table}
	\centering
	\begin{tabular}{c  c }
		Froude Number& $\mathcal{F}^{\eta,\eta} $ \T \B\\
		\hline
		$Fr < -1$ & $H \mathcal{G}^{\eta} + U \mathcal{R}^+_{j+1/2}$ \T\B \\
		$-1 \le Fr \le 1$ & $H\mathcal{G}^{\eta}  + \dfrac{U}{2}\left( \mathcal{R}^-_{j+1/2} +  \mathcal{R}^+_{j+1/2}\right)- \dfrac{\sqrt{gH}}{2} \left ( \mathcal{R}^+_{j+1/2} - \mathcal{R}^-_{j+1/2} \right )$ \T\B\\
		$1 < Fr$ & $H \mathcal{G}^{\eta} + U \mathcal{R}^-_{j+1/2}$ \T\B\\
		\hline
	\end{tabular}
	\caption{Factor $\mathcal{F}^{\eta,\eta} $ that multiples $\eta$ in the flux function for $\eta$ for all finite volume based methods.}
	\label{tab:Fnnfactor}
\end{table}
\begin{table}
	\centering
	\begin{tabular}{c  c }
		Froude Number& $\mathcal{F}^{G,\eta} $ \T \B\\
		\hline
		$Fr < -1$ & $UH \mathcal{G}^{\eta} + gH \mathcal{R}^+_{j+1/2}$ \T\B \\
		$-1 \le Fr \le 1$ & $ \dfrac{U\sqrt{gH}}{2} \left( \mathcal{R}^-_{j+1/2} - \mathcal{R}^+_{j+1/2}  \right) + UH\mathcal{G}^{\eta} + \dfrac{gH}{2} \left( \mathcal{R}^-_{j+1/2} +\mathcal{R}^+_{j+1/2} \right)$ \T\B\\
		$1 < Fr$ & $UH \mathcal{G}^{\eta} + gH \mathcal{R}^-_{j+1/2}$ \T\B\\
		\hline
	\end{tabular}
	\caption{Factor $\mathcal{F}^{G,\eta} $ that multiples $\eta$ in the flux function for $G$ for all finite volume based methods.}
	\label{tab:FGnfactor}
\end{table}
\begin{table}
	\centering
	\begin{tabular}{c  c }
		Froude Number& $\mathcal{F}^{G,G} $ \T \B\\
		\hline
		$Fr < -1$ & $U\mathcal{R}^+_{j+1/2}  +  UH \mathcal{G}^G$ \T\B \\
		$-1 \le Fr \le 1$ & $ UH\mathcal{G}^{G} + + \dfrac{U}{2} \left( \mathcal{R}^-_{j+1/2} +\mathcal{R}^+_{j+1/2} \right) - \dfrac{\sqrt{g H}}{2} \left (\mathcal{R}^+_{j+1/2} - \mathcal{R}^-_{j+1/2} \right )$ \T\B\\
		$1 < Fr$ & $U\mathcal{R}^+_{j+1/2}  +  UH \mathcal{G}^G$ \T\B\\
		\hline
	\end{tabular}
	\caption{Factor $\mathcal{F}^{G,G} $ that multiples $G$ in the flux function for $G$ for all finite volume based methods}
	\label{tab:FGGfactor}
\end{table}

Using the appropriate expressions for the reconstruction operators $\mathcal{R}_j$, $\mathcal{R}^+_{j-1/2}$, $\mathcal{R}^-_{j+1/2}$ and the velocity solve operators $\mathcal{G}^\eta$ and $\mathcal{G}^G$ used by the method, one can obtain all the terms of $\matr{F}$ \eqref{eqn:APPFlux} for the finite volume based methods.

The expressions for these fundamental operators of all finite volume based methods are given in Table \ref{tab:Mfactor} for $\mathcal{R}_j$, Table \ref{tab:Rpfactor} for $\mathcal{R}^+_{j-1/2}$, Table \ref{tab:Rmfactor} for $\mathcal{R}^-_{j+1/2}$ and Table \ref{tab:GGfactor} for $\mathcal{G}^G$. Since $\mathcal{G}^\eta = -U\mathcal{G}^G $ we have only provided the table for $\mathcal{G}^G$. Since the perturbations are Fourier modes \eqref{eqn:FourierNode} the reconstruction operator $\mathcal{R}^+_{j+1/2}= e^{ i k\Delta x}\mathcal{R}^+_{j-1/2}$. 

Hence all the terms of the flux matrix $\matr{F}$ \eqref{eqn:APPFlux} can be calculated from these expressions for the operators summarised in Tables \ref{tab:Mfactor} - \ref{tab:GGfactor}. Then the evolution matrix $\matr{E}$ can be calculated from $\matr{F}$ based the SSP Runge-Kutta time-stepping expressions summarised in Table \ref{tab:RKstepfactor}. Thus all the evolution matrices of the finite volume based methods can be obtained, as desired.

Tables \ref{tab:Mfactor}-\ref{tab:GGfactor} also include the operators for $\text{FEVM}_2$ summarising the work in Chapter \ref{chp:AnalNumMethod}. Additionally, the analytic value of the operators for an exact method are also provided as well as the lowest order term of the Taylor series of the difference between the operator in a method and the exact operator. The reported Taylor series results demonstrate that all methods use operators with the appropriate order of accuracy or better. 

\begin{table}
	\centering
	\begin{tabular}{l  c  c}
		&&Lowest Order Term of	\\
		Method& $\mathcal{R}_j$& Method - Exact  \B \\
		\hline 
		Exact &$\dfrac{k\Delta x}{2 \sin \left(k\dfrac{\Delta x}{2}\right)}$ & - \T \B \\
		$\text{FDVM}_1$ & $1$ & $-\dfrac{1}{24}k^2 \Delta x^2$ \T \B \\
		$\text{FDVM}_2$ and $\text{FEVM}_2$& $1$ & $-\dfrac{1}{24}k^2 \Delta x^2$ \T \B \\
		$\text{FDVM}_3$& $\dfrac{26 - 2 \cos\left(k \Delta x\right)}{24}$ & $-\dfrac{3}{640}k^4 \Delta x^4$ \T \B  \\
		\hline	\end{tabular}
	\caption{Factor $\mathcal{R}_j$ from reconstructing the nodal value at the midpoint and the lowest order term of the Taylor series of the factor in the method minus the exact factor for all finite volume based methods.}
	\label{tab:Mfactor}
\end{table}
\begin{table}
	\centering
	\begin{tabular}{l  c  c}
		&&Lowest Order Term of	\\
		Method & $\mathcal{R}^+_{j-1/2}$ & Method - Exact \B\\
		\hline \\
		Exact & $\dfrac{k\Delta x}{2 \sin\left(\dfrac{k \Delta x}{2}\right)}\; e^{-\dfrac{ik\Delta x}{2}}$ & -   \T\B\\
		$\text{FDVM}_1$ & $1$ & $\dfrac{i}{2}k \Delta x$ \T\B \\
		$\text{FDVM}_2$ and $\text{FEVM}_2$& $ 1 - \dfrac{i \sin\left(k\Delta x \right)}{2}$ & $\dfrac{1}{12}k^2 \Delta x^2$  \T\B\\
		$\text{FDVM}_3$& $\dfrac{1}{6}\left({5 + 2e^{-i k {\Delta x}} - e^{i k {\Delta x}}} \right)$ & $\dfrac{i}{12}k^3 \Delta x^3$  \T\B \\
		\hline
	\end{tabular}
	\caption{Factor $\mathcal{R}^+_{j-1/2}$ from the reconstruction of $\eta$ and $G$ at $x_{j+1/2}$ from the ${(j+1)^{th}}$ cell and the lowest order term of the Taylor series of the factor in the method minus the exact factor for all finite volume based methods. }
	\label{tab:Rpfactor}
\end{table}
\begin{table}
	\centering
	\begin{tabular}{l  c  c}
		&&Lowest Order Term of	\\
		Method& $\mathcal{R}^-_{j+1/2}$ &  Method - Exact\B \\
		\hline \\
		Exact & $ \dfrac{k\Delta x}{2 \sin\left(\dfrac{k \Delta x}{2}\right)} \; e^{\dfrac{ik\Delta x}{2}}$ & - \T\B \\
		$\text{FDVM}_1$& $1$ & $-\dfrac{i}{2}k \Delta x$  \T\B \\
		$\text{FDVM}_2$ and $\text{FEVM}_2$& $1 +  \dfrac{i \sin\left(k\Delta x \right)}{2}$ & $\dfrac{1}{12}k^2 \Delta x^2$  \T\B \\
		$\text{FDVM}_3$& $\dfrac{1}{6}\left({5 - e^{-i k {\Delta x}} +2 e^{i k {\Delta x}}} \right)$ & $-\dfrac{i}{12}k^3 \Delta x^3$  \T\B \\
		\hline
	\end{tabular}
	\caption{Factor $\mathcal{R}^-_{j+1/2}$ from the reconstruction of $\eta$ and $G$ at $x_{j+1/2}$ using the ${j^{th}}$ cell and the lowest order term of the Taylor series of the factor in the method minus the exact factor for all finite volume based methods.}
	\label{tab:Rmfactor}
\end{table}
%TABLE TOO BIG
%LandScape it
\begin{landscape}
	\begin{table}
		\centering   
		\begin{tabular}{l  c  c}
			Method& $\mathcal{G}^G$ & Lowest Order Term of Method - Exact \T \\
			\hline \\ 
			Exact &  $\dfrac{3 }{3H + H^3k^2} \dfrac{k\Delta x}{2 \sin\left(\dfrac{k \Delta x}{2}\right)} \;e^{\dfrac{ik\Delta x}{2}}$ & - \\ \\
			$\text{FDVM}_1$& $\dfrac{3\Delta x^2 \left(1 + e^{ik\Delta x}\right)}{6\Delta x^2 H - 2H^3 \left(2\cos\left(k\Delta x\right) - 2\right)}$ & $-\dfrac{6 +H^2k^2}{4H \left(3 + H^2k^2\right)^2}k^2 \Delta x^2$  \\ \\
			$\text{FDVM}_2$& $\dfrac{3 \Delta x^2 \left({1 + e^{ik\Delta x}}\right)}{6 \Delta x^2 H - 2H^3 \left(2\cos\left(k\Delta x\right) - 2\right)}$ & $-\dfrac{6 +H^2k^2}{4H \left(3 + H^2k^2\right)^2}k^2 \Delta x^2$  \\ \\
			& $\dfrac{\Delta x}{6} \left(1 + \dfrac{i \sin\left(k \Delta x\right)}{2} + e^{ik\Delta x}\left[1 - \dfrac{i \sin\left(k \Delta x\right)}{2}\right] \right)$ & \\  $\text{FEVM}_2$ & $\div  \Bigg( H\dfrac{\Delta x}{30} \left[4\cos\left(\dfrac{k \Delta x}{2}\right) - 2\cos\left({k \Delta x}\right) + 8\right] $  & $\dfrac{12 + 5H^2k^2}{40H \left(3 + H^2k^2\right)^2}k^2 \Delta x^2$ \\ &$+ \dfrac{H^3 }{9\Delta x}\left[-16\cos\left(\dfrac{k\Delta x}{2}\right) + 2 \cos\left(k \Delta x\right) + 14\right]    \Bigg)$ & \\ \\
			$\text{FDVM}_3$&  $\dfrac{9 \Delta x^2 \left({-e^{-ik\Delta x} + 9e^{ik\Delta x} - e^{2ik\Delta x} + 9}\right)}{144 \Delta x^2H - 4H^3\left(32\cos\left(k \Delta x\right) -2\cos\left(2k \Delta x\right) - 30\right)}$ & $-\dfrac{243 + 49H^2k^2}{960H\left(3 + H^2k^2\right)^2}k^4 \Delta x^4$ \T \B \\
			\hline
		\end{tabular}
		\caption{Factor $\mathcal{G}^G$ that multiples $G$ given by solving \eqref{eqn:LinConSerreGu0} for $\upsilon_{j+1/2}$ and the lowest order term of the Taylor series of the factor in the method minus the exact factor for all finite volume based methods.}
		\label{tab:GGfactor} 
	\end{table}
\end{landscape}


\section{Finite Difference Methods}
The FDM solve the linearised Serre equations with a horizontal bed in their non-conservative form \eqref{appC:Lin}. The FDM rely on previous time steps as well as the current time step to update the quantities. One particular way of expressing this is with a $4\times4$ evolution matrix $\matr{E}$ producing the following relationship
\begin{equation}
\label{eqn:App4x4Def}
\begin{bmatrix}
{\eta}^{n+1} \\{\mu}^{n+1} \\ {\eta}^{n} \\{\mu}^{n} 
\end{bmatrix}_j = \matr{E} \begin{bmatrix}{\eta}^{n} \\{\mu}^{n} \\ {\eta}^{n-1} \\{\mu}^{n-1} 
\end{bmatrix}_j
\end{equation} 
where the time superscript was brought inside the vector to make clear the time step at which the different elements are placed. Since the FDM are used to calculate ${\eta}^{n+1}_j$ and $\mu^{n+1}_j$ given ${\eta}^{n}_j$, $\mu^{n}_j$, ${\eta}^{n-1}_j$ and $\mu^{n-1}_j$ their evolution matrices have the following structure
\begin{equation}
\label{eqn:App4x4}
\matr{E} = \begin{bmatrix}E_{0,0} & E_{0,1} & E_{0,2} & E_{0,3}\\
E_{1,0} & E_{1,1} & E_{1,2} & E_{1,3}\\
1 & 0 & 0 & 0\\
0 & 1 & 0 & 0\\
\end{bmatrix}.
\end{equation}
Because ${\eta}^{n-1}_j = e^{-i \omega^\pm \Delta t}{\eta}^{n}_j $ and ${\mu}^{n-1}_j = e^{-i \omega^\pm \Delta t}{\mu}^{n}_j$ as $\eta$ and $\mu$ are Fourier modes \eqref{eqn:FourModeApp} then \eqref{eqn:App4x4Def} can be rewritten as
\begin{equation}
\begin{bmatrix}
{\eta} \\{\mu}
\end{bmatrix}^{n+1} _j = \matr{E}^{\left(2 \times 2\right)} \begin{bmatrix}{\eta} \\{\mu}
\end{bmatrix}^{n}_j
\end{equation}  
where $\matr{E}^{\left(2 \times 2\right)}$ is a $2 \times 2$ matrix that depends on the elements of $\matr{E}$ \eqref{eqn:App4x4} in the following way
\begin{equation}
\label{eqn:App4x4D1}
\matr{E}^{\left(2 \times 2\right)} = \begin{bmatrix}E_{0,0} + e^{-i \omega^{\pm }\Delta t} E_{0,2} & E_{0,1} + e^{-i \omega^{\pm }\Delta t} E_{0,3} \\
E_{1,0} + e^{-i \omega^{\pm }\Delta t} E_{1,2} & E_{1,1} + e^{-i \omega^{\pm }\Delta t} E_{1,3} \\
\end{bmatrix}.
\end{equation}

Since the evolution matrix can be represented in two ways, we now state whether $\matr{E}$ or $\matr{E}^{\left(2 \times 2\right)}$ where used in the convergence and dispersion analysis.   

The stability analysis was performed by finding the spectral radius of the naive evolution matrix $\matr{E}$ of the FDM \eqref{eqn:App4x4Def}. The consistency analysis was based on comparing the $2\times2$ evolution matrix $\matr{E}^{\left(2 \times 2\right)}$ \eqref{eqn:App4x4D1} to the exact evolution matrix $e^{i \omega^\pm \Delta t}\matr{I} $. Finally, the dispersion error was based on the eigenvalues of $\matr{E}$ \eqref{eqn:App4x4Def}, this matrix has an additional two eigenvalues beyond the ones given by $e^{i \omega^+ \Delta t}$ and $e^{i \omega^- \Delta t}$ that were ignored. We found the methods had the same stability and dispersion properties when $\matr{E}^{\left(2 \times 2\right)}$ was investigated. We will now present the $4\times4$ evolution matrices for $\mathcal{D}$ and $\mathcal{W}$. Given these matrices the corresponding $2\times 2$ evolution matrix $\matr{E}^{\left(2 \times 2\right)}$ can be calculated using \eqref{eqn:App4x4D1}.



%FD matrices
By using \eqref{eqn:fourierfactor} all the derivative approximations in the finite difference methods $\mathcal{D}$ and $\mathcal{W}$ can be written as operators that are constant in $j$ and $n$ as was done for the finite volume based methods.

The evolution matrix for $\mathcal{D}$ is 
 \begin{equation}
\label{eqn:AppEvolD}
\matr{E} = \begin{bmatrix}
{E}_{0,0} & {E}_{0,1}  & 1 &0 \\ \\
{E}_{1,0} & {E}_{1,1}  & 0 &1 \\
1  & 0  &0 &0 \\
0  & 1  &0 &0 
\end{bmatrix} 
\end{equation}
with
\begin{align*}
&{E}_{0,0} = -  \dfrac{2 i\Delta t }{\Delta x} U\sin\left(k \Delta x\right) , \\ \\
&{E}_{0,1} = -  \dfrac{2 i\Delta t}{\Delta x} H \sin\left(k \Delta x\right),\\ \\
& {E}_{1,0} =-\dfrac{6 gi \Delta x\Delta t}{3 \Delta x^2 -2{H^2} \left( \cos\left(k \Delta x\right) - 1 \right)}{ \sin\left(k \Delta x\right)}, \\\\
& {E}_{1,1} =-\dfrac{2i \Delta t }{\Delta x} U \sin\left(k \Delta x\right).
\end{align*}



While for $\mathcal{W}$ the evolution matrix is 
\begin{equation}
\label{eqn:AppEvolW}
\matr{E} = \begin{bmatrix}
{E}_{0,0} & {E}_{0,1} & 0 & E_{0,3} \\
{E}_{1,0} & E_{1,1} &0 & 1 \\
1&0&0&0\\
0&1&0&0
\end{bmatrix}
\end{equation}
with
\begin{align*}
{E}_{0,0} = &1 - \dfrac{\Delta t}{\Delta x}\left(-\dfrac{6 gi \Delta x\Delta t}{3 \Delta x^2 -2{H^2} \left( \cos\left(k \Delta x\right) - 1 \right)}{ \sin\left(k \Delta x\right)}\right)H\dfrac{i\sin\left(k\Delta x\right)}{2} \\  & - \dfrac{\Delta t}{\Delta x}U\left(i\sin\left(k\Delta x\right) - \dfrac{\Delta t}{\Delta x}U\left(\cos\left(k\Delta x\right) - 1\right)\right), \\ \\
{E}_{0,1} = &- \dfrac{\Delta t}{\Delta x} \Bigg(H\dfrac{i\sin\left(k\Delta x\right)}{2}\left[ 1 -\dfrac{2i \Delta t }{\Delta x} U \sin\left(k \Delta x\right) \right] \\ & -U\left[\dfrac{\Delta t}{\Delta x}H\left(\cos\left(k\Delta x\right) - 1\right)\right] \Bigg),\\ \\
E_{0,3} = &- \dfrac{\Delta t}{\Delta x}H\dfrac{i\sin\left(k\Delta x\right)}{2},  \\ \\
 {E}_{1,0} = &-\dfrac{6 gi \Delta x\Delta t}{3 \Delta x^2 -2{H^2} \left( \cos\left(k \Delta x\right) - 1 \right)}{ \sin\left(k \Delta x\right)}, \\ \\
{E}_{1,1} = &-\dfrac{2i \Delta t }{\Delta x} U \sin\left(k \Delta x\right).
\end{align*}

%[][][]
\section{Consistency Results}
The consistency results for $\text{FDVM}_1$, $\text{FDVM}_2$, $\text{FDVM}_3$, $\mathcal{D}$ and $\mathcal{W}$ are provided here. For $\text{FDVM}_1$, $\text{FDVM}_2$ and $\text{FDVM}_3$ we compare the evolution matrices $\matr{E}$ \eqref{eqn:AppDEFDVM} to the exact evolution matrix $e^{i\omega^\pm \Delta t } \matr{I}$ \eqref{eqn:consistencyTndef}. Since the results are similar for $\omega^-$ and $\omega^+$ we only give the results for $\omega^+$. These results are presented in Tables \ref{tab:EerrFDVM1dxerror} and \ref{tab:EerrFDVM1dterror} for $\text{FDVM}_1$, Table \ref{tab:EerrFDVM2} for $\text{FDVM}_2$ and Tables \ref{tab:EerrFDVM3dxerror} and \ref{tab:EerrFDVM3dterror} for $\text{FDVM}_3$.

For $\mathcal{D}$ and $\mathcal{W}$ we compare the $2\times 2$ evolution matrices $\matr{E}^{\left(2\times 2\right)}$ \eqref{eqn:App4x4D1} to the exact evolution matrix $e^{i\omega^\pm \Delta t } \matr{I}$ \eqref{eqn:consistencyTndef}. Since the results are similar for $\omega^-$ and $\omega^+$ we only give the results for $\omega^+$. The results for $\omega^+$ are presented in Table \ref{tab:EerrD} for $\mathcal{D}$ and Table \ref{tab:EerrW} for $\mathcal{W}$.

\begin{table}
	\centering
	\begin{tabular}{l c c c}
		\hline
		Element & \multicolumn{3}{c}{Lowest Order $\Delta x$ Term of $\matr{E} - e^{i \omega^+ \Delta t} \matr{I}$ for $\text{FDVM}_1$} \T \B \\ 
		\cline{2-4}
		& $Fr < -1$& $-1 < Fr < 1$ & $Fr > 1$ \T \B \\ 
		\hline
		$E_{0,0} -  e^{i \omega^+ \Delta t} $& $ \dfrac{1}{2} k^2 U \Delta t \Delta x$& $ - \dfrac{1}{2} \sqrt{gH} k^2 \Delta t\Delta x$ &$- \dfrac{1}{2} k^2 U \Delta t \Delta x$\T \B \\
		$E_{0,1}$& $\dfrac{1}{2}gHk^2 \Delta t \Delta x $&$ \dfrac{3 + \beta}{4 \beta^2}i k^3\Delta  t\Delta x^2$ & $\dfrac{1}{2}gHk^2 \Delta t \Delta x $ \T \B \\
		$E_{1,0}$& $ - \dfrac{1}{2} \sqrt{gH} k^2 \Delta t\Delta x$&$ - \dfrac{1}{2} \sqrt{gH} k^2 \Delta t\Delta x$ & $ - \dfrac{1}{2} \sqrt{gH} k^2 \Delta t\Delta x$ \T \B  \\
		$E_{1,1} -  e^{i \omega^+ \Delta t}$& $ \dfrac{1}{2} k^2 U \Delta t \Delta x$&$ - \dfrac{1}{2} \sqrt{gH} k^2 \Delta t\Delta x$ & $- \dfrac{1}{2} k^2 U \Delta t \Delta x$  \T\B  \\
		\hline
	\end{tabular}
	\caption{Lowest order $\Delta x$ term of the Taylor series for the elements of $\matr{E} - e^{i \omega^+ \Delta t} \matr{I}$ for $\text{FDVM}_1$. Here $\beta = 3 + k^2 H^2$.}
	\label{tab:EerrFDVM1dxerror} 
\end{table}
\begin{table}
	\centering
	\begin{tabular}{l c}
		\hline 
		Element & \multicolumn{1}{c}{Lowest Order $\Delta t$ Term of $\matr{E} - e^{i \omega^+ \Delta t} \matr{I}$ for $\text{FDVM}_1$}\T\B \\
		\hline 
		$E_{0,0} -  e^{i \omega^+ \Delta t} $ & $\dfrac{\sqrt{3gH \beta} + 3U}{\beta} ik \Delta t$ \T \B \\
		$E_{0,1}$&$ - \dfrac{3}{\beta} ik\Delta t$ \T \B \\
		$E_{1,0}$& $ \left(-gH + \dfrac{3U^2}{\beta}\right)ik \Delta t$ \T \B  \\
		$E_{1,1} -  e^{i \omega^+ \Delta t}$& $\dfrac{\sqrt{3gH \beta} - 3U}{\beta} ik \Delta t$ \T \B  \\
		\hline
	\end{tabular}
	\caption{Lowest order $\Delta t$ term of the Taylor series for the elements of $\matr{E} - e^{i \omega^+ \Delta t} \matr{I}$ for $\text{FDVM}_1$ for all values of $Fr$. Here $\beta = 3 + k^2 H^2$.}
	\label{tab:EerrFDVM1dterror} 
\end{table}
\begin{table}
	\centering
	\begin{tabular}{l c c}
		\hline
		Element & \multicolumn{2}{c}{Lowest Order Terms of $\matr{E} - e^{i \omega^+ \Delta t} \matr{I}$ for $\text{FDVM}_2$}\T \B  \\
		\cline{2-3}
		& $\Delta x$&$\Delta t$\T \B  \\
		\hline
		$E_{0,0} -  e^{i \omega^+ \Delta t} $& $ -\dfrac{i \left(27 + 9H^2k^2 + H^4k^4\right)}{12\beta^2} U k^3 \Delta x^2$ & $\dfrac{\sqrt{3gH \beta} + 3U}{\beta} ik \Delta t$ \T \B  \\
		$E_{0,1}$& $ \dfrac{3 + \beta}{4 \beta^2}i k^3\Delta  t\Delta x^2$ &$ - \dfrac{3}{\beta} ik\Delta t$ \T \B \\
		$E_{1,0}$& $ -\left(gH + \dfrac{3U^2}{\beta} + \dfrac{9U^2}{\beta^2}\right)  \dfrac{k^3}{12}\Delta t\Delta x^2$ &$ \left(-gH + \dfrac{3U^2}{\beta}\right)ik \Delta t$ \T \B  \\
		$E_{1,1} -  e^{i \omega^+ \Delta t}$& $ \dfrac{-9 + H^2k^2\beta}{\beta^2} \dfrac{k^3}{12} i U \Delta t\Delta x^2$ & $\dfrac{\sqrt{3gH \beta} - 3U}{\beta} ik \Delta t$ \T \B  \\
		\hline 
	\end{tabular}
	\caption{Lowest order terms of the Taylor series for the elements of $\matr{E} - e^{i \omega^+ \Delta t} \matr{I}$ for $\text{FDVM}_2$ for all values of $Fr$. Here $\beta = 3 + k^2 H^2$.}
	\label{tab:EerrFDVM2} 
\end{table}

\begin{table}
	\centering
	\begin{tabular}{l c c c}
		\hline
		Element & \multicolumn{3}{c}{Lowest Order $\Delta x$ Term of $\matr{E} - e^{i \omega^+ \Delta t} \matr{I}$ for $\text{FDVM}_3$} \T \B \\
		\cline{2-4} 
		& $Fr < -1$& $-1 < Fr < 1$ & $Fr > 1$ \T \B \\
		\hline
		$E_{0,0} -  e^{i \omega^+ \Delta t} $& $\dfrac{1}{12} k^4 U \Delta t \Delta x^3$& $ - \dfrac{1}{12} \sqrt{gH} k^4 \Delta t\Delta x^3$ &$ -\dfrac{1}{12} k^4 U \Delta t \Delta x^3$  \T \B  \\
		$E_{0,1}$& $\dfrac{1}{4 \beta}iUk^5 \Delta t^2 \Delta x^3 $&$ \dfrac{\sqrt{gH}}{4 \beta}i k^5\Delta  t ^2\Delta x^3$ & $-\dfrac{1}{4 \beta}iUk^5 \Delta t^2 \Delta x^3 $  \T \B  \\
		$E_{1,0}$& $\dfrac{1}{12} gHk^4 \Delta t^2 \Delta x^3 $&$ - \dfrac{1}{12} \sqrt{gH} k^4 \Delta t\Delta x^3$ & $-\dfrac{1}{12} gHk^4 \Delta t^2 \Delta x^3 $  \T \B \\
		$E_{1,1} -  e^{i \omega^+ \Delta t}$& $ \dfrac{1}{12} k^4 U \Delta t \Delta x^3$&$ - \dfrac{1}{12} \sqrt{gH} k^4 \Delta t\Delta x^3$ & $ -\dfrac{1}{12} k^4 U \Delta t \Delta x^3$ \T \B  \\ 
		\hline
	\end{tabular}
	\caption{Lowest order $\Delta x$ term of the Taylor series for the elements of $\matr{E} - e^{i \omega^+ \Delta t} \matr{I}$ for $\text{FDVM}_3$. Here $\beta = 3 + k^2 H^2$.}
	\label{tab:EerrFDVM3dxerror} 
\end{table}
\begin{table}
	\centering
	\begin{tabular}{l  c}
		\hline
		Element & \multicolumn{1}{c}{Lowest Order $\Delta t$ Term of $\matr{E} - e^{i \omega^+ \Delta t} \matr{I}$ for $\text{FDVM}_3$} \T \B \\
		\hline  
		$E_{0,0} -  e^{i \omega^+ \Delta t} $ & $\dfrac{\sqrt{3gH \beta} + 3U}{\beta} ik \Delta t$ \T \B  \\
		$E_{0,1}$&$ - \dfrac{3}{\beta} ik\Delta t$ \T \B   \\
		$E_{1,0}$& $ \left(-gH + \dfrac{3U^2}{\beta}\right)ik \Delta t$ \T \B  \\
		$E_{1,1} -  e^{i \omega^+ \Delta t}$& $\dfrac{\sqrt{3gH \beta} - 3U}{\beta} ik \Delta t$ \T \B \\ 
		\hline
	\end{tabular}
	\caption{Lowest order $\Delta t$ term of the Taylor series for the elements of $\matr{E} - e^{i \omega^+ \Delta t} \matr{I}$ for $\text{FDVM}_3$ for all values of $Fr$. Here $\beta = 3 + k^2 H^2$.}
	\label{tab:EerrFDVM3dterror} 
\end{table}

%USE ELEMENTS of E_2
\begin{table}
	\centering
	\begin{tabular}{l  c c}
		\hline
		Element & \multicolumn{2}{c}{Lowest Order Terms of $\matr{E}^{\left(2\times 2\right)} - e^{i \omega^+ \Delta t} \matr{I}$ for $\mathcal{D}$} \T \B \\
		\cline{2-3}
		& $\Delta x$&$\Delta t$ \T \B \\
		\hline
		${E}^{\left(2\times 2\right)}_{0,0} -  e^{i \omega^+ \Delta t} $&$\dfrac{ik^3}{3} U \Delta t \Delta x^2$ & $ \sqrt{\dfrac{3gH}{\beta}} 2ik \Delta t $ \T \B  \\
		${E}^{\left(2\times 2\right)}_{0,1}$& $\dfrac{iHk^3}{3} \Delta t \Delta x^2$ & $-2Hi k \Delta t$ \T \B  \\
		${E}^{\left(2\times 2\right)}_{1,0}$& $ \dfrac{ig \left(3 + \beta\right)}{2\beta^2} k^3\Delta t \Delta x^2$ &$ -\dfrac{6igk}{\beta} \Delta t$ \T \B  \\
		${E}^{\left(2\times 2\right)}_{1,1} -  e^{i \omega^+ \Delta t}$& $\dfrac{ik^3}{3} U \Delta t \Delta x^2$ & $ \sqrt{\dfrac{3gH}{\beta}} 2ik \Delta t $ \T \B  \\  \hline
	\end{tabular}
	\caption{Lowest order terms of the Taylor series for the elements of $\matr{E}^{\left(2\times 2\right)} - e^{i \omega^\pm \Delta t} \matr{I}$ for $\mathcal{D}$ for all values of $Fr$. Here $\beta = 3 + k^2 H^2$.}
	\label{tab:EerrD} 
\end{table}
\begin{table}
	\centering
	\begin{tabular}{l  c c}
		\hline
		Element & \multicolumn{2}{c}{Lowest Order Terms of $\matr{E}^{\left(2\times 2\right)} - e^{i \omega^+ \Delta t} \matr{I}$ for $\mathcal{W}$} \T \B \\
		\cline{2-3} 
		& $\Delta x$&$\Delta t$ \T \B \\
		\hline 
		$E^{\left(2\times 2\right)}_{0,0} -  e^{i \omega^+ \Delta t} $&  $\dfrac{ik^3}{6} U \Delta t \Delta x^2$ & $ \sqrt{\dfrac{3gH}{\beta}} ik \Delta t $ \T \B \\
		$E^{\left(2\times 2\right)}_{0,1}$& $\dfrac{iHk^3}{6} \Delta t \Delta x^2$ &  $-Hi k \Delta t$ \T \B \\
		$E^{\left(2\times 2\right)}_{1,0}$& $ \dfrac{ig \left(3 + \beta\right)}{2\beta^2} k^3\Delta t \Delta x^2$ &  $ -\dfrac{6igk}{\beta} \Delta t$ \T \B  \\
		$E^{\left(2\times 2\right)}_{1,1} -  e^{i \omega^+ \Delta t}$& $\dfrac{ik^3}{3} U \Delta t \Delta x^2$ & $ \sqrt{\dfrac{3gH}{\beta}} 2ik \Delta t $ \T \B  \\  \hline
	\end{tabular}
	\caption{Lowest order terms of the Taylor series for the elements of $\matr{E}^{\left(2\times 2\right)} - e^{i \omega^+ \Delta t} \matr{I}$ for $\mathcal{W}$ for all values of $Fr$. Here $\beta = 3 + k^2 H^2$.}
	\label{tab:EerrW} 
\end{table}
