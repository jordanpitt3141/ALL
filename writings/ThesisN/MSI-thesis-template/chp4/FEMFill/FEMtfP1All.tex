\documentclass[12pt]{article}
\usepackage{amsmath}
\usepackage{ amssymb }
\usepackage{breqn}
\begin{document}

\section{Elliptic Equation}
The linearised elliptic equation is

\[G = H\upsilon -\frac{H^3}{3} \left(\frac{\partial^2 \upsilon}{\partial x^2}\right)\]

Taking the weak version of this we get that

\[\int_{\Omega}G  v \; dx = H \int_{\Omega} \upsilon v \; dx -\frac{H^3}{3}  \int_{\Omega} \frac{\partial^2 \upsilon}{\partial x^2} v \; dx\]

\[\int_{\Omega}G  v \; dx = H \int_{\Omega} \upsilon v \; dx  + \frac{H^3}{3}  \int_{\Omega} \frac{\partial \upsilon}{\partial x} \frac{\partial v}{\partial x} \; dx\]

In particular for the basis function $\phi_j$ we must have

\[\int_{\Omega}G  \phi_j \; dx = H \int_{\Omega} \upsilon \phi_j \; dx  + \frac{H^3}{3}  \int_{\Omega} \frac{\partial \upsilon}{\partial x} \frac{\partial \left(\phi_j\right)}{\partial x} \; dx\]

We use the FEM discretisation from []

\begin{equation*}
G = \sum_j G^+_{j-1/2}\psi^+_{j-1/2}  + G^-_{j+1/2}\psi^-_{j+1/2}
\end{equation*}

and

\begin{equation}
\upsilon = \sum_j \upsilon_{j-1/2}\phi_{j-1/2} + \upsilon_{j}\phi_{j}  + \upsilon_{j+1/2}\phi_{j+1/2}
\end{equation}

Now for our evolution equations we only need to to get the errors introduced from our calculation of $\upsilon_{j+1/2}$ and $\upsilon_{j}$, as we can get $\upsilon_{j-1/2}$ from just a shift. We previously demonstrated how the coefficient matrices are calculated for the FEM so we now just skip ahead to give the equations. 

The FEM gives

\begin{multline}
\sum_j \frac{\Delta x}{2}\begin{bmatrix} \frac{1}{3}G^+_{j -1/2} \\ \frac{2}{3} G^+_{j -1/2}+ \frac{2}{3} G^-_{j +1/2} \\ \frac{1}{3} G^-_{j +1/2} \end{bmatrix} = \\\sum_j \left(H\frac{\Delta x}{2}\begin{bmatrix} \frac{4}{15} &\frac{2}{15} &-\frac{1}{15} \\\frac{2}{15} &\frac{16}{15} &\frac{2}{15}  \\-\frac{1}{15} &\frac{2}{15} &\frac{4}{15} \end{bmatrix} + \frac{2 H^3 }{3\Delta x}\begin{bmatrix} \frac{7}{6} &-\frac{4}{3} &\frac{1}{6} \\-\frac{4}{3} &\frac{8}{3} &-\frac{4}{3}  \\\frac{1}{6} &-\frac{4}{3} &7  \end{bmatrix} \right) \begin{bmatrix} u_{j -1/2} \\u_{j} \\ u_{j +1/2} \end{bmatrix}
\end{multline}

\begin{multline}
\sum_j \frac{\Delta x}{6}\begin{bmatrix} G^+_{j -1/2} \\2 G^+_{j -1/2}+2 G^-_{j +1/2} \\ G^-_{j +1/2} \end{bmatrix} = \\\sum_j \left(H\frac{\Delta x}{30}\begin{bmatrix} 4 &2 &-1 \\2 &16 &2  \\-1 &2 &4 \end{bmatrix} + \frac{H^3 }{9\Delta x}\begin{bmatrix} 7 &-8 &1  \\-8 &16 &-8  \\1 &-8 &7  \end{bmatrix} \right) \begin{bmatrix} u_{j -1/2} \\u_{j} \\ u_{j +1/2} \end{bmatrix}
\end{multline}


\begin{multline}
\sum_j \frac{\Delta x}{6}\begin{bmatrix} G^+_{j -1/2} \\2 G^+_{j -1/2}+2 G^-_{j +1/2} \\ G^-_{j +1/2} \end{bmatrix} = \\\sum_j \left(H\frac{\Delta x}{30}\begin{bmatrix} 4 &2 &-1 \\2 &16 &2  \\-1 &2 &4 \end{bmatrix} + \frac{H^3 }{9\Delta x}\begin{bmatrix} 7 &-8 &1  \\-8 &16 &-8  \\1 &-8 &7  \end{bmatrix} \right) \begin{bmatrix} e^{-ik\frac{\Delta x}{2}} \\1 \\ e^{ik\frac{\Delta x}{2}} \end{bmatrix} u_j
\end{multline}


\begin{multline}
\sum_j \frac{\Delta x}{6}\begin{bmatrix} e^{-ik\Delta x} \mathcal{R}^+_2 \\2 e^{-ik\Delta x} \mathcal{R}^+_2 +2 \mathcal{R}^-_2 \\ \mathcal{R}^-_2 \end{bmatrix} G_j = \\\sum_j \Bigg(H\frac{\Delta x}{30}\begin{bmatrix} -5i\sin\left(k \frac{\Delta x}{2}\right) + 3\cos\left(k \frac{\Delta x}{2}\right) + 2\\16 + 4 \cos\left(k \frac{\Delta x}{2}\right) \\ 5i\sin\left(k \frac{\Delta x}{2}\right) + 3\cos\left(k \frac{\Delta x}{2}\right) + 2 \end{bmatrix} \\+ \frac{H^3 }{9\Delta x}\begin{bmatrix} -6i\sin\left(k \frac{\Delta x}{2}\right) + 8\cos\left(k \frac{\Delta x}{2}\right) - 8 \\ 32 \sin^2 \left(\frac{k \Delta x}{4}\right) \\ 6i\sin\left(k \frac{\Delta x}{2}\right) + 8\cos\left(k \frac{\Delta x}{2}\right) - 8 \end{bmatrix}  \Bigg) u_j
\end{multline}

assemble into complete matrices so we get, change notation to denote we sum the corresponding matrices together

\begin{multline}
\sum_j \frac{\Delta x}{6}\begin{bmatrix} e^{-ik\Delta x} \mathcal{R}^+_2 + e^{-ik\Delta x} \mathcal{R}^-_2   \\2 e^{-ik\Delta x} \mathcal{R}^+_2 +2 \mathcal{R}^-_2 \\ \mathcal{R}^-_2 + \mathcal{R}^+_2 \end{bmatrix} G_j = \\\sum_j \Bigg(H\frac{\Delta x}{30}\begin{bmatrix}e^{-ik\frac{\Delta x}{2}}\left(4\cos\left(\frac{k\Delta x}{2}\right) -2 \cos\left(k \Delta x\right)+8\right)\\16 + 4 \cos\left(k \frac{\Delta x}{2}\right) \\ e^{ik\frac{\Delta x}{2}}\left(4\cos\left(\frac{k\Delta x}{2}\right) -2 \cos\left(k \Delta x\right)+8\right)\end{bmatrix} \\+ \frac{H^3 }{9\Delta x}\begin{bmatrix} e^{-ik\frac{\Delta x}{2}}\left(-16\cos\left(\frac{k\Delta x}{2}\right) + 2\cos\left(k \Delta x\right) + 14\right) \\ 32 \sin^2 \left(\frac{k \Delta x}{4}\right) \\ e^{ik\frac{\Delta x}{2}}\left(-16\cos\left(\frac{k\Delta x}{2}\right) + 2\cos\left(k \Delta x\right) + 14\right) \end{bmatrix}  \Bigg) u_j
\end{multline}

\begin{multline}
\sum_j \frac{\Delta x}{6} \begin{bmatrix}  \mathcal{R}^+_2 +  \mathcal{R}^-_2   \\2 \\ \mathcal{R}^-_2 + \mathcal{R}^+_2 \end{bmatrix}^T \begin{bmatrix} e^{-ik\Delta x} \\ 1 \\ 1\end{bmatrix} G_j = \\\sum_j \Bigg(H\frac{\Delta x}{30}\begin{bmatrix}4\cos\left(\frac{k\Delta x}{2}\right) -2 \cos\left(k \Delta x\right)+8\\16 + 4 \cos\left(k \frac{\Delta x}{2}\right) \\4\cos\left(\frac{k\Delta x}{2}\right) -2 \cos\left(k \Delta x\right)+8\end{bmatrix} \\+ \frac{H^3 }{9\Delta x}\begin{bmatrix} -16\cos\left(\frac{k\Delta x}{2}\right) + 2\cos\left(k \Delta x\right) + 14 \\ 32 \sin^2 \left(\frac{k \Delta x}{4}\right) \\-16\cos\left(\frac{k\Delta x}{2}\right) + 2\cos\left(k \Delta x\right) + 14 \end{bmatrix}  \Bigg) \begin{bmatrix} e^{-ik\frac{\Delta x}{2}} \\ 1 \\ e^{ik\frac{\Delta x}{2}} \end{bmatrix}  u_j
\end{multline}

The equation for $u_{j+1/2}$ is then

\begin{multline}
\frac{\Delta x}{6}\left(\mathcal{R}^+_2 +\mathcal{R}^-_2\right)  G_j = \\ \Bigg(H\frac{\Delta x}{30} \left(4\cos\left(\frac{k\Delta x}{2}\right) -2 \cos\left(k \Delta x\right) + 8 \right) \\+ \frac{H^3 }{9\Delta x}\left(-16\cos\left(\frac{k\Delta x}{2}\right) 2 \cos\left(k \Delta x\right) + 14 \right)   \Bigg)  e^{ik\frac{\Delta x}{2}} u_j. 
\end{multline}








\end{document}
