\chapter{Finite Element Volume Method}
%[][][][][] Leveque Reference
\label{chp:HFVMMethod}
In this chapter the notation for the numerical grids is introduced, followed by a detailed description of the second-order Finite Element Volume Method (FEVM).

A variety of numerical methods have been used to solve the Serre equations; from complete finite difference methods \cite{Cienfuegos-etal-2006-1217,El-etal-2006} and finite element methods \cite{Mitsotakis-etal-2014,Li-2014-169,Mitsotakis-etal-2017} to combinations of finite difference and finite volume methods \cite{Hank-etal-2010-2034,Zoppou-etal-2017}. Splitting techniques have also been employed, most commonly to split the Serre equations into their non-linear and dispersive parts; resulting in an elliptic operator for the dispersive part and the SWWE for the non-linear part \cite{Bradford-Sanders-2002-953,Dutykh-etal-2013-761,Filippini-etal-2016-381}. 

Numerical methods that make use of the conservation law form of the Serre equations \eqref{eqn:FullSerreCon} \cite{Hank-etal-2010-2034,Li-2014-169,Zoppou-etal-2017} are the most promising for the two dimensional Serre equations with variable bathymetry. The primary reason for this is that these methods are robust and extend well to unstructured meshes with complex geometries which are the meshes most commonly used for modelling physical scenarios. Additionally, to properly handle the elliptic operator produced by the non-linear and dispersive splitting requires overly restrictive assumptions about the smoothness of the physical quantities, particularly the water depth. 

An extension of the Finite Difference Volume Methods (FDVM) \cite{Hank-etal-2010-2034,Zoppou-etal-2017} that uses a finite element method in place of the finite difference method was developed during the research underpinning this thesis. This second-order FEVM which will be referred to as $\text{FEVM}_2$ was a main objective of this thesis; it consists of two parts a Finite Element Method (FEM) to solve \eqref{defn:SerreEqnConservedQuantity1} and a Finite Volume Method (FVM) to solve \eqref{eqn:FullSerreCon}, hence its name. Making use of a FEM and a FVM results in a numerical method with a number of desirable properties: it is accurate in the presence of steep gradients in the free surface \cite{Pitt-2018-61}, it accurately models the wetting and drying of beds and the FEM provides a complete profile of the horizontal velocity inside a cell. This last point indicates that this method is the ideal variant of the finite volume based methods \cite{Zoppou-etal-2017} for solving the two-dimensional Serre equations on unstructured meshes with parallelised code.
 
 
In addition to the $\text{FEVM}_2$, the first- and second-order FDVM of \citet{Hank-etal-2010-2034} and \citet{Zoppou-etal-2017} were reproduced and will be referred to as $\text{FDVM}_1$ and $\text{FDVM}_2$ respectively. Furthermore, the third-order $\text{FDVM}_3$ was implemented during the research underpinning this thesis. Finally, the second-order naive finite difference method \cite{Pitt-J-2014} and the finite difference method of \citet{El-etal-2006} which are referred to as $\mathcal{D}$ and $\mathcal{W}$ respectively were reproduced. Descriptions of all of these methods have previously been published \cite{Zoppou-etal-2017,Pitt-2018-61} and therefore, are omitted from this thesis.


\section{Notation for Numerical Grids}

In the $\text{FEVM}_2$ time is discretised into time levels separated by a constant duration $\Delta t$ while space is discretised into cells of constant width $\Delta x$. The FEVM can be extended to allow for varying $\Delta t$ and $\Delta x$ values, with this description restricted to the constant case for simplicity. The notation for time is quite simple; from an initial time $t^0$ the $n^{th}$ time level where $n \in \mathbb{N}$ is
\begin{align*}
t^n &= t^0 + n \Delta t.
\end{align*}
The goal of $\text{FEVM}_2$ is to update the quantities at the current time level $t^n$ to the next time level $t^{n+1}$ by solving the Serre equations. 

The notation for space is more complicated as multiple locations inside a cell require definition. The cells are defined by their midpoints; which are given from a starting location $x_0$, thus the midpoint of the $j^{th}$ cell where $j \in \mathbb{N}$ is
\begin{align*}
x_j &= x_0 + j \Delta x.
\end{align*}
Other points inside the $j^{th}$ cell can be defined in relation to the midpoint so that 
\begin{align*}
x_{j + s} &= x_j + s \Delta x
\end{align*}
where $s \in \left[-\frac{1}{2} , \frac{1}{2}\right] \subset\mathbb{R}$, although for the remainder of the thesis only rational values of $s$ are required. Utilising the spatial notation the $j^{th}$ cell spans $\left[x_{j -1/2},x_{j + 1/2}\right]$. These discretisations in space and time result in the grids displayed in Figure \ref{fig:NumericalGrid}.

\begin{figure}
	\centering
	\includegraphics[width=0.8\textwidth]{./chp3/figures/Discretisation.pdf}
	\caption{Diagram of the time levels $t^{n-1}$, $t^n$ and $t^{n+1}$ at which the numerical solution of the Serre equations will be calculated. Also shown is the $j^{th}$ cell with midpoint $x_{j}$ spanning $x_{j-1/2}$ to $x_{j+1/2}$ which is a volume of the FVM and an element of the FEM.}
	\label{fig:NumericalGrid}
\end{figure}

The grid notation in space and time naturally extends to the quantities of interest, for example, for a general quantity $q$
\begin{equation*}
q^n_j = q(x_j ,t^n) 
\end{equation*}
which are the nodal values of $q$. Since the FEVM uses a FVM, a definition of the cell averages of quantities are also required. The average of a quantity $q$ over the $j^{th}$ cell at time level $t^n$ is
\begin{equation*}
\overline{q}_j^n = \frac{1}{\Delta x} \int_{x_{j-1/2}}^{x_{j+1/2}} q(x,t^n) \; dx.
\end{equation*}

In the FEVM the quantities are reconstructed at various points inside the cell using the adjacent cell average values. At the cell edges $x_{j\pm1/2}$, two reconstructions are possible from each of the neighbouring cells, these two possible reconstructions are distinguished using superscripts. For example, for the cell edge $x_{j+1/2}$ and a general quantity $q$, there is the reconstructed value $q^-_{j+1/2}$ from the $j^{th}$ cell and the reconstructed value $q^+_{j+1/2}$ from the $(j+1)^{th}$ cell.

\section{Structure Overview}
\label{sec:StructOverview}
%define w,b
The description of the FEVM begins with an overview of the evolution step, followed by a detailed explanation for each of its components. 

The evolution step from $t^n$ to $t^{n+1}$ begins with the cell averages for the height of the water above the bed $h$, the absolute location of the free-surface $w$ and the conserved quantity $G$ at time $t^n$. Additionally, all the nodal values of the bed profile $b$ are known. These quantities are written as vectors capturing their values from the $0^{th}$ to the $m^{th}$ cell in the following way
\begin{align*} \overline{\vecn{q}}^n = \begin{bmatrix} \TM
\overline{q}_0^n \\\overline{q}_1^n \\ \vdots \\ \overline{q}_m^n \BM\end{bmatrix} & , & \vecn{b} = \begin{bmatrix} \TM
b_{0} \\ b_{1} \\ \vdots \\b_{m} \BM
\end{bmatrix} \\
\end{align*}
where $q$ is a generic quantity representing the vectors for $h$, $G$ and $w$. The evolution step proceeds by (i) reconstructing the quantities over the cell, (ii) calculating the fluid velocity, (iii) approximating the flux, (iv) approximating the source term, (v) updating the cell averages and then (vi) applying second-order Strong Stability Preserving (SSP) Runge-Kutta time stepping.

\begin{enumerate}[(i)]
	\item Reconstruction: The locations for the reconstruction of all the quantities in the $j^{th}$ cell are displayed in Figure \ref{fig:ReconLocs}. For the $j^{th}$ cell the quantities $h$, $w$ and $G$ are reconstructed at $x_{j-1/2}$, $x_{j}$ and $x_{j+1/2}$ from their cell average values using the second-order reconstruction operators $\mathcal{R}^+_{j-1/2}$, $\mathcal{R}_{j}$ and $\mathcal{R}^-_{j+1/2}$ respectively. The bed profile $b$ in the $j^{th}$ cell is reconstructed at $x_{j-1/2}$, $x_{j-1/6}$, $x_{j+1/6}$ and $x_{j+1/2}$ from its nodal values using the fourth-order reconstruction operators $\mathcal{B}_{j-1/2}$, $\mathcal{B}_{j-1/6}$, $\mathcal{B}_{j+1/6}$ and $\mathcal{B}_{j+1/2}$ respectively. 
	\begin{figure}
		\centering
		\includegraphics[width=0.8\textwidth]{./chp3/figures/FEVMRecon.pdf}
		\caption{The locations of the reconstructions for $h$, $w$, $G$ (\circlet{blue}) and $b$ (\crosst{red}) inside the $j^{th}$ cell.}
		\label{fig:ReconLocs}
	\end{figure}
	
	Therefore, the following reconstructions are performed
	\begin{align*}
	&q^\pm_{j\pm 1/2} = \mathcal{R}^\pm_{j \pm1/2} \left(\overline{\vecn{q}}^n\right), &&b_{j\pm 1/2} = \mathcal{B}_{j\pm 1/2} \left(\vecn{b}\right),  \\
	&q_{j} = \mathcal{R}_{j} \left(\overline{\vecn{q}}^n\right), &&b_{j\pm 1/6} = \mathcal{B}_{j\pm 1/6}  \left(\vecn{b}\right)
	\end{align*}
	where $q$ is a generic quantity representing $h$, $w$ and $G$. Note that the time superscript is omitted from the reconstructed quantities to simplify the notation.	Performing the reconstruction operations on all the cells produces the following vectors; $\vecn{\hat{h}}$, $\vecn{\hat{w}}$, $\vecn{\hat{G}}$ and $\vecn{\hat{b}}$ at time $t^n$ which are given by
	\begin{align*}\vecn{\hat{q}} = \begin{bmatrix} \TM
	q^+_{-1/2} \\ q_0 \\ q^-_{1/2} \\ \vdots  \\ q^-_{m+1/2} \BM \end{bmatrix} & , & \vecn{\hat{b}} = \begin{bmatrix} \TM
	b_{-1/2} \\ b_{-1/6} \\ b_{1/6}  \\b_{1/2}  \\ \vdots \\ b_{m+1/2} \BM
	\end{bmatrix}
	\end{align*}
	where $q$ is a generic quantity demonstrating the vectors for $h$, $w$ and $G$. Note that since the domain contains all the cells from the $0^{th}$ cell to the $m^{th}$ cell that $q^+_{-1/2}$ and $q^-_{m+1/2}$ are the reconstructed values of $q$ at the left and right boundaries of the domain, respectively.	
	\item Fluid Velocity: The remaining unknown quantity, the depth-averaged horizontal velocity of the fluid column, $u$ is calculated at $x_{j-1/2}$, $x_j$ and $x_{j+1/2}$ in each cell by solving \eqref{defn:SerreEqnConservedQuantity1} with a second-order FEM. The solution of the FEM for $u$ is denoted by the function $\mathcal{G}$, which takes $\vecn{\hat{h}}$, $\vecn{\hat{G}}$ and $\vecn{\hat{b}}$ as inputs and thus,
	\begin{equation*}
	\vecn{\hat{u}} = 
	\begin{bmatrix} \TM	u_{-1/2} \\ u_0 \\ u_{1/2} \\ \vdots \\ u_{m+1/2} \BM
	\end{bmatrix} = \mathcal{G}\left( \vecn{\hat{h}}, \vecn{\hat{G}},\vecn{\hat{b}} \right).
	\end{equation*}
	\item Flux Across Cell Interfaces: The temporally averaged fluxes $F^n_{j-1/2}$ and $F^n_{j+1/2}$ across the cell boundaries $x_{j-1/2}$ and $x_{j+1/2}$ are calculated using $\mathcal{F}_{j-1/2}$ and $\mathcal{F}_{j+1/2}$ hence,
		\begin{align*}	
		F^n_{j\pm 1/2} &=\mathcal{F}_{j\pm 1/2} \left( \vecn{\hat{h}}, \vecn{\hat{G}},\vecn{\hat{b}}, \vecn{\hat{u}}  \right).
		\end{align*}
	\item Source Terms: The contribution of the source term to the cell average of a quantity over a time step $S^n_{j}$ is calculated using the operator $\mathcal{S}_j$
	\begin{equation*}	
	S^n_{j} =\mathcal{S}_{j} \left( \vecn{\hat{h}},\vecn{\hat{w}},\vecn{\hat{b}}, \vecn{\hat{u}}  \right).
	\end{equation*}
	\item Update Cell Averages: The cell average values are updated from time $t^n$ to $t^{n+1}$ with a forward Euler approximation, resulting in a method that is second-order accurate in space and first-order in time.
	\item Second-Order SSP Runge-Kutta Method: Steps (i)-(v) are repeated and an SSP Runge-Kutta time stepping method is employed to obtain $\overline{\vecn{h}}$ and $\overline{\vecn{G}}$ at $t^{n+1}$ with second-order accuracy in space and time.
\end{enumerate}


\subsection{Reconstruction}
\label{subsec:Reconstruction}
The details for the reconstruction of $h$, $w$, $G$ and $b$ in the $j^{th}$ cell at the locations shown in Figure \ref{fig:ReconLocs} are now provided. For $h$, $w$ and $G$ the reconstructions are performed from the cell averages, while $b$ is reconstructed from the nodal values. For simplicity it is assumed that the mesh is structured. The reconstruction methods described below can be extended to unstructured meshes through generalisations of the employed interpolation techniques \cite{Preneter-1975,atkinson-1978}.

\subsubsection{Reconstruction of the $h$, $w$ and $G$}
The quantities $h$, $w$ and $G$ are reconstructed with piecewise linear functions over a cell from neighbouring cell averages. Since $h$, $w$ and $G$ use the same reconstruction operators, a general quantity $q$ will be used to demonstrate the operators. For the $j^{th}$ cell the values of $q$ are reconstructed at $x_{j-1/2} $, $x_{j} $ and $x_{j+1/2}$ in the following way
%cite someone 
\begin{subequations}
	\begin{align}
	q^+_{j-1/2} & = \mathcal{R}^+_{j-1/2} \left(\overline{\vecn{q}}\right) = \overline{q}_j - \dfrac{\Delta x}{2} d_j, \\
	q_{j} &= \mathcal{R}_{j} \left(\overline{\vecn{q}}\right) =\overline{q}_j ,\\
	q^-_{j+1/2} &= \mathcal{R}^-_{j+1/2} \left(\overline{\vecn{q}}\right) = \overline{q}_j + \dfrac{\Delta x}{2} d_j
	\end{align}
	\label{eqn:ReconforhwG}
\end{subequations}
where 
\begin{equation}
d_j = \text{minmod}\left(\theta \dfrac{\overline{q}_j -\overline{q}_{j-1} }{\Delta x}, \dfrac{\overline{q}_{j+1} -\overline{q}_{j-1} }{2\Delta x}, \theta\dfrac{\overline{q}_{j+1} -\overline{q}_{j} }{\Delta x}\right)
\label{eqn:slopehGrecon}
\end{equation}
with $\theta \in \left[1,2\right]$. The choice of the $\theta$ parameter alters the diffusion introduced by the reconstruction. When $\theta =1$ the reconstruction introduces the most diffusion and is equivalent to the minmod reconstruction \cite{Roe-1986-337}. When $\theta = 2$ the reconstruction introduces the least diffusion and is equivalent to the monotized central reconstruction \cite{VanLeer-1977-276}.

\begin{defn}
The minmod function
\begin{equation*}
\text{minmod}\left(a_0,a_1,\dots\right) := \left\lbrace \begin{array}{l l}
\min\left\lbrace a_i\right\rbrace & a_i > 0 \; \text{for all } i \\
\max\left\lbrace a_i\right\rbrace & a_i < 0  \; \text{for all } i \\
0 & \text{otherwise}\\
\end{array} \right.
\end{equation*}
takes a list of $a_i \in \mathbb{R}$. If all the inputs have the same sign then minmod returns the input with smallest absolute value, otherwise it returns zero. 
\end{defn}
The non-linear limiting used to calculate $d_j$ ensures that the reconstruction of $h$, $w$ and $G$ inside the cell is Total Variation Diminishing (TVD) \cite{Harten-1983-357}, hence it does not introduce non-physical oscillations and is therefore stable. The TVD property of this reconstruction is achieved by constraining the slope $d_j$ to zero near local extrema, resulting in a piecewise constant reconstruction which is TVD. Away from local extrema $d_j$ will be the gradient with the smallest absolute value, producing a second-order accurate reconstruction. The flexibility offered by the free parameter $\theta$ and its successful use when solving the SWWE with ANUGA \cite{ANUGA} makes the generalised minmod limiter an attractive option for the $\text{FEVM}_2$.

The reconstruction operator $\mathcal{R}_{j} $ is second-order accurate, due to the second-order accuracy of the midpoint quadrature rule \cite{Davis-Rabinowitz-1984}
\begin{equation}
\overline{q}^n_j = \frac{1}{\Delta x} \int_{x_{j-1/2}}^{x_{j+1/2}} q(x,t^n) \; dx = q_j^n + \mathcal{O}\left(\Delta x^2\right).
\end{equation}


\subsubsection{Reconstruction of the Bed Profile}
The $\text{FEVM}_2$ requires a reconstruction of the bed profile that is at least second-order accurate for $b$, $\partial b / \partial x$ and $\partial^2 b / \partial x^2$. To accomplish this $b$ is reconstructed with a cubic polynomial $C_j(x)$ centred around $x_j$
\begin{equation*}
C_j(x) = c_0 \left(x - x_j\right)^3 + c_1 \left(x - x_j\right)^2 + c_2 \left(x - x_j\right) + c_3.
\label{eqn:cubicforbedrecon}
\end{equation*}

The cubic polynomial $C_j(x)$ passes through the nodal values $b_{j-2}$, $b_{j-1}$, $b_{j+1}$ and $b_{j+2}$ therefore,
\begin{equation*}
\begin{bmatrix}
\TM -8\Delta x^3 & 4\Delta x^2  & -2\Delta x & 1\\
-\Delta x^3 & \Delta x^2  &-\Delta x & 1\\
\Delta x^3 & \Delta x^2  & \Delta x & 1\\
8\Delta x^3 & 4\Delta x^2  & 2\Delta x & 1 \BM 
\end{bmatrix}
\begin{bmatrix}
 \TM c_0 \\ c_1 \\ c_2 \\ c_3  \BM\end{bmatrix} =  \begin{bmatrix}
 \TM b_{j-2} \\ b_{j-1}\\ b_{j+1} \\ b_{j+2} \BM
\end{bmatrix}.
\end{equation*}
Solving this matrix equation the polynomial coefficients for $C_j(x)$ are obtained
\begin{align*}
c_0 &=  \dfrac{-b_{j-2} + 2b_{j-1} - 2 b_{j+1} + b_{j+2}}{12 \Delta x^3},\\ \\
c_1 &=  \dfrac{b_{j-2} - b_{j-1} - b_{j+1} + b_{j+2}}{6 \Delta x^2},\\ \\
c_2 &=  \dfrac{b_{j-2} - 8b_{j-1} + 8 b_{j+1} - b_{j+2}}{12 \Delta x},\\ \\
c_3 &=  \dfrac{-b_{j-2}  + 4b_{j-1} + 4 b_{j+1} - b_{j+2}}{6}.
\end{align*}
The method requires a continuous bed profile and so the two reconstructions at the cell edge from the adjacent cells are averaged. Therefore, the reconstructing cubic for the bed profile in the $j^{th}$ cell is the cubic which takes these values
\begin{subequations}
\begin{align}
b_{j-1/2} &=  \mathcal{B}_{j-1/2}\left(\vecn{b}\right) =  \frac{1}{2}\left( C_j(x_{j-1/2}) + C_{j-1}(x_{j-1/2})\right),\\
b_{j-1/6} &=  \mathcal{B}_{j-1/6}\left(\vecn{b}\right) =  C_j(x_{j-1/6}),\\
b_{j+1/6} &=  \mathcal{B}_{j+1/6}\left(\vecn{b}\right) =  C_j(x_{j+1/6}),\\
b_{j+1/2} &=  \mathcal{B}_{j+1/2}\left(\vecn{b}\right) =  \frac{1}{2}\left( C_j(x_{j+1/2}) + C_{j+1}(x_{j+1/2})\right).
\end{align}
\label{eqn:BedReconDef}
\end{subequations}



\subsection{Fluid Velocity}
\label{subsec:FluidVelocity}
To calculate the remaining unknown primitive variable $u$ from the known quantities $h$, $G$ and $b$ a FEM is used to solve \eqref{defn:SerreEqnConservedQuantity1}. The FEM uses the weak form of \eqref{defn:SerreEqnConservedQuantity1} with a test function $v$ over the spatial domain $\Omega$ which is 
\begin{equation*}
	\int_{\Omega } G v \; dx =  \int_{\Omega } uh \left(1 + \frac{\partial h}{\partial x}\frac{\partial b}{\partial x} + \frac{1}{2}h\frac{\partial^2 b}{\partial x^2} +  \left[\frac{\partial b}{\partial x}\right]^2 \right) v - \frac{\partial}{\partial x}\left(\frac{1}{3}h^3  \frac{\partial {u}}{\partial x}\right) v \; dx.
\end{equation*}
Integrating this equation by parts with zero Dirichlet boundary conditions produces
\begin{multline}
\int_{\Omega } G v \; dx = \int_{\Omega } uh \left(1 + \left[\frac{\partial b}{\partial x}\right]^2 \right) v \; dx +  \int_{\Omega } \frac{1}{3}h^3  \frac{\partial {u}}{\partial x} \frac{\partial v}{\partial x} \; dx  \\ - 
\int_{\Omega }   \frac{1}{2} u h^2\frac{\partial b}{\partial x}  \frac{\partial v }{\partial x}\; dx - 
\int_{\Omega }   \frac{1}{2}h^2\frac{\partial b}{\partial x}  \frac{\partial u }{\partial x}v \; dx.
\label{eqn:WeakFormDomain}
\end{multline}
Note that this formulation of the weak form of \eqref{defn:SerreEqnConservedQuantity1} has moved the second derivative of $b$ and the derivative of $h^3 {\partial u}/{ \partial x}$ onto the test function, reducing the smoothness of these quantities required to produce a solution of \eqref{defn:SerreEqnConservedQuantity1}.


Assuming that time is fixed so that all the functions only vary in space, this formulation implies that by ensuring that $G$, $h$, $b$ and $\partial b / \partial x$ have finite integrals over $\Omega$, then $u$ and $\partial u / \partial x$ must have finite integrals as well. To approximate the flux and source terms \eqref{eqn:FullSerreCon} requires $u$ and $\partial u / \partial x$ to be well defined, and thus integrable. Therefore, it will be assumed that for each time $t$ that $h,G \in \mathbb{L}^2(\Omega)$, the space of square integrable functions and $b \in\mathbb{W}^{1,2}(\Omega)$ the space of square integrable functions whose first weak derivatives are also square integrable. These assumptions imply $u \in \mathbb{W}^{1,2}(\Omega) $ when solving \eqref{eqn:WeakFormDomain} and thus $u$ and $\partial u / \partial x$ will be well defined, as desired. For a precise definition of $\mathbb{L}^2(\Omega)$ and $\mathbb{W}^{1,2}(\Omega)$, see Appendix \ref{app:FEMIntegrals}.


To apply the FEM to \eqref{eqn:WeakFormDomain} the integration is performed over the cells and then summed to get the equation for the entire domain
\begin{multline}
\label{eq:elementwiseint}
 \sum_{j=0}^m \Bigg(  \int_{x_{j-1/2} }^{{x_{j+1/2}}} \Bigg[  \left( uh \left(1 + \left[\frac{\partial b}{\partial x}\right]^2 \right)  - \frac{1}{2}h^2\frac{\partial b}{\partial x}  \frac{\partial u }{\partial x}  -  G \right) v   \\ +  \left(\frac{1}{3}h^3  \frac{\partial {u}}{\partial x}    -     \frac{1}{2} uh^2\frac{\partial b}{\partial x}    \right) \frac{\partial v }{\partial x} \Bigg]dx \Bigg)  = 0
\end{multline}
which holds for all test functions $v$. The next step is to replace the functions for $h$, $G$, $b$, $v$ and $u$ with their corresponding basis function approximations. Expressions for these basis functions are provided in Appendix \ref{app:FEMIntegrals}.

\subsubsection{Basis Function Approximations}
%mention what these basis functions ARE!@!!!!!
For $h$ and $G$ the basis functions $\psi$ \eqref{eqn:App1:PsiDef} are used, the functions are linear inside a cell and zero elsewhere and so are not continuous as shown in Figure \ref{fig:P1DiscBasis}. These basis functions are consistent with the reconstructions of $h$ and $G$ which are linear inside a cell and discontinuous across the cell edges. Since these basis functions are in $\mathbb{L}^2(\Omega)$ the basis function approximations to $h$ and $G$ are in the appropriate function space.
\begin{figure}
	\centering
	\includegraphics[width=0.8\textwidth]{./chp3/figures/P1.pdf}
	\caption{Support of the discontinuous linear basis functions $\psi$ which are non-zero over the $j^{th}$ cell.}
	\label{fig:P1DiscBasis}
\end{figure}

The basis functions $\psi$ produce the following representation of $h$ and $G$ in the FEM written using the generic quantity $q$
\begin{align}
\label{eqn:FEapproxtohG}
q &= \sum_{j=0}^m \left( q^+_{j-1/2}\psi^+_{j-1/2}  + q^-_{j+1/2}\psi^-_{j+1/2} \right).
\end{align}

To calculate the flux and source terms in \eqref{eqn:Serreconsconmom} a locally calculated second-order accurate approximation to the first derivative of $u$ is required. To produce the appropriate approximation a quadratic representation of $u$ in each cell is used. Furthermore, since $u\in\mathbb{W}^{1,2}(\Omega)$ the quadratic representation of $u$ must be continuous across the cell edges $x_{j \pm 1/2}$. Therefore, the continuous quadratic basis functions $\phi_{j\pm1/2} $ and $\phi_{j}$ \eqref{eqn:App1:PhiDef} depicted in Figure \ref{fig:P2ContBasis} are used.
\begin{figure}
	\centering
	\includegraphics[width=0.8\textwidth]{./chp3/figures/P2.pdf}
	\caption{Support of the continuous piecewise quadratic basis functions $\phi$ which are non-zero over the $j^{th}$ cell.}
	\label{fig:P2ContBasis}
\end{figure}

From the basis functions $\phi$ the basis function approximation to $u$ is
\begin{equation}
u = u_{-1/2}\phi_{-1/2} + \sum_{j=0}^m \left( u_{j}\phi_{j} + u_{j+1/2}\phi_{j+1/2} \right).
\label{eqn:FEapproxtou}
\end{equation}

Approximating the source term of the evolution of $G$ equation \eqref{eqn:Serreconsconmom} requires a local approximation to the second derivative of the bed that is also second-order accurate. To allow for an appropriate second derivative of the bed profile, $b$ must be a member of $\mathbb{W}^{2,2}(\Omega)$ which is smoother than required by \eqref{eqn:WeakFormDomain}. Therefore, the cubic basis functions $\gamma$ \eqref{eqn:App1:GamDef} which are continuous across the cell edges are used. These basis functions are shown in Figure \ref{fig:P3ContBasis} and from them the basis function approximation to $b$ is
\begin{figure}
	\centering
	\includegraphics[width=0.8\textwidth]{./chp3/figures/P3.pdf}
	\caption{Support of the continuous piecewise cubic basis functions $\gamma$ which are non-zero over the $j^{th}$ cell.}
	\label{fig:P3ContBasis}
\end{figure}
\begin{equation}
b = b_{-1/2}\gamma_{-1/2} +  \sum_{j=0}^m \left(b_{j-1/6}\gamma_{j-1/6}  + b_{j+1/6}\gamma_{j+1/6} + b_{j+1/2}\gamma_{j+1/2} \right).
\label{eqn:FEapproxtob}
\end{equation}

\subsubsection{Calculation of Elementwise Matrices}
The integral equation \eqref{eq:elementwiseint} holds for all $v$. However, since the solution space has the basis functions $\phi$ it is sufficient to satisfy \eqref{eq:elementwiseint} for all $\phi$ to generate the solution. Since only the basis functions $\phi_{j-1/2}$, $\phi_{j}$ and $\phi_{j+1/2}$ are non-zero over the $j^{th}$ cell the $j^{th}$ term in the sum \eqref{eq:elementwiseint} can be calculated as follows
\begin{multline}
\int_{x_{j-1/2} }^{{x_{j+1/2}}} \Bigg(  \left[ uh \left(1  +  \left[\frac{\partial b}{\partial x}\right]^2 \right)  - \frac{1}{2}h^2\frac{\partial b}{\partial x}  \frac{\partial u }{\partial x}  -  G \right] \begin{bmatrix} \TM
\phi_{j-1/2}\\\phi_j \\\phi_{j+1/2} \BM\end{bmatrix}   \\ +  \left[ \frac{1}{3}h^3  \frac{\partial {u}}{\partial x}    -     \frac{1}{2}h^2\frac{\partial b}{\partial x} u    \right] \frac{\partial}{\partial x}\left(\begin{bmatrix} \TM
\phi_{j-1/2}\\\phi_j \\\phi_{j+1/2}  \BM
\end{bmatrix} \right) \Bigg)dx
\label{eqn:WeakFormElemXspace}
\end{multline}
where the finite element approximations for $h$ \eqref{eqn:FEapproxtohG}, $G$ \eqref{eqn:FEapproxtohG}, $u$ \eqref{eqn:FEapproxtou} and $b$ \eqref{eqn:FEapproxtob} are used. This integral can be generalised for all cells by moving to the natural reference $\xi$-space, as the basis functions which are non-zero in one element are just translations of the non-zero basis functions in another element. The mapping from the $x$-space to the $\xi$-space is
\begin{equation*}
x = x_j + \xi \frac{\Delta x}{2}.
\end{equation*}
Therefore, the $j^{th}$ cell $\left[x_{j-1/2}, x_{j+1/2}\right]$ gets mapped to $\left[-1,1\right]$ in the $\xi$-space. Performing the change of variables from $x$ to $\xi$ in \eqref{eqn:WeakFormElemXspace} produces
\begin{multline*}
\frac{\Delta x}{2}\int_{-1 }^{1} \Bigg( \left[ uh \left(1 + \frac{4}{\Delta x^2} \left[\frac{\partial b}{\partial \xi}\right]^2 \right)  - \frac{2}{\Delta x^2} h^2 \frac{\partial b}{\partial \xi}  \frac{\partial u }{\partial \xi}  -  G \right] \begin{bmatrix} \TM
\phi_{j-1/2}\\\phi_j \\\phi_{j+1/2} \BM
\end{bmatrix}   \\ + \frac{4}{\Delta x^2} \left[\frac{1}{3}h^3 \frac{\partial {u}}{\partial \xi}    -     \frac{1}{2}h^2 \frac{\partial b}{\partial \xi} u    \right] \frac{\partial}{\partial \xi}\left(\begin{bmatrix} \TM
\phi_{j-1/2}\\\phi_j \\\phi_{j+1/2} \BM
\end{bmatrix} \right) \Bigg)d\xi.
\end{multline*}

The remainder of the process will be demonstrated for the $uh$ term as an example with the remaining integrals provided \href{https://sites.google.com/view/jordanpitt/phd-thesis-resources/finite-element-integrals}{\color{blue}\underline{online}} (https://sites.google.com/view/jordanpitt/phd-thesis-resources/finite-element-integrals).
The $uh$ term is 
\begin{equation*}
\frac{\Delta x}{2}\int_{-1 }^{1} uh \begin{bmatrix} \TM
\phi_{j-1/2}\\\phi_j \\\phi_{j+1/2} \BM
\end{bmatrix} d\xi.
\end{equation*}
Since the integral is computed over $\left[-1,1\right]$, there are only a few non-zero contributions from the finite element approximations to $h$ and $u$, so the $uh$ term becomes
\begin{multline*}
\frac{\Delta x}{2}\int_{-1 }^{1}  \Bigg( \left(u_{j-1/2}\phi_{j-1/2} + u_{j}\phi_{j} + u_{j+1/2}\phi_{j+1/2}\right) \\ \times\left(h^+_{j-1/2}\psi^+_{j-1/2}  + h^-_{j+1/2}\psi^-_{j+1/2}\right) \begin{bmatrix} \TM
\phi_{j-1/2}\\\phi_j \\\phi_{j+1/2} \BM
\end{bmatrix} \Bigg) d\xi\\
=\frac{\Delta x}{2}\Bigg( h^+_{j-1/2} \int_{-1 }^{1} \psi^+_{j-1/2}  \begin{bmatrix} \TM
\phi_{j-1/2} \phi_{j-1/2} & \phi_{j}  \phi_{j-1/2}  & \phi_{j+1/2} \phi_{j-1/2}\\\phi_{j-1/2} \phi_{j} & \phi_{j} \phi_{j} &  \phi_{j + 1/2} \phi_{j}\\\phi_{j+1/2} \phi_{j-1/2} &  \phi_{j+1/2} \phi_{j} & \phi_{j+1/2} \phi_{j+1/2} \BM
\end{bmatrix} d\xi  \\ +  h^-_{j+1/2}\int_{-1 }^{1} \psi^-_{j+1/2} \begin{bmatrix} \TM
\phi_{j-1/2} \phi_{j-1/2} & \phi_{j}  \phi_{j-1/2}  & \phi_{j+1/2} \phi_{j-1/2}\\\phi_{j-1/2} \phi_{j} & \phi_{j} \phi_{j} &  \phi_{j + 1/2} \phi_{j}\\\phi_{j+1/2} \phi_{j-1/2} &  \phi_{j+1/2} \phi_{j} & \phi_{j+1/2} \phi_{j+1/2} \BM
\end{bmatrix} d\xi \Bigg)  \begin{bmatrix} \TM
u_{j-1/2}\\u_j \\u _{j+1/2} \BM
\end{bmatrix}.
\end{multline*}

Calculating the integrals of all the basis function combinations yields
\begin{multline*}
\frac{\Delta x}{2}\int_{-1 }^{1}  uh \begin{bmatrix} \TM
\phi_{j-1/2}\\\phi_j \\\phi_{j+1/2} \BM
\end{bmatrix} d\xi =  \\ \frac{\Delta x}{60}  \begin{bmatrix} \TM
7 h^+_{j-1/2} + h^-_{j+1/2} & 4 h^+_{j-1/2}   & - h^+_{j-1/2} - h^-_{j+1/2}\\ 4 h^+_{j-1/2} & 16 h^+_{j-1/2} + 16 h^-_{j+1/2}& 4 h^-_{j+1/2}\\ - h^+_{j-1/2} - h^-_{j+1/2} &  4 h^-_{j+1/2} &  h^+_{j-1/2} + 7 h^-_{j+1/2} \BM
\end{bmatrix}  \begin{bmatrix} \TM
u_{j-1/2}\\u_j \\u _{j+1/2} \BM
\end{bmatrix}.
%\label{eqn:FEMutermexample}
\end{multline*}

\subsubsection{Assembly of the Global Matrix}
By combining all the matrices generated by the integral of each of the $u$ terms, the contribution of the $j^{th}$ cell to the stiffness matrix $\matr{A}_j$ is obtained. Likewise all the integrals of the remaining term $Gv$ in \eqref{eq:elementwiseint} generate the elementwise vector $\vecn{g}_{j}$. These elementwise matrices and vectors are then assembled into the global stiffness matrix $\matr{A}$ and the global right hand-side term $\vecn{g}$ thus \eqref{eq:elementwiseint} is rewritten as
\begin{equation}
\label{eqn:FEMElemMatrixJ}
 \matr{A} \vecn{\hat{u}} = \vecn{g}.
\end{equation}
This is a penta-diagonal matrix equation which can be solved by direct banded matrix solution techniques such as those of \citet{NumRecC-1996} to obtain
\begin{equation}
\vecn{\hat{u}} =\mathcal{G}\left( \vecn{\hat{h}}, \vecn{\hat{G}},\vecn{\hat{b}} \right) =   \matr{A}^{-1}\vecn{g}
\label{eqn:usolvefromGhb}
\end{equation}
as desired.

The matrix $\matr{A}$ is well conditioned as long as $h\gg0$. However, as $h$ vanishes the condition number of the matrix increases. To avoid an inaccurate solution for $u$ when $h$ is small the techniques described in Section \ref{sec:DryBeds} are employed.  
\subsection{Flux Across the Cell Interfaces}

The method of \citet{Kurganov-etal-2001-707} is used to approximate the flux across a cell interface. This method was employed because it can handle discontinuities across the cell boundary and only requires an estimate of the maximum and minimum wave speeds. This is precisely the situation for the Serre equations which do not have a known expression for the characteristics but do possess estimates on the maximum and minimum wave speeds \eqref{eqn:WaveVelocitiesBound}.

Only the calculation of the flux term $F_{j+1/2}$ is demonstrated as the process to calculate the flux term $F_{j-1/2}$ is identical but with different cells. For a general quantity $q$ the approximation of the flux term given by \citet{Kurganov-etal-2001-707} is
\begin{equation}\label{eqn:HLL_flux}
F_{j+\frac{1}{2}} = \dfrac{a^+_{j+\frac{1}{2}} f\left(q^-_{j+\frac{1}{2}}\right) - a^-_{j+\frac{1}{2}} f\left(q^+_{j+\frac{1}{2}}\right)}{a^+_{j+\frac{1}{2}} - a^-_{j+\frac{1}{2}}}  + \dfrac{a^+_{j+\frac{1}{2}} \, a^-_{j+\frac{1}{2}}}{a^+_{j+\frac{1}{2}} - a^-_{j+\frac{1}{2}}} \left(  q^+_{j+\frac{1}{2}} - q^-_{j+\frac{1}{2}} \right)
\end{equation}
where $a^+_{j+\frac{1}{2}}$ and $a^-_{j+\frac{1}{2}}$ are given by bounds on the wave speed. Applying the wave speed bounds \eqref{eqn:WaveVelocitiesBound}
\begin{align}
a^-_{j+\frac{1}{2}} &= \min\left\lbrace 0\;,\;  u^-_{j + 1/2} - \sqrt{g h^-_{j + 1/2}}  \;,\;u^+_{j + 1/2} - \sqrt{g h^+_{j + 1/2}} \right\rbrace  ,\\
a^+_{j+\frac{1}{2}} &= \max\left\lbrace 0 \;,\;  u^-_{j + 1/2} + \sqrt{g h^-_{j + 1/2}}  \;,\;u^+_{j + 1/2} + \sqrt{g h^+_{j + 1/2}} \right\rbrace .
\label{eqn:WaveSpeedBoundsFluxApprox}
\end{align}

The flux functions $f(q^-_{j+\frac{1}{2}})$ and $f(q^+_{j+\frac{1}{2}})$ across the cell edge $x_{j+1/2}$ are evaluated using the reconstructed values $q^-_{j+\frac{1}{2}}$ from the $j^{th}$ cell and $q^+_{j+\frac{1}{2}}$ from the $(j+1)^{th}$ cell. From the continuity equation \eqref{eqn:FullSerreConMass} it is obtained that
\begin{align*}
f\left(h^\pm_{j+\frac{1}{2}}\right) &= u^\pm_{j + 1/2}  h^\pm_{j + 1/2}.
\end{align*}

For the evolution of $G$ equation \eqref{eqn:Serreconsconmom} it is obtained that 
\begin{align}
f\left(G^\pm_{j+\frac{1}{2}}\right) &=  u^\pm_{j + 1/2} G^\pm_{j + 1/2}  + \frac{g}{2}\left(h^\pm_{j + 1/2} \right)^2 - \frac{2}{3}\left(h^\pm_{j + 1/2}\right)^3 \left[\left(\frac{\partial {u}}{\partial x} \right)^\pm_{j + 1/2} \right]^2 \nonumber\\ &+ \left(h^\pm_{j + 1/2}\right)^2 u^\pm_{j + 1/2} \left(\frac{\partial {u}}{\partial x} \right)^\pm_{j + 1/2} \left(\frac{\partial b}{\partial x} \right)^\pm_{j + 1/2} .
\label{eqn:FluxIrrotNum}
\end{align}

The quantities $h^+_{j - 1/2}$, $h^-_{j + 1/2}$, $G^+_{j - 1/2}$ and $G^-_{j + 1/2}$ were calculated during the reconstruction and the FEM provided $u^\pm_{j+1/2} = u_{j+1/2}$ as $u$ is continuous across the cell boundaries.

\subsubsection{Calculation of Derivatives}
Approximations to $\left(\dfrac{\partial {b}}{\partial x} \right)^\pm_{j + 1/2}$ and $\left(\dfrac{\partial {u}}{\partial x} \right)^\pm_{j + 1/2}$ are required to calculate the flux \eqref{eqn:FluxIrrotNum}. To calculate these derivatives in $u$ and $b$ the basis function approximation to these quantities in the FEM are used. For $u$ the reconstructing quadratic polynomial is
	\begin{equation}
	P^u_j(x) = p^u_0 \left(x - x_j\right)^2 + p^u_1 \left(x - x_j\right) + p^u_2
	\label{eqn:Polyforucell}
	\end{equation}
which passes through $u_{j-1/2}$, $u_j$ and $u_{j+1/2}$ \eqref{eqn:usolvefromGhb}. While for $b$ the reconstructing cubic polynomial is
	\begin{equation}
	P^b_j(x) = p^b_0 \left(x - x_j\right)^3 + p^b_1 \left(x - x_j\right)^2 + p^b_2 \left(x - x_j\right)  + p^b_3
	\label{eqn:Polyforbcell}
	\end{equation}
which passes through $b_{j-1/2}$, $b_{j-1/6}$, $b_{j+1/6}$ and $b_{j+1/2}$ \eqref{eqn:BedReconDef}. Because the cell edge values were averaged during the reconstruction of the bed, $P^b_j(x)$ will be different from $C_j(x)$.

For $P^u_j(x)$ the coefficients are
\begin{align*}
p^u_0 &=  \dfrac{u_{j-1/2} - 2u_j + u_{j+1/2}}{2 \Delta x^2},\\
p^u_1 &=  \dfrac{-u_{j-1/2} + u_{j+1/2}}{\Delta x},\\
p^u_2 &=  u_j.
\end{align*}
While for $P^b_j(x)$ the coefficients are
\begin{align*}
p^b_0 &=  \dfrac{-9b_{j-1/2} + 27b_{j-1/6} - 27 b_{j+1/6} + 9b_{j+1/2}}{2 \Delta x^3},\\ \\
p^b_1 &=  \dfrac{9b_{j-1/2} - 9b_{j-1/6} - 9b_{j+1/6} + 9b_{j+1/2}}{4 \Delta x^2},\\ \\ 
p^b_2 &=  \dfrac{b_{j-1/2} - 27b_{j-1/6} + 27 b_{j+1/6} - b_{j+1/2}}{8 \Delta x},\\\\
p^b_3 &=  \dfrac{-b_{j-1/2}  + 9b_{j-1/6} + 9 b_{j+1/6} - b_{j+1/2}}{16}.
\end{align*}
Taking the derivative of the polynomials \eqref{eqn:Polyforucell} and \eqref{eqn:Polyforbcell} produces
	\begin{align*}
	\frac{\partial }{\partial x}P^u_j(x) &= 2p^u_0 \left(x - x_j\right) + p^u_1, \\
	\frac{\partial }{\partial x}P^b_j(x) &= 3p^b_0 \left(x - x_j\right)^2 + 2p^b_1 \left(x - x_j\right) + p^b_2.
	\end{align*}
This is a second-order approximation to the derivative of $u$ and $b$ at $x_{j+1/2}$ for the $j^{th}$ cell. The process for the $(j+1)^{th}$ cell is the same and therefore, 
	\begin{align*}
	\left(\dfrac{\partial {u}}{\partial x} \right)^-_{j + 1/2} &= \frac{\partial }{\partial x}P^u_j(x_{j+1/2}),  \\
	\left(\dfrac{\partial {u}}{\partial x} \right)^+_{j + 1/2} &= \frac{\partial }{\partial x}P^u_{j+1}(x_{j+1/2}),  \\
	\left(\dfrac{\partial {b}}{\partial x} \right)^-_{j + 1/2} &= \frac{\partial }{\partial x}P^b_j(x_{j+1/2}), \\
	\left(\dfrac{\partial {b}}{\partial x} \right)^+_{j + 1/2} &= \frac{\partial }{\partial x}P^b_{j+1}(x_{j+1/2}). 	\end{align*}

Hence, all the terms needed to calculate the approximation to the flux \eqref{eqn:HLL_flux} for $h$ and $G$ are possessed. However, to ensure that the FEVM is well-balanced and recovers the lake at rest steady state solution, the approximation to the intercell fluxes must be modified.

\subsubsection{Well-Balancing Modification to Flux Approximation}
To recover the lake at rest steady state solution the method of \citet{Klein-etal-2004-2050} is employed. This method was initially designed for the SWWE and has been extended to the Serre equations \cite{Pitt-J-2014}. To enforce well-balancing the reconstruction of $h$ is modified at the cell edges in the following way.

First calculate
\begin{align}
\dot{b}^-_{j+1/2} = w^-_{j+1/2} - h^-_{j+1/2} &, &\dot{b}^+_{j+1/2} = w^+_{j+1/2} - h^+_{j+1/2}.
\label{eqn:BedReDefWmH}
\end{align}
Find the maximum
\begin{align*}
\ddot{b}_{j+1/2} = \max\left\lbrace\dot{b}^-_{j+1/2} , \dot{b}^+_{j+1/2} \right\rbrace
\end{align*}
then define
\begin{subequations}
\begin{align}
\ddot{h}^-_{j+1/2} &= \max\left\lbrace 0, w^-_{j+1/2} - \ddot{b}_{j+1/2}  \right\rbrace, \\  \ddot{h}^+_{j+1/2} &= \max\left\lbrace 0, w^+_{j+1/2} - \ddot{b}_{j+1/2} \right\rbrace.
\end{align}
\label{eqn:ModifiedHValue}
\end{subequations}
This generates the vector $\vecn{\ddot{h}}$
\begin{equation*}
\vecn{\ddot{h}}= \begin{bmatrix} \TM
\ddot{h}^+_{-1/2} \\ h_0 \\ \ddot{h}^-_{1/2} \\ \vdots  \\ \ddot{h}^-_{m + 1/2}  \BM \end{bmatrix}
\end{equation*}
which is used to calculate the flux term $F_{j+1/2}$ in \eqref{eqn:HLL_flux} for $h$ and $G$ instead of $\vecn{\hat{h}}$. Applying the same process but with different cells $F_{j-1/2}$ is obtained and thus
\begin{align*}	
F^n_{j\pm 1/2} &=\mathcal{F}_{j\pm1/2} \left(\vecn{\ddot{h}}, \vecn{\hat{G}},\vecn{\hat{b}}, \vecn{\hat{u}}  \right)
\end{align*}
for the evolution of $h$ and $G$ equations as desired. The modification to the intercell flux as well as the process to approximate the source terms described below ensures that the method recovers the lake at rest steady state, as shown in Chapter \ref{chp:NumMethodComp}.

\subsection{Source Terms}
\label{subsec:SourceTerm}
To evolve the Serre equations \eqref{eqn:FullSerreCon}, an approximation to the source term at the cell centre $x_j$ from time $t^n$ to $t^{n+1}$ is required, it will be denoted as $S^n_j$. Equation \eqref{eqn:FullSerreConMass} has no source term, therefore only the calculation of the source term for equation \eqref{eqn:Serreconsconmom} is presented.

Following the work of \citet{Klein-etal-2004-2050} to produce a well-balanced method, the approximation to $S^n_j$ is split into the naive source term approximation $S_{ci}$ and the corrective interface source terms $S^{-}_{j + \frac{1}{2}}$ and $S^{+}_{j + \frac{1}{2}}$
\begin{equation*}
S^n_j =  S^{-}_{j + \frac{1}{2}} + \Delta x S_{ci} + S^{+}_{j - \frac{1}{2}}.
\end{equation*}
The corrective interface terms $S^{-}_{j + \frac{1}{2}}$ and $S^{+}_{j + \frac{1}{2}}$ together with the modifications to the intercell flux ensure that the flux and source terms cancel for the lake at rest solution. 

The centred source term is calculated like so
\begin{equation*}
 S_{ci} = -\frac{1}{2}\left(h_j\right)^2 {u_j}\left( \frac{\partial {u}}{\partial x} \right)_j \left(\frac{\partial^2 b}{\partial x^2} \right)_j  + h_j \left(u_j\right)^2 \left(\frac{\partial b}{\partial x}\right)_j \left(\frac{\partial^2 b}{\partial x^2}\right)_j - gh_j\left(\frac{\partial b}{\partial x}\right)_j.
\end{equation*}
Where $h_j$ from the reconstruction process \eqref{eqn:ReconforhwG} and $u_j$ from the solution of \eqref{eqn:usolvefromGhb} are used. To calculate the derivatives of $u$ and $b$ their polynomial representations \eqref{eqn:Polyforucell} and \eqref{eqn:Polyforbcell} inside a cell are used. However, to ensure that the terms cancel properly for a lake at rest, the approximation to ${\partial b}/{\partial x}$ is modified to use $\dot{b}^-_{j+1/2}$ and $\dot{b}^+_{j+1/2}$ from \eqref{eqn:BedReDefWmH}. Therefore, the following approximations are used to calculate $S_{ci}$
	\begin{align*}
	\left(\dfrac{\partial {u}}{\partial x} \right)_{j} &= \frac{\partial }{\partial x}P^u_j(x_{j}),  \\
\left(\dfrac{\partial {b}}{\partial x} \right)_{j} &=  \frac{\dot{b}^-_{j+1/2} - \dot{b}^+_{j-1/2}}{\Delta x} , \\	
	\left(\dfrac{\partial^2 {b}}{\partial x^2} \right)_{j} &= \frac{\partial^2 }{\partial x^2}P^b_j(x_{j}).
	\end{align*}

To ensure well-balancing the corrective interface source terms
	\begin{align*}
	 S^{-}_{j + \frac{1}{2}} &=  \frac{g}{2} \left(\ddot{h}^{-}_{j + \frac{1}{2}} \right)^2 - \frac{g}{2} \left(h^{-}_{j + \frac{1}{2}} \right)^2, \\ \\
	  S^{+}_{j - \frac{1}{2}} &=  \frac{g}{2} \left(h^{+}_{j - \frac{1}{2}}\right)^2 - \frac{g}{2}\left(\ddot{h}^{+}_{j - \frac{1}{2}}\right)^2 
	\end{align*}
are also added. These corrective terms make use of $h^{-}_{j + \frac{1}{2}}$ and $h^{+}_{j + \frac{1}{2}}$ obtained from the reconstruction \eqref{eqn:ReconforhwG} and the modified values $\ddot{h}^{-}_{j + \frac{1}{2}}$ and $\ddot{h}^{+}_{j + \frac{1}{2}}$ from \eqref{eqn:ModifiedHValue}. Combining the centred and interface source terms the approximation to the source term for $G$ is 
\begin{equation*}
S^n_j = \mathcal{S}_j\left( \vecn{\hat{h}},\vecn{\ddot{h}},\vecn{\hat{w}},\vecn{\hat{b}}, \vecn{\hat{u}}  \right)=   S^{-}_{j + \frac{1}{2}} + \Delta x S_{ci} + S^{+}_{j - \frac{1}{2}}.
\end{equation*}
This operator $\mathcal{S}_j$ is slightly different to the example given in the overview of the evolution step as it takes $\vecn{\hat{h}}$ and $\vecn{\ddot{h}}$ as inputs. This change was made to make clear that $\vecn{\ddot{h}}$ is required to calculate $S^n_j$. This would be obscured by a more consistent notation that is possible since $\vecn{\ddot{h}}$ only depends on $\vecn{\hat{h}}$, $\vecn{\hat{w}}$ and $\vecn{\hat{b}}$. 


\subsection{Update Cell Averages}
Applying a forward Euler approximation with the approximation to the flux and source terms it is obtained that
\begin{align}
\overline{q}^{n+1}_j &= \overline{q}^{n}_j + \frac{\Delta t}{\Delta x} \left(F^n_{j+\frac{1}{2}} - F^n_{j-\frac{1}{2}} + S^n_j\right)
\label{eqn:UpdateMethod}
\end{align}
where $F^n_{j+\frac{1}{2}}$, $F^n_{j-\frac{1}{2}}$ and $S^n_j$ are all calculated using the quantities at time $t^n$. This update formula is first-order in time.


\subsection{Second-Order SSP Runge-Kutta Method}
To increase the order of accuracy in time a Strong Stability Preserving (SSP) Runge-Kutta method \cite{Gottlieb-etal-2003-89} is employed. The second-order SSP Runge Kutta method is a convex combination of the first-order time steps \eqref{eqn:UpdateMethod} in the following way
\begin{subequations}
\begin{align}
\overline{q}_j^{(1)} &= \overline{q}^{n}_j + \frac{\Delta t}{\Delta x} \left(F^n_{j+\frac{1}{2}} - F^n_{j-\frac{1}{2}} + S^n_j\right),\\
\overline{q}_j^{(2)} &= \overline{q}_j^{(1)} + \frac{\Delta t}{\Delta x} \left(F_{j+\frac{1}{2}}^{(1)} - F_{j-\frac{1}{2}}^{(1)}  + S_j^{(1)} \right), \\
\overline{q}^{n+1}_j &= \frac{1}{2} \left( \overline{q}^n_j +  \overline{q}_j^{(2)}  \right).
\end{align}
\label{eqn:SSPRKStep1}
\end{subequations}
The second-order SSP Runge-Kutta method is a time stepping method that preserves the stability of the first-order method \eqref{eqn:UpdateMethod} and is second-order accurate in time. Since all the spatial approximations are second-order accurate, the steps (i)-(vi) should result in a second-order accurate FEVM for the Serre equations, as desired. 


\section{CFL condition}
To ensure the stability of the FEVM the Courant-Friedrichs-Lewy (CFL) condition \cite{Courant-etal-1967-215} is used, as it is a necessary condition for stability. The CFL condition ensures that time steps are small enough so that information is only transferred between neighbouring cells. For the Serre equations the CFL condition is 
\begin{equation}
\Delta t \le \frac{Cr }{\max_{j} \left\lbrace a^\pm_{j+1/2} \right\rbrace} \Delta x
\label{eqn:CFLcond}
\end{equation}
where $a^\pm_{j+1/2} $ are the wave-speed bounds used in the flux approximation \eqref{eqn:WaveSpeedBoundsFluxApprox} and $0\le Cr \le 1$ is the Courant number. Typically, the conservative $Cr = 0.5$ is used in the numerical experiments in this thesis. 

\section{Boundary Conditions}
To numerically model the Serre equations over finite spatial domains boundary conditions must be enforced at the left and right edge of the domain; $x_{-1/2}$ and $x_{m+1/2}$ respectively. For simplicity, only Dirichlet boundary conditions for the FEVM were considered in this thesis. These Dirichlet boundary conditions are enforced using ghost cells located outside the domain boundaries. These ghost cells contain the complete representation of their respective quantities over the cell. Additionally, since a finite volume based method is employed, only a first-order accurate approximation to the quantities at the boundaries is required to maintain global second-order accuracy \cite{Gustafsson}. For $h$, $w$, $G$ and $u$ one ghost cell at each boundary is required, while for $b$ two ghost cells at each boundary are required. The ghost cells for $h$, $w$ and $G$ written for a generic quantity $q$ are
\begin{align*}
&\vecn{\hat{q}}_{-1} = \begin{bmatrix} \TM
q^+_{-3/2} \\ q_{-1} \\ q^-_{-1/2} \BM \end{bmatrix},&\vecn{\hat{q}}_{m+1} = \begin{bmatrix} \TM
q^+_{m+1/2} \\ q_{m+1} \\q^-_{m+3/2} \BM\end{bmatrix}&.
\end{align*}
For $u$ and $b$ the ghost cells are
\begin{align*}
&\vecn{\hat{u}}_{-1} = \begin{bmatrix}\TM
u_{-3/2} \\ u_{-1} \\ u_{-1/2} \BM\end{bmatrix},&\vecn{\hat{u}}_{m+1} = \begin{bmatrix}\TM
u_{m+1/2} \\ u_{m+1} \\ u_{m+3/2} \BM\end{bmatrix},&
\end{align*}
\begin{align*}
\vecn{\hat{b}}_{-2} = \begin{bmatrix} \TM
b_{-5/2} \\ b_{-13/6} \\ b_{-11/6} \\ b_{-3/2} \BM\end{bmatrix}&, &\vecn{\hat{b}}_{-1} = \begin{bmatrix} \TM
b_{-3/2} \\ b_{-7/6} \\ b_{-5/6} \\ b_{-1/2} \BM\end{bmatrix}&, &\vecn{\hat{b}}_{m+1} = \begin{bmatrix} \TM
b_{m+1/2} \\ b_{m+5/6} \\ b_{m+7/6} \\ b_{m+3/2} \BM\end{bmatrix}&, &\vecn{\hat{b}}_{m+2} = \begin{bmatrix} \TM
b_{m+3/2} \\ b_{m+11/6} \\ b_{m+13/6} \\ b_{m+5/2} \BM\end{bmatrix}.
\end{align*}

To ensure that the solution of $u$ using \eqref{eqn:usolvefromGhb} agrees with the boundary conditions $\vecn{\hat{u}}_{-1}$ and $\vecn{\hat{u}}_{m}$ the elementwise stiffness matrices $\matr{A}_0$ and $\matr{A}_m$ and vectors $\vecn{g}_0$ and $\vecn{g}_m$ must be modified in the following way 

\begin{align*}
\matr{A}_{0} = 
\begin{bmatrix}
\TM 1 &0 &0 \\
a_{21} & a_{22} & a_{23} \\
a_{31} & a_{32} & a_{33} \BM
\end{bmatrix} &,& \vecn{g}_{0} = \begin{bmatrix}
\TM u_{-1/2} \\
g_1 \\
g_2 \BM
\end{bmatrix},\\
\matr{A}_{m} = 
\begin{bmatrix}
\TM a_{11} & a_{12} & a_{13} \\
a_{21} & a_{22} & a_{23} \\
0 & 0 & 1 \BM
\end{bmatrix} &,& \vecn{g}_{m} = \begin{bmatrix}
\TM g_0\\ 
g_1 \\
u_{m+1/2} \BM
\end{bmatrix}.
\end{align*}
These are then assembled with the other element contributions in the global stiffness matrix $\matr{A}$ and right hand side vector $\vecn{g}$ in \eqref{eqn:FEMElemMatrixJ}.
\section{Dry Beds}
\label{sec:DryBeds}
Dry beds are handled adequately by all steps of the FEVM in their current form, except the FEM for $u$. A dry bed presents two issues; when $h$ and $G$ are small then small errors in $h$ and $G$ can produce large errors in $u$ leading to instabilities and when $h=0$ the stiffness matrix $\matr{A}$ \eqref{eqn:usolvefromGhb} becomes singular.

The issue of large errors in $u$ when $h$ is small also arises when solving the SWWE \cite{RICCHIUTO20091071}; due to $u = (uh)/h $ being undefined as $u h $ and $h$ go to zero. For the Serre equations with horizontal beds when $h \ll 1$ \eqref{defn:SerreEqnConservedQuantity1HorizBed} becomes
\begin{equation}
G = uh + \mathcal{O}\left(h^3\right).
\end{equation}
Since $h \ll 1$ the $\mathcal{O}\left(h^3\right)$ terms are neglected, and thus when $h$ is small $G$ is equal to the momentum $uh$, and the challenges posed by $h \rightarrow 0$ for the SWWE and the Serre equations are equivalent. Therefore, the dry bed handling techniques for the SWWE can be applied to the Serre equations such a desingularisation transformations \cite{Kurganov-Petrova-2007-707} or zeroing of the velocity when $h$ is small \cite{RICCHIUTO20091071}. Since the first derivative of $u$ is required in the flux approximation the smooth desingularisation transformations of \citet{Kurganov-Petrova-2007-707} were preferred for the Serre equations.

These desingularisation transforms act by modifying the calculation of $u$ given $h$ and $uh$ to avoid the singularity as the numerator and denominator go to zero, hence their name. The simplest such transformation is
\begin{equation}
u = \frac{(uh) h}{h\left(h + h_{base}\right)}
\label{eqn:calculationofugivenuhandh}
\end{equation}
where $h_{base}$ is a small chosen parameter. The analytical error introduced by this transformation decreases as $h_{base}$ decreases. However, as noted by \citet{Kurganov-Petrova-2007-707} small values of $h_{base}$ lead to large numerical errors in the calculation of $u$. To avoid such errors $h_{base}$ can be made larger or following \citet{Kurganov-Petrova-2007-707} different desingularisation transformations can be employed. For the main purpose of this thesis; the validation tests reported in Chapter \ref{chp:NumMethodComp} the simpler transformation with small values of $h_{base}$ were more useful, although large numerical errors in $u$ were possible for small values of $h$. 

To adapt the calculation of $u$ in \eqref{eqn:calculationofugivenuhandh} to \eqref{defn:SerreEqnConservedQuantity1} it is viewed as a transformation of the quantity $h$ which is equivalent to
\begin{equation}
h \rightarrow h \left( \frac{h + h_{base}}{h} \right).
\end{equation}
This transformation is ill-defined when $h = 0$ so a small term $h_{tol}$ is added to the denominator. The term $h_{tol}$ serves as the cut-off value with any cells with $h < h_{tol}$ being considered dry. Therefore, the transformation for the reconstructed values of $h$ in the finite element method is
\begin{subequations}
	\begin{align}
	h^+_{j-1/2} & = h^+_{j-1/2} \left(\frac{ h^+_{j-1/2}  + h_{base}}{h^+_{j-1/2} + h_{tol}}\right) , \\ \nonumber\\
	h^-_{j+1/2} & = h^-_{j+1/2} \left(\frac{ h^-_{j+1/2}  + h_{base}}{h^-_{j+1/2} + h_{tol}}\right)
	\end{align} 
	\label{eqn:hdrytransform}
\end{subequations}
where on the right hand side are the reconstructed values of $h$ from \eqref{eqn:ReconforhwG} and the left hand side are the values of $h$ used to defined the basis functions of the FEM \eqref{eqn:FEapproxtohG}. This transformation is applied to all terms in the FEM avoiding the singularity as $h \rightarrow 0$; and in the case where $G = uh$ the transformation is equivalent to \eqref{eqn:calculationofugivenuhandh} for the SWWE.

Even with the transform \eqref{eqn:hdrytransform}, the matrix $\matr{A}$ can become singular.
The methods of \citet{Zoppou-etal-2017} made use of direct banded matrix solvers such as the Thomas algorithm \cite{Conte-DeBoor-1980} to solve \eqref{eqn:usolvefromGhb} which rely on non-singular matrices making them unsuitable when $h = 0$. This was resolved by employing an LU decomposition algorithm described by \citet{NumRecC-1996}. This algorithm solves banded matrix problems using an LU decomposition with partial pivoting. When the value of a pivot is below some tolerance value $p_{tol}$ the algorithm replaces its value with $p_{tol}$, avoiding the errors associated with small pivot values during an LU decomposition. Furthermore, this LU decomposition retains the banded matrix structure, and so is not as memory intensive as a standard LU decomposition. Typically, the value $p_{tol} = 10^{-20}$ was used, allowing the matrix solver to accurately invert $\matr{A}$ and thus solve \eqref{eqn:usolvefromGhb} when $h = 0$. 

Solving \eqref{eqn:usolvefromGhb} using the LU decomposition algorithm of \citet{NumRecC-1996} where the transformation \eqref{eqn:hdrytransform} has been applied to the reconstructed values of $h$ an approximation to $u$ is obtained that is valid in the presence of dry beds. Additionally, to avoid numerical errors becoming dominant when $h$ is very small a cut-off is placed on $h$ beyond which $h=G= u = 0$ and the cells are dry; this is given by $h_{tol}$. The drying of the cells is performed for the whole cell based on the cell average value of $h$ so that if $\overline{h}_j \le h_{tol}$ then
\begin{align*}
 & 	h^+_{j-1/2}  = 0   & 	G^+_{j-1/2}  = 0  & & 	w^+_{j-1/2}  = b_{j-1/2}   \\
 &	h_{j} = 0 & 	G_{j}  = 0  & 	&w_{j}  = b_{j},\\
 & 	h^-_{j+1/2}  = 0  & 	G^-_{j+1/2}  = 0 & 	&w^-_{j+1/2}  = b_{j+1/2} 
 \end{align*}
 and
 \begin{align*}
 & 	u_{j-1/2}  = 0  &\text{if}& &\overline{h}_{j-1}\le h_{tol} \\
 & 	u_{j}  = 0 \\
 & 	u_{j+1/2}  = 0  &\text{if}& &\overline{h}_{j+1} \le h_{tol}.
\end{align*}
The drying procedure occurs after the solution of \eqref{eqn:usolvefromGhb}. In the numerical experiments the typical values used were $h_{tol} = 10^{-12}$ and $h_{base} = 10^{-8}$.

\medskip

In this chapter $\text{FEVM}_2$ was described, including the details for the well-balancing and dry bed handling procedures. A linear analysis of the convergence and dispersion properties of $\text{FEVM}_2$ will now be performed. 
