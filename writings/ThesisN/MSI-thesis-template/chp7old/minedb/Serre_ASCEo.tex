\documentclass[SingleSpace,12pt,Proceedings]{Serre_ASCE}
\usepackage[dvips]{graphicx}
\usepackage[dvipsnames]{xcolor}
\usepackage{amsmath}
\usepackage{amsfonts}
\usepackage{amssymb}
\usepackage{lineno}
\usepackage{enumerate}
\usepackage{url}
\usepackage{times}
\usepackage{subfigure}
\usepackage{graphicx,psfrag}
\usepackage[skip=0pt]{caption}

% TIME ON EVERY PAGE AS WELL AS THE FILE NAME
\usepackage{fancyhdr}
\usepackage{currfile}
\usepackage[us,12hr]{datetime} % `us' makes \today behave as usual in TeX/LaTeX
\fancypagestyle{plain}{
\fancyhf{}
\rfoot{\small Draft Paper \\ File Name: {\currfilename} \\ Date: {\ddmmyyyydate\today} at \currenttime}
\lfoot{Page \thepage}
\renewcommand{\headrulewidth}{0pt}}
\pagestyle{plain}

\newcommand\solidrule[1][0.25cm]{\rule[0.5ex]{#1}{1pt}}
\newcommand\dashedrule{\mbox{%
  \solidrule[2mm]\hspace{2mm}\solidrule[2mm]}}

\newcommand{\dotrule}[1]{%
	\parbox{#1}{\dotfill}} 

\makeatletter
\newcommand \Dotfill {\leavevmode \cleaders \hb@xt@ .22em{\hss .\hss }\hfill \kern \z@}
\makeatother
 
\newcommand{\Dotrule}[1]{%
   \parbox{#1}{\Dotfill}} 

\begin{document}

\title{Behaviour of the Serre Equations for the Dam-Break Problem}

\author{
Jordan~Pitt,%
\thanks{Mathematical Sciences Institute, Australian National University, Canberra, ACT 0200, Australia, E-mail: Jordan.Pitt@anu.edu.au. The work undertaken by the first author was supported financially by an Australian National University Scholarship.}
\\
Christopher~Zoppou,\footnotemark[1]%
%
% Adding a second author with the same affiliation (still using \thanks):
\\
Stephen~G.~Roberts,\footnotemark[1]
}

\maketitle

\begin{abstract}

\end{abstract}

\KeyWords{dispersive waves, conservation laws, Serre equation, finite volume method, finite difference method}

\linenumbers

%--------------------------------------------------------------------------------
\section{Introduction} \label{intro} 
%plan:
%background dam break/ dispersive models
%why discontinuities are important to model:
%	bores (tsunamis)
%	steepening
%	plasmas?
% differences in the literature, comment on literature
%aim of paper/ organisation of paper
The dam break problem is a classical problem in shallow water modelling. Its main challenge is that it contains a discontinuity in the initial conditions at the dam wall. For some equations such as the Shallow Water Wave equations (SWWE) this is not a problem for finite volume based numerical methods \cite{Zoppou-Roberts-2003-11} as they're in conservative form. For dispersive shallow water models such as the Serre equations this discontinuity does present an issue as the equations usually require certain smoothness of the initial conditions and aren't in conservative form \cite{Dutykh-2014-315}. 

The dam break problem for a dispersive model results in undular bores as the sharp shock front decomposes into solitary waves at the shock front \cite{Peregrine-D-1966-321}. Correctly modelling dispersion is important to accurately represent a range of fluid problems from tidal bores to tsunamis \cite{Glimsdal-etal-2013-1507}. Shoaling of waves can also result in very steep gradients and so to have accurate models of the nearshore behaviour of fluids requires methods that are robust to large gradients in water depth. 

Our equations of interest are the Serre equations \cite{Serre-F-1953-857,Su-Gardener-1969-536,Seabra-Santos-etal-1987-117,Green-Naghdi-1976-237}, which up to wave breaking are considered a very good approximation to the full water wave problem in shallow water \cite{Bonneton-etal-2011-1479}. The literature around numerical methods for the Serre equations deal with the problem in many ways from simple finite difference methods (FDM) \cite{Antunes-do-Carmo-etal-1993-725,Nwogu-O-1993-618,El-etal-2006} to finite element methods (FEM) \cite{Guyenne-etal-2014-169,Dutykh-2014-315} and even hybrid finite difference-volume hybrid methods (FDVHM) \cite{Hank-etal-2010-2034,Zoppou-etal-2017,Filippini-etal-2016-381,Bradford-Sanders-2002-953}. There are however little results concerning discontinuities or even very steep gradients \cite{El-etal-2006,Hank-etal-2010-2034,Dutykh-2014-315}. Among these results only one handles the problem well \cite{El-etal-2006} with the others only presenting very shallow analysis of the problem \cite{Hank-etal-2010-2034,Dutykh-2014-315}. There is also disagreement among these papers about the nature of these solutions. 

The aim of this paper is then to further investigate the dam break problem and the behaviour of steep gradients for the Serre equations. To do this we make use of a first, second and third order numerical method that is robust to steep gradients, with the first order method being a recreation of the numerical method of \citeN{Hank-etal-2010-2034}. For comparison we use two finite difference schemes, one being the naive one and the other being a recreation of the method of \citeN{El-etal-2006}. Since some papers \cite{Dutykh-2014-315,El-etal-2006} have used smoothing of the initial conditions to handle discontinuities we do the same and investigate the effects of various steepness on the results of our numerical methods as well. We also present analytic comparisons that have been used in the literautre the SWWE analytic results for the dam break problem \cite{Wu-etal-1999-1210} and Whitham modulation results by \citeN{El-etal-2006}.

The paper is organised as follows: The Serre equations are given as well as some important properties for validation, a reproducable explanation of the two FD methods are given as well as the reformulation of the Serre equations into conservative form, then some numerical results are given for the soliton problem to validate the FD methods and then our investigation into the behaviour of the Serre equautions applied to the dam break problem. 

%--------------------------------------------------------------------------------
\section{Serre Equations}
\label{section:Serre Equations}
The Serre equations can derived by integrating the full incompressible Euler equations over the water depth, see for example \citeN{Su-Gardener-1969-536}. They can also be derived as an asymptotic expansion of the Euler equations, see for example \citeN{Bonneton-Lannes-2009-16601}. Assuming a constant horizontal bed the Serre equations are \cite{Guyenne-etal-2014-169}
\begin{linenomath*}
\begin{subequations}\label{eq:Serre_nonconservative_form}
\begin{gather}
\dfrac{\partial h}{\partial t} + \dfrac{\partial (uh)}{\partial x} = 0
\label{eq:Serre_continuity}
\end{gather}
and
\begin{gather}
\underbrace{\underbrace{\dfrac{\partial (uh)}{\partial t} + \dfrac{\partial}{\partial x} \left ( u^2h + \dfrac{gh^2}{2}\right )}_{\text{Shallow Water Wave Equations}} + \underbrace{\dfrac{\partial}{\partial x} \left (  \dfrac{h^3}{3} \left [ \dfrac{\partial u }{\partial x} \dfrac{\partial u}{\partial x} - u\dfrac{\partial^2 u}{\partial x^2}  - \dfrac{\partial^2 u}{\partial x \partial t}\right ] \right )}_{\text{Dispersion Terms}} = 0.}_{\text{Serre Equations}}
\label{eq:Serre_momentum}
\end{gather}
\end{subequations}
\end{linenomath*}
Where $u$ is the average horizontal velocity over the depth of water $h$ and $g$ is the acceleration due to gravity. 

\subsection{Conservation Laws}
The Serre equations conserve a number of physical quantities, thus our numerical methods should reflect this. The total amount of a quantity $q$ in a system is measured by

\begin{gather}
\label{eqn:Condef}
\mathcal{C}_q(t) = \int_{-\infty}^{\infty} q\, dx
\end{gather}
so that we have for all $t$ both $\mathcal{C}_{h}(0) = \mathcal{C}_{h}(t)$, $\mathcal{C}_{uh}(0) = \mathcal{C}_{uh}(t)$ and $\mathcal{C}_{\mathcal{H}}(0) = \mathcal{C}_{\mathcal{H}}(t)$  which represents conservation of mass, momentum and the Hamiltonian \cite{Li-Y-2002,Hank-etal-2010-2034,Green-Naghdi-1976-237} respectively. The Hamiltonian is
\begin{gather}
\label{eqn:Hamildef}
\mathcal{H}(x,t) = \frac{1}{2} \left(hu^2 + \frac{h^3}{3} \left(\frac{\partial u}{\partial x}\right)^2 + gh^2\right).
\end{gather}
The Hamiltonian represents the energy in the system and is the sum of the kinetic energies in the horizontal ($hu^2$) and vertical ($\frac{h^3}{3} \left(\frac{\partial u}{\partial x}\right)^2$) directions and the gravitational potential energy ($gh^2$).   

%--------------------------------------------------------------------------------
\section{Direct Numerical Methods for Solving the Serre Equations} 
\label{sec:DirNumMet}
%--------------------------------------------------------------------------------
The presence of the mixed spatial and temporal derivatives in the momentum equation \eqref{eq:Serre_momentum} makes the Serre equations difficult to solve with standard numerical methods. A naive way to avoid this is to approximate \eqref{eq:Serre_momentum} by finite differences, the results of this are presented here.
\subsection{Finite Difference Appximation to Conservation of Momentum Equation} 
\label{subsec:FDA2conmom}
\citeN{Zoppou-etal-2017} demonstrated that a numerical scheme for solving the Serre equations must be at least second-order accurate. Therefore, the derivatives in \eqref{eq:Serre_momentum} will be approximated by second-order finite differences. Firstly \eqref{eq:Serre_momentum} must be expanded, making use of \eqref{eq:Serre_continuity} one obtains
\begin{linenomath*}
\begin{subequations}
\begin{gather}
h\dfrac{\partial u}{\partial t} + X - h^2\frac{\partial^2 u}{\partial x \partial t} - \frac{h^3}{3}\frac{\partial^3 u}{\partial x^2 \partial t}  =0 
\label{eq:expandedu}
\end{gather}
where $X$ contains only spatial derivatives and is
\begin{gather}
X = uh\frac{\partial u}{\partial x} + gh\frac{\partial h}{\partial x} + h^2\frac{\partial u}{\partial x}\frac{\partial u}{\partial x} + \frac{h^3}{3}\frac{\partial u}{\partial x}\frac{\partial^2 u}{\partial x^2} - h^2u\frac{\partial^2 u}{\partial x^2}- \frac{h^3}{3}u\frac{\partial^3 u}{\partial x^3} .
\end{gather}
\end{subequations}
\end{linenomath*} 
Taking the second-order centred finite difference approximation on a uniform grid in space and time to \eqref{eq:expandedu} gives
\begin{linenomath*}
\begin{gather}
h^{n}_iu^{n+1}_i - \left(h^{n}_i\right)^2 \left(\frac{u^{n+1}_{i+1} -u^{n+1}_{i-1} }{2 \Delta x}\right) - \frac{\left(h^{n}_i\right)^3}{3}\left(\frac{u^{n+1}_{i+1} - 2u^{n+1}_{i} + u^{n+1}_{i-1} }{\Delta x^2}\right) = - Y^n_i
\label{eq:expandedutdisc3}
\end{gather}
\end{linenomath*}
and
\begin{linenomath*}
\begin{gather*}
Y_i^n = 2\Delta tX_i^{n} - h_i^{n}u_i^{n-1} + \left(h_i^{n}\right)^2\left(\frac{u^{n-1}_{i+1} -u^{n-1}_{i-1} }{2 \Delta x}\right) + \frac{\left(h_i^{n}\right)^3}{3}\left(\frac{u^{n-1}_{i+1} - 2u^{n-1}_{i} + u^{n-1}_{i-1} }{\Delta x^2}\right)
\label{eq:expandfactor Xp}
\end{gather*}
\end{linenomath*}
where $\Delta x = x_{i+1} - x_i \; \forall i$, $\Delta t = t^{n+1} - t^{n} \; \forall n$,  $h_i = h(x_i)$ and $h^n = h(t^n)$. Equation \eqref{eq:expandedutdisc3} can be rearranged into an explicit update scheme for $u$ given its current and previous values, so that
\begin{linenomath*}
\begin{gather}
\left[\begin{array}{c}
 u^{n+1}_0 \\
 \vdots \\
 u^{n+1}_m \end{array}\right]
 = A^{-1} \left[\begin{array}{c}
  -Y^n_0 \\
  \vdots \\
  -Y^n_m \end{array}\right] =: \mathcal{G}_u\left(\boldsymbol{u}^n,\boldsymbol{h}^n, \boldsymbol{u}^{n-1}, \Delta x, \Delta t \right)
\label{eq:FDcentforu}
\end{gather}
\end{linenomath*}
where $A$ is a tri-diagonal matrix.

\subsection{The Lax Wendroff Method for Conservation of Mass Equation}
\label{section:}
The conservation of mass equation \eqref{eq:Serre_continuity} has no mixed derivative term allowing standard numerical techniques for conservation laws to be used. In particular the Lax-Wendroff method as was done by \citeN{El-etal-2006}, here we present the method in replicable detail.

Note that \eqref{eq:Serre_continuity} is in conservative law form for $h$ where the flux is $uh$. Thus the two step Lax-Wendroff update for $h$ is
\begin{linenomath*}
\begin{gather}
h^{n + 1/2}_{i+ 1/2} = \frac{1}{2}\left(h^{n}_{i+1} + h^{n}_i\right) - \frac{\Delta t}{2\Delta x}\left(u^n_{i+1}h^n_{i+1} - h^n_{i}u^n_{i}\right),
\end{gather}
\begin{gather}
h^{n + 1/2}_{i- 1/2} = \frac{1}{2}\left(h^{n}_{i} + h^{n}_{i-1}\right) - \frac{\Delta t}{2\Delta x}\left(u^n_{i}h^n_{i} - h^n_{i-1}u^n_{i-1}\right)
\end{gather}
and
\begin{gather}
h^{n+1}_i = h^{n}_i - \frac{\Delta t}{\Delta x}\left(u^{n + 1/2}_{i+ 1/2}h^{n + 1/2}_{i+ 1/2} - u^{n + 1/2}_{i- 1/2}h^{n + 1/2}_{i- 1/2}\right).
\label{eq:LW4h}
\end{gather}
\end{linenomath*}
The quantities $u^{n + 1/2}_{i \pm 1/2}$ are calculated using $u^{n+1}$ obtained by appling $\mathcal{G}_u$ to $u^n$ then linearly interpolating in space and time to give
\begin{gather}
u^{n + 1/2}_{i+ 1/2} = \frac{u^{n+1}_{i+1} + u^{n}_{i+1} + u^{n+1}_{i} + u^{n}_{i} }{4}
\end{gather}
and
\begin{gather}
u^{n + 1/2}_{i- 1/2} = \frac{u^{n}_{i} + u^{n}_{i} + u^{n+1}_{i-1}+ u^{n}_{i-1} }{4}.
\end{gather}
Thus we have the following update scheme for \eqref{eq:Serre_continuity}
\begin{linenomath*}
	\begin{gather}
	\boldsymbol{h}^{n+1} = \mathcal{E}_h\left(\boldsymbol{u}^n,\boldsymbol{h}^n,\boldsymbol{u}^{n+1}, \Delta x, \Delta t \right). 
	\end{gather}
\end{linenomath*}
The update scheme for all of \eqref{eq:Serre_nonconservative_form} is
\begin{linenomath*}
\begin{gather}
	\left.
	\begin{array}{l l}
	\boldsymbol{u}^{n+1}&=\mathcal{G}_u\left(\boldsymbol{u}^n,\boldsymbol{h}^n, \boldsymbol{u}^{n-1}, \Delta x, \Delta t \right) \\
	\boldsymbol{h}^{n+1}&=\mathcal{E}_h\left(\boldsymbol{u}^n,\boldsymbol{h}^n,\boldsymbol{u}^{n+1}, \Delta x, \Delta t \right)
	\end{array} \right\rbrace \mathcal{E}\left(\boldsymbol{u}^n,\boldsymbol{h}^n, \boldsymbol{u}^{n-1},\boldsymbol{h}^{n-1}, \Delta x, \Delta t \right).
\end{gather}
\end{linenomath*}

%--------------------------------------------------------------------------------
\subsection{Second-Order Naive Finite Difference Method}
%--------------------------------------------------------------------------------
Here we also present a completely naive method for comparative purposes, to do this we apply the procedure used above on \eqref{eq:Serre_momentum} to \eqref{eq:Serre_continuity}. The derivatives are expanded then approximated by second-order centered finite differences after rearranging this to give an update formula we obtain
\begin{linenomath*}
\begin{gather}
h^{n+1}_i = h^{n-1}_i - \Delta t \left(u^{n}_{i}\frac{h^{n}_{i+1} - h^{n}_{i-1}}{\Delta x} + h^{n}_{i}\frac{u^{n}_{i+1} - u^{n}_{i-1}}{\Delta x}\right).
\end{gather}
\end{linenomath*}
Preforming this update for all $i$ will be denoted by $\mathcal{G}_h\left(\boldsymbol{u}^n,\boldsymbol{h}^n,\boldsymbol{h}^{n-1} ,\Delta x, \Delta t \right)$.
Thus we get the naive second-order centred finite difference method for the Serre equations
\begin{linenomath*}
\begin{gather}
\left.
\begin{array}{l l}
\boldsymbol{h}^{n+1}&=\mathcal{G}_h\left(\boldsymbol{u}^n,\boldsymbol{h}^n,\boldsymbol{h}^{n-1} \Delta x, \Delta t \right) \\
\boldsymbol{u}^{n+1}&=\mathcal{G}_u\left(\boldsymbol{u}^n,\boldsymbol{h}^n, \boldsymbol{u}^{n-1}, \Delta x, \Delta t \right)
\end{array} \right\rbrace \mathcal{G}\left(\boldsymbol{u}^n,\boldsymbol{h}^n, \boldsymbol{u}^{n-1},\boldsymbol{h}^{n-1}, \Delta x, \Delta t \right).
\end{gather}
\end{linenomath*}
%-------------------------------------------------------------------------------- 
\section{Conservative Form of The Serre Equations}
To overcome the aforementioned difficulty of mixed derivatives the Serre equations \eqref{eq:Serre_nonconservative_form} can be reformulated into conservative form. This was accomplished by the introduction of a new quantity \cite{Hank-etal-2010-2034,Zoppou-2014}
\begin{linenomath*}
\begin{gather}
\label{eq:Gdefinition}
G = uh - h^2 \dfrac{\partial h}{\partial x} \dfrac{\partial u}{\partial x} - \frac{h^3}{3} \dfrac{\partial^2 u}{\partial x^2}.
\end{gather}
\end{linenomath*}
Consequently, \eqref{eq:Serre_nonconservative_form} can be rewritten as
\begin{linenomath*}
\begin{subequations}
\begin{gather}
\dfrac{\partial h}{\partial t} + \dfrac{\partial (uh)}{\partial x} = 0
\label{eq:Serrecon_continuity}
\end{gather}
and
\begin{gather}
\dfrac{\partial G}{\partial t} + \dfrac{\partial}{\partial x}\left(Gu + \dfrac{gh^2}{2} - \dfrac{2h^3}{3}\dfrac{\partial u}{\partial x}\dfrac{\partial u}{\partial x}\right) = 0.
\label{eq:Serrecon_momentum}
\end{gather}
\label{eq:Serrecon}
\end{subequations}
\end{linenomath*}

\subsection{A Hybrid Finite Difference-Volume Method for Serre Equations in Conservative Form}
\label{section:hybridmethod}
%--------------------------------------------------------------------------------
The conservative form \eqref{eq:Serrecon} allows for a wider range of numerical techniques such as finite element methods \cite{Guyenne-etal-2014-169} and finite volume methods \cite{Hank-etal-2010-2034,Zoppou-2014}. In this paper the first ($\mathcal{V}_1$), second ($\mathcal{V}_2$) and third-order ($\mathcal{V}_3$) finite difference-volume methods (FDVM) of \citeN{Zoppou-etal-2017} will be used. These have been validated and their order of accuracy confirmed.

\subsection{Stability Condition} 
To ensure stability of the FDVMs the time-step $\Delta t$ must satisfy the Courant-Friedrichs-Lewy (CFL) criteria \cite{Harten-etal-1983-357}

\begin{gather}
\label{eq:CFL}
\Delta t < \frac{Cr \Delta x}{2\max \left\lbrace |\lambda| \right\rbrace}
\end{gather}

 with $0<Cr\le 1$ where $\lambda$ is the charachteristic speed. For the Serre equations it has been demonstrated that the wave speed is bounded by the charachteristic speed of the Shallow Water Wave equations \cite{Hank-etal-2010-2034}.
 [stability for the twwo FD methods][]

\section{Numerical Simulations}
\label{section:Numerical Simulations}
%--------------------------------------------------------------------------------
In this section the methods introduced in this paper will be validated by using them to approximate an analytic solution of the Serre equations, this will also be used to verify their order of accuracy. Then an in-depth comparison of these methods for a smooth approximation to the discontinuous dam break problem will be provided to investigate the behaviour of these numerical schemes in the presence of steep gradients. This is a problem that so far has only received a proper treatment in \cite{El-etal-2006}, with other research giving only a cursory investigation into the topic. [][refs]

%--------------------------------------------------------------------------------
\section{Soliton}
\label{section:Convergence Rate}
%--------------------------------------------------------------------------------
Currently cnoidal waves are the only family of analytic solutions to the Serre equations \cite{Carter-Cienfuegos-2010-259}. Solitons are a particular instance of cnoidal waves that travel without deformation which have been used to verify the convergence rates of the described methods in this paper. 

The solitons of the Serre equations have the following form
\begin{linenomath*}
\begin{subequations}
\begin{gather}
h\left(x,t\right) = a_0 + a_1\text{sech}^2\left( \kappa\left(x - ct\right)\right),
\end{gather}
\begin{gather}
u\left(x,t\right) = c\left(1 - \dfrac{a_0}{h(x,t)} \right),
\end{gather}
\begin{gather}
\kappa = \dfrac{\sqrt{3a_1}}{2a_0 \sqrt{ a_0 + a_1}}
\end{gather}
and
\begin{gather}
c = \sqrt{g \left(a_0 + a_1\right)}
\end{gather}
\end{subequations}
\label{eq:sol}
\end{linenomath*}
where $a_0$ and $a_1$ determine the depth of the quiescent water and the maximum height of the soliton above that respectively. In the simulation $a_0 = 1\text{m}$, $a_1 = 0.7\text{m}$ for $x\in\left[-50\text{m},250\text{m}\right]$ and $t\in\left[0\text{s},50\text{s}\right]$. With $\Delta t = 0.5 \lambda^{-1} \Delta x$ where $\lambda = \sqrt{g \left(a_0 + a_1\right)}$ which is the maximum wave speed, this satisfies the CFL condition \eqref{eq:CFL}. [draw this examples comparison to zoppou paper to understand why we use this constraint][] 

%--------------------------------------------------------------------------------
\subsection{Results}
%--------------------------------------------------------------------------------
This numerical experiment and its results for the FDVM have been reported by \citeN{Zoppou-etal-2017}, this paper only reports the results for $\mathcal{G}$ and $\mathcal{E}$. From Figure \ref{fig:FDsolh} it can be seen that $\mathcal{G}$ and $\mathcal{E}$ accurately model the highly non-linear soliton problem reproducing the analytic solution up to graphical accuracy.

To demonstrate that in fact $\mathcal{E}$ and $\mathcal{G}$ are consistent, two measures were used. The first measures the relative distance of the numerical results from the analytic solution for $h$ and $u$, it is defined for a general quantity $q$ and an approximation to it $q^*$ at $n$ values as
\begin{gather}
L_1 = \dfrac{\sum_{i = 1}^{n} \left| q_i - q^*_i\right|}{\sum_{i = 1}^{n} \left| q_i\right|}.
\end{gather}
The second measures how well the schemes conserve a quantity $q$ over time
\begin{gather}
C_1 = \dfrac{\left| \mathcal{C}_q(0) - \mathcal{C}_{q^*}(t_f) \right|}{\left| \mathcal{C}_q(0) \right|}
\end{gather}
where $t_f$ is the final time of the numerical experiment. For $\mathcal{C}_q(0)$ the analytic value is used while a numerical calculation is used for $\mathcal{C}_{q^*}(t_f)$ based on summing cell-wise integrals. Since all our schemes have the cell-wise average up to sufficient order for $h$ and $u$,  we multiply the cell-wise average by $\Delta x$ to obtain the cell-wise integral for $h$ and $uh$. The cell-wise integral for $\mathcal{H}$ is calculated by quartic interpolation of $h$ and $u$ utilising neighbouring cells and then applying Gaussian quadrature with 3 points over the cell to get a sufficiently high order approximation, in particular it is at least third-order accurate for the $\partial u / \partial x$ term. 
%
\begin{figure}
\centering
\subfigure[][]{\label{fig:FDsolh}\includegraphics[width=7cm]{pics/results/soliton/ex/FDc.pdf}}
\subfigure[][]{\label{fig:FDsolh}\includegraphics[width=7cm]{pics/results/soliton/ex/FDcz.pdf} }
\caption{Water profile for the soliton problem \eqref{eq:sol} for $\mathcal{G}$ when $\Delta x = 10/2^{12}$ with the initial conditions ({\color{black} \solidrule}), analytic solution ({\color{blue} \solidrule}) and numerical result ({\color{red} $\bullet$}).}
\subfigure[][]{\label{fig:GRsolh}\includegraphics[width=7cm]{pics/results/soliton/ex/grim.pdf}\includegraphics[width=7cm]{pics/results/soliton/ex/grimz.pdf}}
\caption{Water profile for the soliton problem \eqref{eq:sol} for $\mathcal{E}$ when $\Delta x = 10/2^{12}$ with the initial conditions ({\color{black} \solidrule}), analytic solution ({\color{blue} \solidrule}) and numerical result ({\color{red} $\bullet$}).}
\label{fig:FDMsolexp}
\end{figure}
\begin{figure}
\centering
\subfigure[][]{\label{fig:FDo2normL1}\includegraphics[width=7cm]{pics/results/soliton/L1/FDc.pdf}\includegraphics[width=7cm]{pics/results/soliton/C1/FDc.pdf}}
\subfigure[][]{\label{fig:grimo2normL1}\includegraphics[width=7cm]{pics/results/soliton/L1/grim.pdf} \includegraphics[width=7cm]{pics/results/soliton/C1/grim.pdf}}
\caption{On the left $L_1$ for $h$ ({\color{red} $\triangle$}) and $u$ ({\color{blue} $\square$}) and on the right $C_1$ for $h$ ({\color{red} $\triangle$}), $uh$ ({\color{black} $\diamond$}) and $\mathcal{H}$ ({\color{blue} $\circ$}) for the numerical solution of the soliton problem by $\mathcal{G}$(a) and $\mathcal{E}$(b).}
\label{fig:FDMsolnorm}
\end{figure}
%
From Figure \ref{fig:FDMsolnorm} it can be seen that both FD methods are convergent under $L_1$ with second-order accuracy. There is however suboptimal rates of convergence for very small $\Delta x$ due to round off effects and large $\Delta x$ due to the initial conditions not being accurately represented on a coarse grid.  

Figure \ref{fig:FDMsolnorm} demonstrates conservation of mass, momentum and the Hamiltonian to at least second-order accuracy for both FD schemes. Both schemes conserve mass very well with round off error dominance occurring at the same place as for $L_1$. Momentum has the appropriate order of accuracy for larger $\Delta x$ but then stagnates as $\Delta x$ decreases. Surprisingly the Hamiltonian is conserved with an order of accuracy greater than the momentum, with the effects of round off errors occurring earlier due to the greater number of calculations.

All of these measures demonstrate that $\mathcal{G}$ and $\mathcal{E}$ are appropriate to solve highly non-linear problems with smooth initial conditions for the Serre equations. 

%--------------------------------------------------------------------------------
\section{Smoothed Dam-Break}
\label{section:smootheddambreak}
%--------------------------------------------------------------------------------
The discontinuous dam-break problem can be approximated smoothly using the hyperbolic tangent function. Such an approximation will be called a smoothed dam-break problem and will be defined as such
\begin{linenomath*}
\begin{subequations}
\begin{gather}
h(x,0) = h_0 + \frac{h_1 - h_0}{2}\left(1 + \tanh\left(\frac{x_0 - x}{\alpha}\right)\right),
\end{gather}
\begin{gather}
u(x,0) = 0.0m/s.
\end{gather}
\end{subequations}
\label{eq:sdbi}
\end{linenomath*}
Where $\alpha$ is given and controls the width of the transition between the two dam-break heights of $h_0$ and $h_1$. $\alpha$ measures the distance over which $46.117\%$ of the transition between $h_1$ and $h_1$ centered around $x_0$ occurs. Figure \ref{fig:dbsmoothinit} demonstrates various smoothed dam break problems with different $\alpha$ values.
\begin{figure}
\centering
\includegraphics[width=9cm]{pics/explainers/dbs.pdf}
\caption{Initial conditions for the smooth dam-break problem as $\alpha$ varies.}
\label{fig:dbsmoothinit}
\end{figure}
%

The dam break problem for the Serre equations results in the creation of an undular bore that is very similar to the analytic solution of the dam break problem for the SWWE with oscillations occurring on top \cite{Hank-etal-2010-2034}. Because of this some values from the analytic solution to the dam break problem for the SWWE will be used as a reference in this paper; these are the height ($h_2$) and velocity ($u_2$) in the shock as well as the speed of the shock front ($S_2$).  From \citeN{Wu-etal-1999-1210} we have the following equations
\begin{linenomath*}
\begin{gather}
h_2 = \frac{h_0}{2} \left[\sqrt{1 + 8 \left(\frac{2h_2}{h_2 - h_0}\frac{\sqrt{gh_1} - \sqrt{gh_2}}{\sqrt{gh_0}}\right)^2} - 1\right],
\label{eq:h2def}
\end{gather}
\end{linenomath*}
\begin{linenomath*}
\begin{gather}
u_2 = 2\left(\sqrt{gh_1} - \sqrt{gh_2}\right)
\label{eq:u2def}
\end{gather}
\end{linenomath*}
and
\begin{linenomath*}
\begin{gather}
S_2 = \frac{h_2 u_2}{h_2 - h_0}.
\label{eq:S2def}
\end{gather}
\end{linenomath*}


Undular bores for the one dimensional Serre equations were analysed by \citeN{El-etal-2006} and an expression for the amplitude ($a^+$) and speed ($S^+$) of the leading wave of a bore was given
\begin{linenomath*}
\begin{gather}
\frac{\Delta}{\left(a^+ + 1\right)^{1/4}} - \left(\frac{3}{4 -  \sqrt{a^+ + 1}}\right)^{21/10} \left(\frac{2}{1 + \sqrt{a^+ + 1}}\right)^{2/5} = 0
\label{eq:aplusdef}
\end{gather}
\end{linenomath*}
and
\begin{linenomath*}
\begin{gather}
S^+ = \sqrt{g \left(a^+ + 1\right)}
\label{eq:splusdef}
\end{gather}
\end{linenomath*}
where $\Delta = h_r / h_0$, and $h_r$ is the amplitude of the bore.

\begin{figure}
\centering
\includegraphics[width=7cm]{pics/explainers/SWWEana.pdf}
\caption{Analytic solution at $t=30s$ of the SWWE for the dam-break problem.}
\label{fig:SWWEanadiagram}
\end{figure}

\begin{figure}
\centering
\includegraphics[width=7cm]{pics/explainers/SERREex.pdf}
\caption{Demonstration of quantities obtained by Whitham modulation for undular bores of the Serre equations.}
\label{fig:Serreanadiagram}
\end{figure}

In these experiments $h_0 = 1.0m$, $h_1 = 1.8m$ and $x_0 = 500m$. This scenario replicates one presented by \citeN{El-etal-2006} and \citeN{Hank-etal-2010-2034}. The simulations were run with various values of $\Delta x$ and $\beta$. To ensure stability especially of the FD methods a very restrictive time step of $\Delta t = 0.01 \Delta x$ was chosen and for $\mathcal{V}_2$ $\theta = 1.2$. From this description the Hamiltonian at the initial time is
\begin{gather}
\label{eqn:HamilDBinit}
\mathcal{H} (0) = 10398.6 - 0.7848\times\left[\frac{2}{\alpha} \tanh\left(500 \alpha\right)\right].
\end{gather} 
Applying \eqref{eq:h2def}, \eqref{eq:u2def} and \eqref{eq:S2def} to these initial conditions results in $h_2 = 1.36898m$ , $u_2 = 1.074975$ $m/s$ and $S_2 = 3.98835$ $m/s$. For \eqref{eq:aplusdef} and \eqref{eq:splusdef} the process is a little different because $h_r \neq h_2$ but instead comes from the intersection of the Riemann invariant curve and the centred left propagating rarefaction wave curve \cite{El-etal-2006}, which results in $h_r = 1.37082$ thus $\Delta = 1.37082$,  $a^+ = 1.73998$ $m$ and $S^+ = 4.13148$ $m/s$. Of particular note is that due to the different natures of bores for the Serre and SWW equations $S^+ \neq S_2$.

%%%
%% DOWN HERE 
%%	
%%



%--------------------------------------------------------------------------------
\subsection{Results}
%--------------------------------------------------------------------------------

%Intro
We begin this study by looking into the effect of the initial steepness of the smoothed dam break problem for different $\beta$ values observing what happens as $\Delta x \rightarrow 0$ and our numerical solution better approximates the true solution of the Serre equations. To this end we use the highest order well validated model $\mathcal{V}_3$ in the following investigation. From these results we then investigate numerical results for long time scales, how the SWWE analytic values and Els whitham modulation values compare to our results and then finally present some other findings about the behaviour of our numerical solutions.

\subsubsection{effect of $\beta$}
Because the smoothing process is a non physical numerical tool we will first study its effect by decreasing $\beta$ and thus better approximating the dam break. To do this we fix a $\beta$ and then investigate the numerical solutions as $\Delta x \rightarrow 0$ and our well validated numerical methods better approximate the true solution of the equations. %We note that for a fixed $\beta$ some spatial grids will be too coarse to capture the smooth profile of the initial conditions which can cause problems for the less robust methods $\mathcal{G}$ and $\mathcal{E}$.

The first and most important observation is that there are four types of behaviour as $\Delta x \rightarrow 0$ depending on the $\beta$ and the numerical method. The four scenarios are identified by the behaviour of the solutions when $\Delta x$ is small and they correspond to different results in the literature. For brevity the only given examples of these scenarios will be the solutions of $\mathcal{V}_3$ although they also occurred for $\mathcal{E}$, $\mathcal{G}$ and $\mathcal{V}_2$.

The first behaviour which will be referred to as the non-oscillatory scenario has such smooth initial conditions that no oscillations were introduced by $t= 30s$, although given sufficient time the front steepens and an undular bore develops. An example of this behaviour can be seen in Figure \ref{fig:o3a1dxlimflatexp} for $\beta = 117.778$. Because this is a very smooth problem we observe rapid convergence with all the numerical results being graphically identical. This scenario resembles very diffusive solutions of the SWWE in that it contains only a rarefaction and a shock with no dispersive waves. 

Convergence is also present in Figure \ref{fig:o3a1dxlimmeasure} with both the $L_1$ and $H_1$ measures. However, $L_1$ has been modified to use the solution of the smallest $\Delta x$ as an approximation to the analytic solution because none are currently known. For both measures the order of accuracy is the theoretical one, with round-off errors becoming dominant for small $\Delta x$. Since $L_1$ now compares only numerical results, round-off errors result in error stagnation rather than increase. For $H_1$ it can be seen that round-off errors are dominant earlier than in $L_1$ this is because $H_1$ requires many more calculations. This suggests that this family of solutions is also a true representation of the behaviour of the Serre equations when $\beta$ is sufficiently large and in particular $\beta = 117.778$. 


\begin{figure}
\centering
\subfigure[][]{\label{fig:o3a1dxlim}\includegraphics[width=10cm]{pics/results/SDB/numsols/alpha0.025/1-figure0.pdf}}
\caption{Numerical results of $\mathcal{V}_3$  at $t= 30s$ for the smooth dam break problem with $\alpha = 40$ for $\Delta x = 10/2^{4}$ ({\color{black} \solidrule}).}
\label{fig:o3a1dxlimflatexp}
\end{figure}
%
\begin{figure}
\centering
\subfigure[][]{\label{fig:o3a1dxlimL1}\includegraphics[width=7cm]{pics/results/SDB/Lcon/alpha0.025/1.pdf}}
\subfigure[][]{\label{fig:o3a1dxlimH}\includegraphics[width=7cm]{pics/results/SDB/Con/1.pdf}}
\caption{(a) shows $L_1$ for $h$ ({\color{red} $\triangle$}) and $u$ ({\color{blue} $\square$}) and (b) shows $C_1$ for $h$ ({\color{red} $\triangle$}), $uh$ ({\color{black} $\diamond$}) and $\mathcal{H}$ ({\color{blue} $\circ$}) for $\mathcal{V}_3$'s solution for the smooth dam-break problem with $\alpha = 40$.}
\label{fig:o3a1dxlimmeasure}
\end{figure}

The second scenario will be referred to as the flat scenario due to the presence of a constant height state between the oscillations at the shock and rarefaction fan. An example of the numerical results for this scenario can be seen in Figure \ref{fig:o3a6dxlimflatexp} when $\beta = 5.8889$. This scenario corresponds to the results presented by \citeN{Hank-etal-2010-2034} and \citeN{Mitsotakis-etal-2014}. 

As $\Delta x$ decreases the solutions converge so that by $\Delta x = 10 / 2^8$ the solutions for higher $\Delta x$ are visually identical. There is also good agreement between the amplitude of the leading soliton and $a^+$ as well as the plateau height and $h_2$. Although as $\Delta x$ is decreased the plateau seems to be slightly above this value. Since this method is well validated for smooth problems and a small $\Delta x$ has been chosen this suggests that the bore from a dam break problem may differ slightly for the Serre and SWWE although they are still quite close. These results also compare well to the results in \citeN{Mitsotakis-etal-2014} who use the same $\beta$ but different $h_0$ and $h_1$. 

The measures $L_1$ and $H_1$ also demonstrate good convergence with the expected order of accuracy in the middle of the plot. Suboptimal convergence is expected for large $\Delta x$ as the problem is not sufficiently resolved to model the oscillations and so both $H_1$ and $L_1$ suffer. For small $\Delta x$ the measure $H_1$ becomes suboptimal due to round-off errors however this effect is masked by $L_1$ as a numerical solution is the base of the comparison instead of an analytic result.

\begin{figure}
\centering
\subfigure[][]{\label{fig:o3a6dxlimz1}\includegraphics[width=10cm]{pics/results/SDB/numsols/alpha0.5/1-figure0.pdf}}
\subfigure[][]{\label{fig:o3a6dxlimz2}\includegraphics[width=7cm]{pics/results/SDB/numsols/alpha0.5/2-figure0.pdf}}
\caption{Numerical results of $\mathcal{V}_3$  at $t= 30s$ for the smooth dam-break problem with $\alpha = 2$ for $\Delta x = 10/2^{10}$ ({\color{blue} \solidrule}), $10/2^8$ ({\color{red} \solidrule}), $10/2^6$ ({\color{green!60!black} \solidrule}) and $10/2^{4}$ ({\color{black} \solidrule}).}
\label{fig:o3a6dxlimflatexp}
\end{figure}

\begin{figure}
	\centering
	\subfigure[][]{\label{fig:o3a2dxlimL1}\includegraphics[width=7cm]{pics/results/SDB/Lcon/alpha0.5/1.pdf}}
	\subfigure[][]{\label{fig:o3a2dxlimH}\includegraphics[width=7cm]{pics/results/SDB/Con/6.pdf}}
	\caption{(a) shows $L_1$ for $h$ ({\color{red} $\triangle$}) and $u$ ({\color{blue} $\square$}) and (b) shows $C_1$ for $h$ ({\color{red} $\triangle$}), $uh$ ({\color{black} $\diamond$}) and $\mathcal{H}$ ({\color{blue} $\circ$}) for $\mathcal{V}_3$'s solution for the smooth dam-break problem with $\alpha = 2$.}
	\label{fig:o3a2dxlimmeasure}
\end{figure}

The third scenario will be referred to as the contact discontinuity \cite{El-etal-2006} scenario. The contact discontinuity scenarios main feature is that the oscillations from the rarefaction fan and the shock decay and appear to meet at a point as can be seen in Figure \ref{fig:o3a9dxlimcdexp} when $\beta = 1.1778$. All the higher order methods so far have not shown a converged solution as $\Delta x$ decreases. However, it does appear that convergence is likely with the solutions getting closer together, especially since for the smaller $\Delta x$ this problem is still smooth. These results also compare very well in terms of the lead soliton amplitude and the bore height reference values given on the plots. This scenario was observed by \citeN{El-etal-2006} for $\mathcal{E}$ and indeed we have replicated them for all the high order methods in this paper. The necessity of a $\beta$ lower than $5.8889$ to recover the `contact discontinuity' explains why \cite{Mitsotakis-etal-2014} could not replicate the results of \cite{El-etal-2006}. 

The assertion that these results are close to converged is supported by Figure \ref{fig:o3a3dxlimmeasure} for the $L^*_1$ and $H_1$ measured. As can be seen in Figure \ref{fig:o3a9dxlimz2} the final solutions have not yet even graphically converged, thus we modify $L_1$ to omit this section from $[520m,540m]$ and call this modified measure $L^*_1$. Thus $L^*_1$ demonstrates that even though this middle section has not been fully resolved we do see that there is convergence at the appropriate order outside this region. Suggesting that the effect of better resolving this contact discontinuity will only be felt locally around the contact discontinuity and not significantly change the solution away from it. $H_1$ demonstrates the appropriate order of accuracy in the Hamiltonian demonstrating that we are indeed approaching a solution to this problem as $\Delta x$ is increased. 

\begin{figure}
\centering
\subfigure[][]{\label{fig:o3a9dxlim}\includegraphics[width=10cm]{pics/results/SDB/numsols/alpha2.5/1-figure0.pdf}}
\subfigure[][]{\label{fig:o3a9dxlimz1}\includegraphics[width=7cm]{pics/results/SDB/numsols/alpha2.5/2-figure0.pdf}}
\subfigure[][]{\label{fig:o3a9dxlimz2}\includegraphics[width=7cm]{pics/results/SDB/numsols/alpha2.5/3-figure0.pdf}}
\caption{Numerical results of $\mathcal{V}_3$  at $t= 30s$ for the smooth dam-break problem with $\alpha = 0.4$ for $\Delta x = 10/2^{10}$ ({\color{blue} \solidrule}), $10/2^8$ ({\color{red} \solidrule}), $10/2^6$ ({\color{green!60!black} \solidrule}) and $10/2^{4}$ ({\color{black} \solidrule}).}
\label{fig:o3a9dxlimcdexp}
\end{figure}

\begin{figure}
	\centering
	\subfigure[][]{\label{fig:o3a3dxlimL1}\includegraphics[width=7cm]{pics/results/SDB/Lcon/alpha2.5/1.pdf}}
	\subfigure[][]{\label{fig:o3a3dxlimH}\includegraphics[width=7cm]{pics/results/SDB/Con/9.pdf}}
	\caption{(a) shows $L^*_1$ for $h$ ({\color{red} $\triangle$}) and $u$ ({\color{blue} $\square$}) and (b) shows $C_1$ for $h$ ({\color{red} $\triangle$}), $uh$ ({\color{black} $\diamond$}) and $\mathcal{H}$ ({\color{blue} $\circ$}) for $\mathcal{V}_3$'s solution for the smooth dam-break problem with $\alpha = 0.4$.}
	\label{fig:o3a3dxlimmeasure}
\end{figure}


The fourth scenario will be referred to as the bump scenario due to the oscillations no longer decaying down towards a point but rather growing around the contact discontinuity forming a bump as can be seen in Figure \ref{fig:o3a20dxlimcdexp} for $\beta = 0.294$. This behaviour has hitherto not been published and is certainly not an expected result. 

This scenario is even further from graphical convergence in $\Delta x$ around the contact discontinuity than the previous scenario as can be seen in Figure \ref{fig:o3a20dxlimcdexp}. $L^*_1$ demonstrates good convergence outside this middle region as can be seen in Figure \ref{fig:o3a4dxlimmeasure}. $H_1$ also converges but only has the appropriate order of accuracy in the last few $\Delta x$ points. This suggests that to properly resolve this scenario requires smaller grids or higher-order schemes. Because, convergence is not assured by these numerical results there is the possibility that the wave amplitudes around the contact discontinuity could explode. This however has not been observed, with numerical results where $\beta = 0.00294$ and $\Delta x = 10.0/ 2^{10}m = 0.009765625m$ at which point the initial conditions are basically a discontinuous dam break showing an increase but not an explosion in amplitude.

\begin{figure}
\centering
\subfigure[][]{\label{fig:o3a20dxlim}\includegraphics[width=10cm]{pics/results/SDB/numsols/alpha10/1-figure0.pdf}}
\subfigure[][]{\label{fig:o3a20dxlimz1}\includegraphics[width=7cm]{pics/results/SDB/numsols/alpha10/2-figure0.pdf}}
\subfigure[][]{\label{fig:o3a20dxlimz2}\includegraphics[width=7cm]{pics/results/SDB/numsols/alpha10/3-figure0.pdf}}
\caption{Numerical results of $\mathcal{V}_3$  at $t= 30s$ for the smooth dam-break problem with $\alpha = 0.1$ for $\Delta x = 10/2^{10}$ ({\color{blue} \solidrule}), $10/2^8$ ({\color{red} \solidrule}), $10/2^6$ ({\color{green!60!black} \solidrule}) and $10/2^{4}$ ({\color{black} \solidrule}).}
\label{fig:o3a20dxlimcdexp}
\end{figure}

\begin{figure}
	\centering
	\subfigure[][]{\label{fig:o3a4dxlimL1}\includegraphics[width=7cm]{pics/results/SDB/Lcon/alpha10/1.pdf}}
	\subfigure[][]{\label{fig:o3a4dxlimH}\includegraphics[width=7cm]{pics/results/SDB/Con/9.pdf}}
	\caption{(a) shows $L_1^*$ for $h$ ({\color{red} $\triangle$}) and $u$ ({\color{blue} $\square$}) and (b) shows $C_1$ for $h$ ({\color{red} $\triangle$}), $uh$ ({\color{black} $\diamond$}) and $\mathcal{H}$ ({\color{blue} $\circ$}) for $\mathcal{V}_3$'s solution for the smooth dam-break problem with $\alpha = 0.1$.}
	\label{fig:o3a4dxlimmeasure}
\end{figure}

% % % % % % % % % % % % % % % % % % % % MOST WORK DOWN HERE!!!! % % % % % % % % % % % % % % % % % % % % % %
% % % % % % % % % % % % % % % % % % % % % %% % % % % % % % % % % % % % % % % % % % % %% % % % % % % % % % %
% % % % % % % % % % % % % % % % % % % % % %% % % % % % % % % % % % % % % % % % % % % %% % % % % % % % % % %
% % % % % % % % % % % % % % % % % % % % % %% % % % % % % % % % % % % % % % % % % % % %% % % % % % % % % % %
% % % % % % % % % % % % % % % % % % % % % %% % % % % % % % % % % % % % % % % % % % % %% % % % % % % % % % %
% % % % % % % % % % % % % % % % % % % % % %% % % % % % % % % % % % % % % % % % % % % %% % % % % % % % % % %

%smoothing for El as well
%add S_2

Since this result is unexpected and not as supported as the contact discontinuity scenario in the literature \cite{El-etal-2006,Gurevich-Meshcherkin-1984-1277}. The first check should be different numerical methods such as $\mathcal{G}$ and $\mathcal{E}$ to test if some numerical effect from the reformulation of the Serre equations or the elliptic solver are the cause. For comparison all methods discussed in this paper are applied to the same initial conditions and grid resolutions as above are plotted in Figure \ref{fig:MODlim}. The first observation is that $\mathcal{V}_1$ has not recovered this behaviour. This is because as noted by \citeN{Zoppou-etal-2017}, $\mathcal{V}_1$ is very diffusive, dampening these oscillations. To resolve such behaviour for $\mathcal{V}_1$ would require very small $\Delta x$ and as such we have not seen this behaviour yet. The diffusivity of the first-order scheme is the reason why \citeN{Hank-etal-2010-2034} could not replicate the results of \citeN{El-etal-2006} with reasonable $\Delta x$. Secondly, all high order methods recover this bump behaviour and disagree only in the region around the contact discontinuity. The main difference in the oscillations is their phase and amplitude with the dispersive FD methods resulting in larger waves than the diffusive FDVM. 

Dispersive methods decrease oscillation amplitude and number as $\Delta x$ is decreased as can be seen in Figure \ref{fig:FDa6lim}. Since $\mathcal{V}_3$ is diffusive as in Figure \ref{fig:o3a20dxlimcdexp} the true analytic solution should then exist between $\mathcal{V}_3$ and $\mathcal{G}$, which is a bounded bump around the contact discontinuity. Finally it can be seen that $\mathcal{V}_2$ and $\mathcal{V}_3$ are similar. This is because as noted by \citeN{Zoppou-etal-2017} $\mathcal{V}_3$ is not a substantially better method than $\mathcal{V}_2$ and so their results are going to be quite similar. $\mathcal{G}$ well approximates the Serre equations, although the FDVM are still preferred by the authors due to robustness and conservation of quantities as can be seen in Figure \ref{fig:FDMsolnormC1}. Figure \ref{fig:o3a4dxHallsign} demonstrates that the $\mathcal{V}_i$ schemes result in $\mathcal{H}(30s) < \mathcal{H}(0s)$ so energy is only lost where as $\mathcal{G}$ and $\mathcal{E}$ can gain energy and are therefore undesirable.

\begin{figure}
\centering
\subfigure[][]{\label{fig:MODh}\includegraphics[width=10cm]{pics/results/SDB/numsols/modelcomppalpha10dx10/1-figure0.pdf}}
\subfigure[][]{\label{fig:MODh1}\includegraphics[width=7cm]{pics/results/SDB/numsols/modelcomppalpha10dx10/2-figure0.pdf}}
\subfigure[][]{\label{fig:MODh2}\includegraphics[width=7cm]{pics/results/SDB/numsols/modelcomppalpha10dx10/4-figure0.pdf}}
\caption{Numerical results for the smooth dam break problem with $\alpha = 0.1$ and $\Delta x = 10/2^{10}$
for $\mathcal{G}$ ({\color{blue} \solidrule}), $\mathcal{E}$ ({\color{red} \solidrule}), $\mathcal{V}_3$ ({\color{green!60!black} \solidrule}), $\mathcal{V}_2$ ({\color{black} \solidrule}) and $\mathcal{V}_1$ ({\color{violet!80!white} \solidrule}).}
\label{fig:MODlim}
\end{figure}

%Replace with a statement abouit this , dont need to demonstrate



\begin{figure}
\centering
\subfigure[][]{\label{fig:FDa6h}\includegraphics[width=7cm]{pics/results/SDB/numsols/FDcalpha0.5/1-figure0.pdf}}
\subfigure[][]{\label{fig:FDa6hz}\includegraphics[width=7cm]{pics/results/SDB/numsols/FDcalpha0.5/2-figure0.pdf}}
\caption{Numerical results of $\mathcal{G}$  at $t= 30s$ for the smooth dam-break problem with $\alpha = 2$ for $\Delta x = 10/2^{4}$ ({\color{blue} \solidrule}), $10/2^6$ ({\color{red} \solidrule}), $10/2^8$ ({\color{green!60!black} \solidrule}) and $10/2^{10}$ ({\color{black} \solidrule}).}
\label{fig:FDa6lim}
\end{figure}


There is still the possibility that these solutions are caused by some numerical phenomena such as these methods not properly handling contact discontinuities, more research into this topic should be undertaken. However, the agreement of all the discussed methods of sufficiently high order indicates that these results are representative of actual solutions of the smoothed dam break problem with low $\beta$ for the Serre equations. Lastly we replicated this scenario with $\mathcal{E}$ using a similar order of magnitude for $\Delta x$ as \citeN{El-etal-2006}. The absence of a bump scenario in their findings is due to a smoothing of the initial conditions which is absent from the paper but was confirmed later by \citeN{El-Hoefer-2016-11}. This concludes the explaination of how our results fit in with the current literature and now the following section of this paper will be concerned with some further numerical investigation into these results. 



%--------------------------------------------------------------------------------
\subsubsection{Long time}\label{subsubsec:LT}
%--------------------------------------------------------------------------------
The first test of these results will be of its evolution through time, thus an experiment was run with the same parameters on a larger domain with $x \in [-900m, 1800m]$ for $t \in [0,300s]$. The results for $\beta = 0.294$ and $\Delta x = 10/2^{9}$ at $t = 300s$ are presented in Figure \ref{fig:FVlonga20}. For this problem these parameters result in the bump scenario as can be seen in Figure \ref{fig:o3a20dxlimcdexp}, however after sufficient time we can see that this bump has decayed back into a flat scenario although there are still small oscillations present in the middle region. 

We also observe that the values $S^+$ and $a^+$ have not been perfectly replicated with the numerical solution giving larger values than the analytic ones derived by \citeN{El-etal-2006} although these results are close. We also note that as above the bore heights for the Serre and SWWE appear to be slightly different.

To observe the evolution of the water profile the numerical solution has been shifted by $u_2  \times t$ in Figure \ref{fig:FVlongcemt500} to give a dam break that is essentially motionless with respect to the contact discontinuity. It can be seen that at $t =30s$ the solution is in the bump scenario but as time progresses the centre region has decayed into the contact discontinuity scenario by $t=100s$ and then into the flat scenario observed at $t=200s$ and $t=300s$. This could be a property of the solution Serre equations after sufficient time or due to the accumulation of numerical diffusion with Figure \ref{fig:FVcomplonga20} demonstrating that over this timespan we are not close to convergence of the numerical results.    

\begin{figure}
\centering
\subfigure[][]{\label{fig:FVlonga20h}\includegraphics[width=7cm]{pics/results/SDB/numsols/300s/h0t300s.pdf}}
\subfigure[][]{\label{fig:FVlonga20hf}\includegraphics[width=7cm]{pics/results/SDB/numsols/300s/h1t300s.pdf}}
\subfigure[][]{\label{fig:FVlonga20hb}\includegraphics[width=7cm]{pics/results/SDB/numsols/300s/h2t300s.pdf}}
\subfigure[][]{\label{fig:FVlonga20hz}\includegraphics[width=7cm]{pics/results/SDB/numsols/300s/h4t300s.pdf}}
\subfigure[][]{\label{fig:FVlonga20hz}\includegraphics[width=7cm]{pics/results/SDB/numsols/300s/h3t300s.pdf}}
\caption{Numerical solution of smooth dam break problem by $\mathcal{V}_3$ at $t=300s$ with $\alpha = 0.1$ for $\Delta x = 10/2^{9}$ ({\color{blue} \solidrule}).}
\label{fig:FVlonga20}
\end{figure}

\begin{figure}
\centering
\subfigure[][]{\label{fig:FVlonga20a}\includegraphics[width=7cm]{pics/results/SDB/numsols/300s/evolve30s.pdf}}
\subfigure[][]{\label{fig:FVlonga20a}\includegraphics[width=7cm]{pics/results/SDB/numsols/300s/evolve100s.pdf}}
\subfigure[][]{\label{fig:FVlonga20a}\includegraphics[width=7cm]{pics/results/SDB/numsols/300s/evolve200s.pdf}}
\subfigure[][]{\label{fig:FVlonga20a}\includegraphics[width=7cm]{pics/results/SDB/numsols/300s/evolve300s.pdf}}
\caption{Numerical solution of the smooth dam break problem by $\mathcal{V}_3$ with $\alpha = 0.1$ and $\Delta x = 10/2^{9}$ at $t=$ $30s$ (a), $100s$ (b), $200s$ (c) , $300s$ (d).}
\label{fig:FVlongcemt500}
\end{figure}


%coarsefinecomparison
\begin{figure}
\centering
\subfigure[][]{\label{fig:FVcomplonga20h}\includegraphics[width=7cm]{pics/results/SDB/numsols/300s/CF0t300.pdf}}
\subfigure[][]{\label{fig:FVcomplonga20hf}\includegraphics[width=7cm]{pics/results/SDB/numsols/300s/CF1t300.pdf}}
\subfigure[][]{\label{fig:FVcomplonga20hb}\includegraphics[width=7cm]{pics/results/SDB/numsols/300s/CF2t300.pdf}}
\subfigure[][]{\label{fig:FVcomplonga20hz}\includegraphics[width=7cm]{pics/results/SDB/numsols/300s/CF3t300.pdf}}
\caption{Numerical solution of smooth dam-break problem at $t=300s$ by $\mathcal{V}_3$ with $\alpha = 0.1$ for $\Delta x = 10/2^{9}$ ({\color{blue} \solidrule}) and $10/2^{8}$ ({\color{red} \solidrule}).}
\label{fig:FVcomplonga20}
\end{figure}

\subsubsection{SWWE comparison}
Since the SWWE have been used as a guide for the mean behaviour of the solution of the Serre equations in the literature \cite{Hank-etal-2010-2034,Dutykh-2014-315} we would like the investigate how useful they are. 
%cd speed
We begin by studying the speed of the contact discontinuity which should travel at the mean bore velocity \cite{Gurevich-Meshcherkin-1984-1277}. Since as stated before there are analytic solutions for these values for the SWWE, the numerical results can be compared to this. To investigate this $h_1$ was varied to allow for different aspect ratios and thus different bore speeds. The results are plotted in Figure \ref{fig:CDspeed} from which it is quite clear that this discontinuity does in fact travel at the bore speed for a range of aspect ratios. 
\begin{figure}
	\centering
	\subfigure[][]{\label{fig:CDspeed1}\includegraphics[width=7cm]{pics/results/SDB/CDspeed/speed.pdf}}
	\caption{$u_2$ ({\color{blue} \solidrule}) and speed of the contact discontinuity ({\color{red} $\circ$})  for numerical solutions of smoothed dam-break problems with different aspect ratios ($h_1 /h_0$) by $\mathcal{V}_3$ where $\alpha = 0.1$ and $\Delta x = 10/2^{9}$ at $t=100s$.}
	\label{fig:CDspeed}
\end{figure}

%uh comparison
To further demonstrate the SWWE solution as a useful guide for mean behaviour we plot $h - h_2$ and $u - u_2$ for the smoothed dam break problem with $\beta = 0.2944$ and $\Delta x = \frac{10}{2^9}$ in Figure \ref{fig:UHSWWcomp30sall} for $t= 30s$ and Figure \ref{fig:UHSWWcomp30sall} for $t= 300s$. From this we can see that over short time spans both $h_2$ and $u_2$ are good approximations to the mean behaviour of the fluid with both plots oscillating around $0$. However after sufficient time we see that the mean velocity and height of the fluid has diverged slightly from the SWW equation values $h_2$ and $u_2$. With $h_2$ being an underestimate and $u_2$ being an overestimate. From Figure \ref{fig:FVlonga20} it can also be seen that $S_2$ underestimates the speed of the bore front.

\begin{figure}
	\centering
	\subfigure[][]{\label{fig:UHcomp1}\includegraphics[width=7cm]{pics/results/SDB/numsols/SWWCOMP/30s/0.pdf}}
	\subfigure[][]{\label{fig:UHcomp2}\includegraphics[width=7cm]{pics/results/SDB/numsols/SWWCOMP/30s/1.pdf}}
	\subfigure[][]{\label{fig:UHcomp3}\includegraphics[width=7cm]{pics/results/SDB/numsols/SWWCOMP/30s/2.pdf}}
	\subfigure[][]{\label{fig:UHcomp4}\includegraphics[width=7cm]{pics/results/SDB/numsols/SWWCOMP/30s/3.pdf}}
	\caption{$h - h_2$ ({\color{blue} \solidrule}) and $u - u_2$ ({\color{red} \solidrule}) for numerical solution of the smooth dam-break by $\mathcal{V}_3$ with $\alpha = 0.1$ and $\Delta x = 10/2^{9}$ at $t=30s$ as in Figure \ref{fig:o3a20dxlimcdexp}.}
	\label{fig:UHSWWcomp30sall}
\end{figure}

\begin{figure}
	\centering
	\subfigure[][]{\label{fig:UH300comp1}\includegraphics[width=7cm]{pics/results/SDB/numsols/SWWCOMP/300s/0.pdf}}
	\subfigure[][]{\label{fig:UH300comp2}\includegraphics[width=7cm]{pics/results/SDB/numsols/SWWCOMP/300s/1.pdf}}
	\subfigure[][]{\label{fig:UH300comp3}\includegraphics[width=7cm]{pics/results/SDB/numsols/SWWCOMP/300s/2.pdf}}
	\subfigure[][]{\label{fig:UH300comp4}\includegraphics[width=7cm]{pics/results/SDB/numsols/SWWCOMP/300s/3.pdf}}
	\caption{$h - h_2$ ({\color{blue} \solidrule}) and $u - u_2$ ({\color{red} \solidrule}) for numerical solution of the smooth dam-break by  $\mathcal{V}_3$ with $\alpha = 0.1$ and $\Delta x = 10/2^{9}$ at $t=300s$ as in Figure \ref{fig:FVlonga20}.}
	\label{fig:UHSWWcomp300sall}
\end{figure}

From Figure \ref{fig:UHcomp2} and Figure \ref{fig:UHcomp4} it can also be seen that to the left of the contact discontinuity $u$ and $h$ are antiphase. While Figure \ref{fig:UHcomp3} and Figure \ref{fig:UHcomp4} demonstrate that to the right of the contact discontinuity $u$ and $h$ are in phase. Thus the contact discontinuity marks the transition between these two states, for comaprison in both \ref{fig:UHcomp4} and \ref{fig:UH300comp4} I have marked the point $x_2 = tu_2$ which is the point the contact discontinuity would be at if it travelled at $u_2$. From \ref{fig:UH300comp4} it is clear that at $x_2$ $h$ and $u$ are in phase and so $x_2$ is a slight overestimate of the location of the contact discontinuity as $u_2$ is for the speed of the contact discontinuity and $u$.

%%dispersion relation
Because $h$ and $u$ are antiphase to the left of the contact discontinuity they appear to travel leftwards relative to the contact discontinuity while those on the right are in phase and therefore appear to be travelling rightwards relative to the contact discontinuity. Thus these oscillations appear to be forming at the contact discontinuity and then travelling away from it. The phase velocity of the linearised Serre equations is 
\[\upsilon_p = u \pm \sqrt{gh} \sqrt{\frac{3}{h^2 k^2 + 3}} \; \]
where $k$ is the wavenumber. The phase velocity has the following behaviour, as $k \rightarrow \infty$ then $\upsilon_p \rightarrow u$ and as $k \rightarrow 0$ then $\upsilon_p \rightarrow u \pm \sqrt{gh}$ . Since we observe $u$ and $h$ as being antiphase to the left of the contact discontinuity this means we are in the negative branch of the phase velocity $u - \sqrt{gh} \sqrt{\frac{3}{h^2 k^2 + 3}}$ while the in phase right corresponds to the positive branch  $u + \sqrt{gh} \sqrt{\frac{3}{h^2 k^2 + 3}}$. Thus the contact discontinuity corresponds to very high wavenumber oscillations, which explains why it is very sensitive to both smoothing of the initial conditions and numerical diffusion.   

%%% a plus 
\subsubsection{Whitham modulation comparison}
The expressions for the lead soliton amplitude $a^+$ and speed $S^+$ obtained by \citeN{El-etal-2006} are asymptotic results and so we are interested in how our numerical results behave over time. Thus for the dam break problem in the long time subsubsection the lead soliton amplitude was captured over time and plotted in Figure \ref{fig:FVlonga20aplus}. From it we can see that we do appear to approach some value but that value is higher than the analytic value $a^+$. We find that using larger $\beta$ and larger $\Delta x$ allows our numerical solution to better approach $a^+$ and so by better approximating the true solution we actually converge away from $a^+$ not towards it in this timescale for this aspect ratio. This is not inconsistent with the results of \cite{El-etal-2006} as their scale comparing $a^+$ to numerical solutions is too large to see such a small difference. From Figure \ref{fig:FVlonga20} it can be seen that while $S^+$ does not precisely predict the bore speed it is closer than the analytic result of the SWWE $S_2$.
\begin{figure}
	\centering
	\subfigure[][]{\label{fig:FVlonga20a}\includegraphics[width=7cm]{pics/results/SDB/numsols/300s/a.pdf}}
	\caption{Leading wave height plotted over time for the numerical solution of the smooth dam-break problem by $\mathcal{V}_3$ with $\alpha = 0.1$ for $\Delta x = 10/2^{9}$ ({\color{blue} \solidrule}) as in Figure \ref{fig:FVlonga20}.}
	\label{fig:FVlonga20aplus}
\end{figure}

%energy break down
\subsubsection{Energy Breakdown}
The Hamiltonian \eqref{eqn:Hamildef} has $3$ terms representing in order, horizontal kinetic energy $hu^2$, vertical kinetic energy $\frac{h^3}{3}\frac{\partial u}{\partial x}$ and gravitational potential energy $gh^2$. It might be expected that the these rapid oscillations of the undular bore such as in Figure \ref{fig:FVlonga20} would result in significant vertical energies. However, Figure \ref{fig:PHTa12all} demonstrates that this is not the case, as the total vertical kinetic energy in the system is insignificant relative to the other energies. This plot also demonstrates that even with dispersive terms and large oscillations the drivers of change in the dam break problem are the transfer of gravitational potential energy into horizontal kinetic energy which occurs very slowly.

\begin{figure}
	\centering
	\subfigure[][]{\label{fig:PHTa12}\includegraphics[width=7cm]{pics/results/SDB/numsols/300s/HFT-figure0.pdf}}
	\subfigure[][]{\label{fig:PHTa12z}\includegraphics[width=7cm]{pics/results/SDB/numsols/300s/TT-figure0.pdf}}
	\caption{Proportion of $\mathcal{H}$ made up by horizontal kinetic energy ({\color{blue} \solidrule}), vertical kinetic energy ({\color{red} \solidrule}) and gravitational potential energy ({\color{green!80!black} \solidrule}) for $\mathcal{V}_3$ solution of the smooth dam-break with $\alpha = 0.1$ and $\Delta x = 10/2^{9}$ over time as in Figure \ref{fig:FVlonga20}.}
	\label{fig:PHTa12all}
\end{figure}
%--------------------------------------------------------------------------------
\section{Conclusions}
\label{section:Conclusions}
%--------------------------------------------------------------------------------
Utilizing two finite difference methods of second order and three finite difference-volume methods of various orders an investigation into the smooth dambreak problem with various steepnesses was performed. Four different behaviours were uncovered with the general trend being that an increase in steepness increases the size and number of oscillations in the solution. This study explains all current differences in the literature involving the solution of the Serre equations applied to the smoothed dam break problem and also uncovers a new result. We find that while the analytic solution of the dam break problem is a good guide to the mean behaviour of the Serre equations the speed and height of the bores do not match up precisely. While the Whitham modulation results for the Serre equations give better predictotions than the SWW analytic solution it was found that they while close do not line up with our numerical results precisely. It was also demonstrated that the contact discontinuity corresponds to high wave numbers and that vertical kinetic energy is negligible for the dam break problem. 

%more research

%--------------------------------------------------------------------------------
\bibliography{Serre_ASCE}
%--------------------------------------------------------------------------------

\end{document}
