\chapter{Publications}
\label{app:Pub}
This appendix lists the publications my research contributed to in chronological order. This list expands the earlier list in Chapter \ref{chp:Introduction} by providing the abstracts for the paper in place of the summary. My contribution to the paper and the relevance of the paper to this thesis are also provided.

\vspace*{\baselineskip}

\begin{center}
	\vbox{
		\textbf{
			\large A Solution of the Conservation Law Form of the Serre Equations}
		
		\vspace*{\baselineskip}
		
		\textit{Australia and New Zealand Industrial and Applied Mathematics Journal (2016)}
		
		{C. Zoppou, S.G. Roberts and J. Pitt}
		\vspace*{0.5\baselineskip}
	}
\end{center}
\noindent\textbf{Abstract:}

The nonlinear and weakly dispersive Serre equations contain higher-order dispersive terms. These include mixed spatial and temporal derivative flux terms which are difficult to handle numerically. These terms can be replaced by an alternative combination of equivalent temporal and spatial terms, so that the Serre equations can be written in conservation law form. The water depth and new conserved quantities are evolved using a second-order finite-volume scheme. The remaining primitive variable, the depth-averaged horizontal velocity, is obtained by solving a second-order elliptic equation using simple finite differences. Using an analytical solution and simulating the dam-break problem, the proposed scheme is shown to be accurate, simple to implement and stable for a range of problems, including flows with steep gradients. It is only slightly more computationally expensive than solving the shallow water wave equations. 
\vspace*{\baselineskip}

\noindent\textbf{My Contribution:}

I produced an independent reproduction of the method of my coauthors, verifying their results. This method was used to compare the computational cost of solving the Serre equations and the Shallow Water Wave Equations (SWWE).
\vspace*{\baselineskip}

	\noindent\textbf{Relevance to Thesis:}
	
	The method described in this paper was the foundation of the second-order finite difference volume method whose results are reported in this thesis. The main extensions of this method during my research were the inclusion of varying bathymetry \cite{Zoppou-etal-2017} and dry beds. The results of a linear analysis of the second-order finite difference volume method is provided in Chapter \ref{chp:AnalNumMethod} and Appendix \ref{app:LinAnal}. Since the linear analysis is performed for a completely submerged horizontal bed, these results apply to the method described in this paper. Finally, the second-order finite difference volume method was validated against analytic and forced solutions and experimental results in Chapter \ref{chp:NumMethodComp} and Chapter \ref{chp:ExpMethodComp}, respectively. 
	
 \newpage
\begin{center}
	\vbox{
		\textbf{
			\large Numerical Solution of the Fully Non-Linear Weakly Dispersive
			Serre Equations for Steep Gradient Flows}
		
		\vspace*{\baselineskip}
		
		\textit{Applied Mathematical Modelling (2017)}
		
		{C. Zoppou, J. Pitt and S.G. Roberts}
		\vspace*{0.5\baselineskip}
	}
\end{center}
\noindent\textbf{Abstract:}

We demonstrate a numerical approach for solving the one-dimensional non-linear weakly dispersive Serre equations. By introducing a new conserved quantity the Serre equations can be written in conservation law form, where the velocity is recovered from the conserved quantities at each time step by solving an auxiliary elliptic equation. Numerical techniques for solving equations in conservative law form can then be applied to solve the	Serre equations. We demonstrate how this is achieved. The system of conservation equations are solved using the finite volume method and the associated elliptic equation for the velocity is solved using a finite difference method. This robust approach allows us to accurately solve problems with steep gradients in the flow, such as those generated by discontinuities in the initial conditions.

The method is shown to be accurate, simple to implement and stable for a range of problems including flows with steep gradients and variable bathymetry.
\vspace*{\baselineskip}

	\noindent\textbf{My Contribution:}
	
	This paper was based on research produced by me in collaboration with my coauthors which implemented the methods, performed the dispersion analysis and produced the numerical solutions. These results were then written up by my coauthors.
\vspace*{\baselineskip}

	\noindent\textbf{Relevance to Thesis:}
	
	The results of the first-, second- and third-order finite difference volume methods described in this paper are the methods whose results are reported in this thesis. The results of a linear analysis for these methods can be found in Chapter \ref{chp:AnalNumMethod} and Appendix \ref{app:LinAnal}. A validation of some of these methods against analytic and forced solutions and experimental results can be found in Chapter \ref{chp:NumMethodComp} and Chapter \ref{chp:ExpMethodComp}, respectively. 
	
	The linear analysis of the dispersion properties of the methods in this paper was extended in this thesis by allowing for a non-zero background mean flow velocity. The analytic solution validation of this paper was reproduced in this thesis and extended by studying the convergence and conservation properties of more quantities. The experimental results of the second-order finite difference volume method in this paper for the negative rectangular wave experiment \cite{Hammack-Segur-1978-337} and periodic waves over a submerged bar experiment \cite{Beji-Battjes-1994-1} were reproduced in this thesis. 

\newpage



\begin{center}
	\vbox{
		\textbf{
			\large Importance of Dispersion for Shoaling Waves}
		
		\vspace*{\baselineskip}
		
		\textit{22nd International Congress on Modelling and Simulation (2017)}
		
		{J. Pitt, C. Zoppou and S.G. Roberts}
		\vspace*{0.5\baselineskip}
	}
\end{center}
\textbf{Abstract:}

A tsunami has four main stages of its evolution; in the first stage the tsunami is generated, most commonly by seismic activity near subduction zones. The second stage is the tsunamis propagation through the ocean far from the coast, where variation in bathymetry is slight and gradual. The third stage is the shoaling and interaction of the tsunami with bathymetry as it approaches the coastline. Finally the tsunami reaches and inundates the shore. For our purposes the hydrodynamic models we are interested in deal with the final three stages of the evolution of a tsunami.

The propagation of a tsunami with wavelength $\lambda$ through water that is $H$ deep is well understood when $\lambda / H \le 1/ 20$, which we call shallow water as noted by Sorensen (2006). The wavelengths for tsunamis range from a few to hundreds of kilometres, while the maximum water depth is $11km$ at the Marianas trench, so that most tsunamis occur in shallow water. This stage of tsunami behaviour is adequately modelled using the shallow water wave equations. Current research into tsunamis focuses around more complex approximations to the Euler equations for the third and fourth stages. In this paper we focused on the Serre equations as they are considered a very good model for fluid behaviour up to the shoreline, and they reduce to the shallow water wave equations for large wavelengths. 

Although more complicated, the Serre equations provide a better description of the fluid behaviour than the shallow water wave equations and are therefore more computationally expensive to solve numerically. In particular for the methods of this work, we find that the Serre equations have a run-time $50\%$ longer than our equivalent finite volume method for the shallow water wave equations in the one dimensional case. To simulate tsunamis as efficiently as possible it is important to know when using the more complicated Serre equations leads to more accurate predictions of the evolution of a tsunami than the shallow water wave equations. To investigate this we have numerically simulated a laboratory experiment of periodic waves propagating over a submerged bar, and the propagation of a small amplitude wave up a gradual linear slope using both the Serre and the shallow water wave equations.

The results of these simulations demonstrated that the Serre and shallow water wave equations produce similar results for shoaling waves when the wavelength is large compared to the water depth. This is not surprising as this is the regime under which the shallow water wave equations are derived. However, outside this regime the shallow water wave equations are a poor model for wave shoaling and propagation, poorly approximating the shape and maximum height of waves. Furthermore we demonstrate that for steep waves generated by shoaling, the shallow water wave equations can underestimate the arrival time and amplitude of an incoming wave. These results suggest that for a tsunami it is sufficient to use the shallow water wave equations in stages two and some of stage three, even for large changes in bathymetry. Although dispersive equations such as the Serre equations are required to accurately capture fluid behaviour in stages three and four nearer to the coastline, particularly when wavelengths are short or waves are steep. Since the Serre equations represent only a moderate increase in run-times this suggests that our inundation models should be based on them.

\vspace*{\baselineskip}
	\noindent\textbf{My Contribution:}
	
 This paper was primarily produced by me in collaboration with my coauthors, based on research that I primarily undertook.
\vspace*{\baselineskip}


	\noindent\textbf{Relevance to Thesis:}
	
	The results of the second-order finite difference volume method for the experiments of \citet{Beji-Battjes-1994-1} studying periodic waves over a submerged bar are reproduced in Chapter \ref{chp:ExpMethodComp}.

\newpage

 \begin{center}
 	\vbox{
 		\textbf{
 			\large Behaviour of the Serre Equations in the Presence of Steep Gradients Revisited}
 		
 		\vspace*{\baselineskip}
 		
 		\textit{Wave Motion (2018)}
 		
 		{J.P.A. Pitt, C. Zoppou and S.G. Roberts}
 		\vspace*{0.5\baselineskip}
 	}
 \end{center}
\textbf{Abstract:}

We use numerical methods to study the short term behaviour of the Serre equations in
the presence of steep gradients because there are no known analytical solutions for these
problems. In keeping with the literature we study a class of initial condition problems that
are a smooth approximation to the initial conditions of the dam-break problem. This class
of initial condition problems allow us to observe the behaviour of the Serre equations
with varying steepness of the initial conditions. The numerical solutions of the Serre
equations are justified by demonstrating that as the resolution increases they converge
to a solution with little error in conservation of mass, momentum and energy independent
of the numerical method. We observe and justify four different structures of the converged
numerical solutions depending on the steepness of the initial conditions. Two of these
structures were observed in the literature, with the other two not being commonly found
in the literature. The numerical solutions are then used to assess how well the analytical
solution of the shallow water wave equations captures the mean behaviour of the solution
of the Serre equations for the dam-break problem in the short term. Lastly the numerical
solutions are compared to asymptotic results in the literature to approximate the depth
and location of the front of an undular bore.
\vspace*{\baselineskip}

	\noindent\textbf{My Contribution:}
	
 This paper was primarily produced by me in collaboration with my coauthors, based on research that I primarily undertook.
\vspace*{\baselineskip}


	\noindent\textbf{Relevance to Thesis:}
	
	The results of this paper are summarised in Chapter \ref{chp:Serreeqns}. This paper demonstrates the utility of using a finite volume based method to solve the Serre equations in the presence of steep gradients. Hence, the further development of these methods in this thesis.	