\chapter{Finite Element Integrals}
\label{app:FEMIntegrals}
We now provide some of the details left out of Chapter \ref{chp:HFVMMethod}; the definitions for the basis functions in the $\xi$ space and the other finite element matrices. 

%also put the basis funcitons
\section{Basis Functions}
$\psi$
\begin{align}
\psi^+_{j-1/2} &= \left\lbrace \begin{array}{c c}
\frac{1}{2}\left(1 - \xi\right) & -1 \le \xi \le 1\\
0 & \text{otherwise}
\end{array} \right. \\
\psi^-_{j+1/2} &= \left\lbrace \begin{array}{c c}
\frac{1}{2}\left(1 + \xi\right) & -1 \le \xi \le 1\\
0 & \text{otherwise}
\end{array} \right. 
\end{align}

$\phi$
\begin{align}
\phi_{j-1/2} &= \left\lbrace \begin{array}{c c}
2 \left(\xi + \frac{3}{2}\right)\left(\xi + 2\right) & -2 \le \xi \le -1\\
\frac{1}{2}\xi(\xi - 1) & -1 \le \xi \le 1\\
0 & \text{otherwise}\end{array} \right. \\
\phi_{j} &= \left\lbrace \begin{array}{c c}
-\left(\xi - 1\right)\left(\xi + 1\right) & -1 \le \xi \le 1\\
0 & \text{otherwise}
\end{array} \right. \\ 
\phi_{j+1/2} &= \left\lbrace \begin{array}{c c}
\frac{1}{2}\xi(\xi + 1) & -1 \le \xi \le 1\\
2\left(\xi - 2\right)\left(\xi - \frac{3}{2} \right) & 1 \le \xi \le 2\\
0 & \text{otherwise}
\end{array} \right. \\ 
\end{align}

$\gamma$
\begin{align}
\gamma_{j-1/2} &= \left\lbrace \begin{array}{c c}
\frac{9}{2}\left(\xi + \frac{4}{3}\right)\left(\xi + \frac{5}{3}\right)\left(\xi + 2\right) & -2 \le \xi \le -1\\
\frac{9}{16}\left(\xi - 1\right)\left(\xi - \frac{1}{3}\right)\left(\xi  + \frac{1}{3}\right) & -1 \le \xi \le 1\\
0 & \text{otherwise} 
\end{array} \right. \\
\gamma_{j-1/6} &= \left\lbrace \begin{array}{c c}
\frac{27}{16}\left(\xi - 1\right)\left(\xi - \frac{1}{3}\right)\left(\xi + 1\right) & -1 \le \xi \le 1\\
0 & \text{otherwise} 
\end{array} \right. \\
\gamma_{j+1/6} &= \left\lbrace \begin{array}{c c}
-\frac{27}{16}\left(\xi - 1\right)\left(\xi + \frac{1}{3}\right)\left(\xi + 1\right) & -1 \le \xi \le 1\\
0 & \text{otherwise} 
\end{array} \right. \\
\gamma_{j-1/2} &= \left\lbrace \begin{array}{c c}
\frac{9}{16}\left(\xi + 1\right)\left(\xi - \frac{1}{3}\right)\left(\xi  + \frac{1}{3}\right) & -1 \le \xi \le 1\\
-\frac{9}{2}\left(\xi - \frac{4}{3}\right)\left(\xi - \frac{5}{3}\right)\left(\xi - 2\right) & 1 \le \xi \le 2\\
0 & \text{otherwise} 
\end{array} \right.
\end{align}


