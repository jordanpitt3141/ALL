
\chapter{Introduction}
\label{chp:Introduction}

%Where to find all the numerical methods used and what are they named. 


%  Motivation

%Tsunamis
%Water modelling
%Boussinesq, shallow water models
%Serre equations


Understanding the behaviour of waves of the free surface of water is vital to our society. A significant portion of the world's people and critical infrastructure is located near the coast, with shipping being the dominant method of trade in our globalised world. While the ocean provides many opportunities it also poses significant risks, in the form of tsunamis and storm surges. Furthermore, the dynamics of ocean waves significantly impacts our understanding of other physical phenomena; such as the breakup of sea-ice and the erosion of beaches. 

The physics of water is well understood being an application of Newtons second law. Thus, the partial differential equations that describe water; the Euler equations were written very early on \cite{Euler-1755-274}. These were then extended to allow for viscosity to get the full Navier-Stokes equations []. Numerical methods [] have been developed to solve these full water equations; however due to the complexity of these partial differential equations such methods can only accurately resolve fluid behaviour on a small scale and not the scale required to model a tsunami over an ocean basin. 

For this reason the central focus of ocean modelling has been on simplified water wave theories that approximate the behaviour of the free surface of water under the Euler or Navier-Stokes equations. The most popular class of these approximate water wave theories are the shallow water wave theories where the characteristic water depth $h_0$ is far smaller than the characteristic wave length $\lambda_0$. So that $\sigma = h_0 / \lambda_0 \le 1$. If we neglect all terms of order $\mathcal{O}\left(\sigma ^2\right)$ then the full Euler equations reduce to the Shallow Water Wave Equations (SWWE) which describe fully nonlinear non-dispersive waves. Retaining higher powers of $\sigma$ leads to a class of equations known as `Boussinesq-type' equations. Boussinesq-type equations are then classified by the powers of $\sigma$ they retain and their nonlinearity; which is based on their assumption on the size of $\epsilon= a_0 / h_0$ which measures the characteristic amplitude of the waves $a_0$ against $h_0$. These wave models form a spectrum with the SWWE being the simplest and most restrictive model and the `Boussinesq-type' models retaining the highest powers of $\sigma$ and largest $\epsilon$ being the most complex and least restrictive. The Serre equations are one particular Boussniesq-type equation that retains all terms of order $\mathcal{O}\left(\sigma ^4\right)$ and makes no assumption of the size of $\epsilon$ and is thus the best model for water waves for the $\mathcal{O}\left(\sigma ^4\right)$ class of Boussinesq-type equations. 

There has previously been a significant amount of research into developing large scale, efficient and robust computational methods for the SWWE []. However, due to the neglect of all terms of order $\mathcal{O}\left(\sigma ^2\right)$ in the Euler equations the SWWE do not capture all water wave behaviour in particular dispersion. Recent research [] has highlighted the need for dispersive wave models for the evolution of tsunamis. This has lead to the development of a variety of numerical methods for Boussinesq-type equations which incorporate dispersion []. For the purposes of ocean wave modelling the Serre equations are the best placed []; retaining high-order $\sigma$ terms and allowing for any size of $\epsilon$. 



\section{Objectives of the Thesis}
%Method that:
%Extends well to 2D and triangular meshes and can calculate all terms in flux locally after reconstruction
%
%Can handle steep gradients
%
%Can handle dry beds
%Perform a full linear analysis of the methods

The overarching goal then would be the development of large-scale, efficient and robust computational models for the Serre equations.

The primary motivation of this thesis was the development of a numerical method for the one-dimensional Serre equations that is robust to steep gradients in the free surface and dry beds and can be readily extended to the two dimensional Serre equations using unstructured meshes. 

This was achieved through the further development of previously described methods in the literature; particularly those of \citet{Zoppou-2014}. This further development allowed them to adequately handle dry beds and replaced their finite difference method with a finite element method; making the method suitable for unstructured meshes. 

This method was then assessed with a linear analysis, a validation against analytic and forced solutions and finally with a validation against experimental results. At all stages of this assessment the method is compared to at least one other; demonstrating its strengths and weaknesses. Overall, the method is found to be superior to the others and possesses all the desirable properties constituting the main goal of the Thesis.  

\section{Original Contribution of the Thesis}
%The FEVM method, a third order FDVM
%Steep gradient review of literature and new result
%full linear analysis
%Serre model with dry beds validated against forced solutions and experiments

My research made the following original contributions to the field:
\begin{itemize}
	\item Observation and justification of a new structure of the solution of the Serre equations to the dam-break problem.
	\item Development and description of the well balanced second-order finite element volume method that can handle dry beds.
	\item Extension of the second-order finite difference volume method to allow for dry beds.
	\item Implementation of the third-order finite difference volume method.
	\item A linear analysis of convergence for all developed hybrid finite volume methods and some finite difference method.
	\item An analysis of the dispersion properties of all named methods extending previous work by allowing for non-zero mean fluid velocity and analysing the total dispersion error.
	\item Validation of these methods using forced solutions where all terms of the Serre equations are present for wet and dry beds.
	\item Comparison of good approximations to the solution of the Serre equations and experimental results in the presence of dry beds and with wave breaking. 
	\item The title and abstract of three papers that have been either written by me or contain a significant amount of my work throughout my research can be found in Appendix \ref{app:Pub}
\end{itemize}



\section{Organisation of the Thesis}
The thesis proceeds by presenting the one-dimensional Serre equations in conservation law form with a source term, describing their dispersion and conservation properties and then summarising the main results of my investigation into the behaviour of the Serre equations in the presence of steep gradients in the free surface \cite{Pitt-2018-61}.

In Chapter \ref{chp:HFVMMethod} the developed method; which combines a finite element and a finite volume method is described in detail. In this thesis the results of other numerical methods are also provided; descriptions of these methods can be found in the literature \cite{Zoppou-etal-2017,Pitt-2018-61}. However, since the developed method is the preferred method only its description is provided in this thesis. 

Chapter \ref{chp:AnalNumMethod} provides an linear analysis of the convergence and dispersion properties of the developed finite element and volume method in detail. The analysis begins with the linearised Serre equations over a horizontal bed and then derives the evolution matrix; from which the convergence and dispersion properties of the methods can be studied. The results of the linear analysis are also provided for all the methods used by \citet{Pitt-2018-61}.

The convergence and conservation properties of the numerical methods of \citet{Pitt-2018-61} are then assessed in Chapter \ref{chp:NumMethodComp} using analytic and forced solutions of the Serre equations. While Chapter \ref{chp:ExpMethodComp} validates the numerical method against experimental results.

Finally, Chapter \ref{chp:Conclusion} summarises the major contributions and findings of the Thesis and presents a summary of the future work for further development of these methods. 


