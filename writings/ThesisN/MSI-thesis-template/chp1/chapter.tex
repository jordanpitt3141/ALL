\chapter{Introduction}
\label{chp:Introduction}

A significant portion of the world's people and critical infrastructure is located near the coast. While the ocean provides many opportunities it also poses significant hazards from tsunamis and storm surges. Furthermore, the dynamics of ocean waves significantly impacts our understanding of other physical phenomena; such as the breakup of sea-ice and the erosion of beaches. Therefore, accurate modelling of ocean waves is important to our society. 

The physics of water can be described using Newton's second law. From which the partial differential equations initially presented by Euler in 1757 \cite{Euler-1755-274} can be derived. The Euler equations were then extended to include viscosity, producing the full Navier-Stokes equations \cite{navier1823,stokes1845gg}. Numerical methods \cite{Chorin-1967-928,Taylor-Hood-1973-73,Bassi-1997-267} have been developed to solve the Euler equations; however due to their complexity these methods can only accurately resolve fluid behaviour over small scales and not the scales required to model tsunamis along a coastline. 

For this reason the central focus of water wave modelling has been simplified water wave theories that approximate the behaviour of the free surface of water governed by the Euler equations. The most popular class of these approximate water wave theories are the shallow water wave theories where the characteristic water depth $h_0$ is far smaller than the characteristic wave length $\lambda_0$, so that $\sigma = h_0 / \lambda_0 \ll 1$. Neglecting all terms of order $\mathcal{O}\left(\sigma ^2\right)$ the full Euler equations reduce to the Shallow Water Wave Equations (SWWE) \cite{Bonneton-Lannes-2009-16601} which describe fully non-linear non-dispersive waves. Retaining higher powers of $\sigma$ leads to a class of equations known as `Boussinesq-type' equations. Boussinesq-type equations are then classified by the powers of $\sigma$ they retain and their retained non-linearity; which is based on the size of $\epsilon= a_0 / h_0$ which compares the characteristic amplitude of the waves $a_0$ ro the water depth $h_0$. These wave models form a spectrum with the SWWE being the simplest and most restrictive model and the Boussinesq-type models retaining the highest powers of $\sigma$ and largest $\epsilon$ being the most complex and least restrictive. The Serre equations are one particular Boussinesq-type equation that retains all terms of order $\mathcal{O}\left(\sigma ^4\right)$ and makes no assumption on the size of $\epsilon$ \cite{Bonneton-Lannes-2009-16601}. This allows the Serre equations to better model water waves in intermediate depths closer to the coastline where $\sigma^2$ terms can be significant but $\sigma^4$ are not. Since the Serre equations allow arbitrary wave height they are the most appropriate model for water waves for the $\mathcal{O}\left(\sigma ^4\right)$ class of Boussinesq-type equations. 

There has previously been a significant amount of research into developing large scale, efficient and robust computational methods for the SWWE \cite{ClawPack,Comcot,ANUGA}. The SWWE neglect all terms of order $\mathcal{O}\left(\sigma ^2\right)$ in the Euler equations and so do not capture all water wave behaviour; in particular dispersion. Recent research \cite{Grue-etal-2008-113,Kirby-etal-2013-39} has highlighted the need for dispersive wave models for the evolution of tsunamis. For the purposes of ocean wave modelling the Serre equations are the best placed \cite{Bonneton-Lannes-2009-16601}; retaining high-order $\sigma$ terms and allowing arbitrary wave amplitude. Hence the overarching goal of our research is the development of large-scale, efficient and robust computational models for the Serre equations for the purposes of wave modelling.

\section{Objectives of the Thesis}
In view of the overarching goal, the primary motivation of this thesis was the development of a numerical method for solving the one-dimensional Serre equations that is robust to steep gradients in the free surface, can handle dry beds and can be readily extended to the two-dimensional Serre equations using unstructured meshes. 

Some of these goal were achieved through the development of the Finite Element Volume Method (FEVM). The FEVM is an improvement of the Finite Difference Volume Methods described by \citet{Zoppou-2014}. The FEVM can adequately handle dry beds and uses finite element method instead of a finite difference method, making it more suitable for unstructured meshes. 

The FEVM was assessed with a linear analysis, a validation against analytic and forced solutions and experimental results. At all stages of this assessment the method is compared to at least one other method to demonstrate its strengths and weaknesses. Overall, the method is found to be superior to others and satisfies all the objectives of the thesis.

\section{Original Contribution of the Thesis}
My research made the following original contributions to the field:
\begin{itemize}
	\item Implementation of the third-order finite difference volume method.
	\item Observation and justification of a new structure in the solution of the Serre equations in the presence of steep gradients in the free surface.
	\item Extension of the second-order finite difference volume method to allow for dry beds.
	\item Development and description of the well-balanced second-order finite element volume method that can handle dry beds.
	\item A linear analysis of convergence for all developed finite volume based methods as well as some finite difference methods.
	\item The analysis of the dispersion properties extended previous work by allowing for non-zero mean fluid velocity and considers the total dispersion error of the method.
	\item Validation of the method using forced solutions where all terms of the Serre equations are present for both wet and dry beds.
	\item Comparison of numerical solutions of the Serre equations with experimental results in the presence of dry beds and with wave breaking. 
\end{itemize}
My research contributed to the following publications in chronological order.
\begin{center}
	\textbf{
		\Large A Solution of the Conservation Law Form of the Serre Equations}
	
	\vspace*{\baselineskip}
	
	\textit{Australia and New Zealand Industrial and Applied Mathematics Journal (2016)}
	
	{C. Zoppou, S.G. Roberts and J. Pitt}
	\vspace*{0.5\baselineskip}
\end{center}
\textbf{My Contribution:}
I validated the results of my coauthors with my own implementation of the methods. These methods were created to be consistent with a SWWE solver allowing the greater computational cost of solving the Serre equations over the SWWE to be calculated.  
\vspace*{\baselineskip}
\begin{center}
	\textbf{
		\Large Numerical Solution of the Fully Non-Linear Weakly Dispersive
		Serre Equations for Steep Gradient Flows}
	
	\vspace*{\baselineskip}
	
	\textit{Applied Mathematical Modelling (2017)}
	
	{C. Zoppou, J. Pitt and S.G. Roberts}
	\vspace*{0.5\baselineskip}
\end{center}
\textbf{My Contribution:}

The methods, linear dispersion analysis and numerical solutions were all produced by me; these results were then written up into this paper by my coauthors. 

\vspace*{\baselineskip}
\begin{center}
	\textbf{
		\Large Importance of Dispersion for Shoaling Waves}
	
	\vspace*{\baselineskip}
	
	\textit{22nd International Congress on Modelling and Simulation (2017)}
	
	{J. Pitt, C. Zoppou and S.G. Roberts}
	\vspace*{0.5\baselineskip}
\end{center}
 \textbf{My Contribution:}
 
 This paper was produced by me with the support of my coauthors based on my own work.
 
 \vspace*{\baselineskip}
 
 \begin{center}
 	\textbf{
 		\Large Behaviour of the Serre Equations in the Presence of Steep Gradients Revisited}
 	
 	\vspace*{\baselineskip}
 	
 	\textit{Wave Motion (2018)}
 	
 	{J.P.A. Pitt, C. Zoppou and S.G. Roberts}
 	\vspace*{0.5\baselineskip}
 \end{center}
 \textbf{My Contribution:}
 
 This paper was produced by me with the support of my coauthors based on my own work.

\section{Organisation of the Thesis}
Chapter \ref{chp:Serreeqns} proceeds by presenting the one-dimensional Serre equations in conservation law form with a source term. It's dispersion and conservation properties are also presented. Followed by a summary of the main results of my investigation into the behaviour of the Serre equations in the presence of steep gradients in the free surface \cite{Pitt-2018-61}.

This is followed by Chapter \ref{chp:HFVMMethod} which describes in detail the FEVM. In this thesis the results of other numerical methods are also provided. Descriptions of these methods can be found in the literature \cite{Zoppou-etal-2017,Pitt-2018-61}. 

Chapter \ref{chp:AnalNumMethod} provides a linear analysis of the convergence and dispersion properties of the developed finite element volume method in detail. The analysis begins with the linearised Serre equations over a horizontal bed and then derives the evolution matrix; from which the convergence and dispersion properties of the methods can be studied. The results of the linear analysis are also provided for all the methods used by \citet{Pitt-2018-61} for comparison.

The convergence and conservation properties of the numerical methods of \citet{Pitt-2018-61} are then assessed in Chapter \ref{chp:NumMethodComp} using analytic and forced solutions of the Serre equations. While Chapter in \ref{chp:ExpMethodComp} the numerical methods are validated against experimental results.

Finally, Chapter \ref{chp:Conclusion} summarises the major contributions and findings of the thesis and provides ideas for future work. 