\chapter{Introduction}
\label{chp:Introduction}

A significant portion of the world's people and critical infrastructure is located near the coast. While the ocean provides many opportunities it also susceptible to natural hazards, particularly extreme ocean waves such as tsunamis and storm surges. Furthermore, the dynamics of ocean waves drives other physical phenomena; such as the break-up of sea-ice and the erosion of beaches. Therefore, accurate modelling of ocean waves is important to society. 

The physics of water can be described using Newton's second law and were initially presented by Euler in 1757 \cite{Euler-1755-274}. The Euler equations were then extended to include viscosity, producing the full Navier-Stokes equations \cite{navier1823,stokes1845gg}. Numerical methods \cite{Chorin-1967-928,Taylor-Hood-1973-73,Bassi-1997-267} have been developed to solve the Euler equations; however due to their complexity these methods cannot accurately resolve fluid behaviour over the scales required to model tsunamis along a coastline. 

For this reason the central focus of water wave modelling has been simplified water wave theories that approximate the behaviour of the free surface of water governed by the Euler equations. The most popular class of these approximate water wave theories are the shallow water wave theories where the characteristic water depth $h_0$ is far smaller than the characteristic wavelength $\lambda_0$, thus $\sigma = h_0 / \lambda_0 \ll 1$. For tsunamis and storm surges $h_0$ is typically $4km$ far from the coastline and $\lambda_0$ can be $100km$, therefore $\sigma\ll 1$.

Neglecting all terms of order $\mathcal{O}\left(\sigma ^2\right)$ the full Euler equations reduce to the Shallow Water Wave Equations (SWWE) \cite{Bonneton-Lannes-2009-16601} which describe fully non-linear non-dispersive waves. Retaining higher powers of $\sigma$ leads to a class of equations known as `Boussinesq-type' equations. Boussinesq-type equations are then classified by the powers of $\sigma$ they retain and their retained non-linearity. The non-linearity $\epsilon= a_0 / h_0$ compares the characteristic amplitude of the waves $a_0$ to the water depth $h_0$. 

Water wave models form a spectrum with the SWWE being the simplest and most restrictive model and the Boussinesq-type models retaining the highest powers of $\sigma$ and allowing the largest $\epsilon$ values being the most complex and least restrictive. The Serre equations are one particular Boussinesq-type equation that retains all terms of order $\mathcal{O}\left(\sigma ^4\right)$ and makes no assumption on the size of $\epsilon$ \cite{Bonneton-Lannes-2009-16601}. Since the Serre equations allow arbitrary wave height they are the most appropriate model for water waves for the $\mathcal{O}\left(\sigma ^4\right)$ class of Boussinesq-type equations. Furthermore, by retaining $\sigma^2$ terms the Serre equations more accurately model water waves than the SWWE for intermediate water depths where $\sigma^2$ terms can be significant. Intermediate water depths tend to occur as tsunamis and storm surges approach the coastline and interact with the varying bathymetry. Therefore, the Serre equations are well suited to modelling tsunamis and storm surges. 

To model the impact of tsunamis and storm surges on a coastline requires modelling a large area over a long period of time. For this reason, most large scale models of tsunamis and storm surges are performed on supercomputers. To efficiently use supercomputers requires computational methods that are efficient, parallelisable and robust. A method is robust if it can stably and accurately model all appropriate physical scenarios. Robustness ensures that multiple simulations are not required because a method encountered a physical situation that lead to instabilities. This thesis will focus on robustness of the methods in the presence of steep gradients in the water surface and for the inundation of beaches.  

A variety of numerical methods have been developed to solve the SWWE from finite difference methods \cite{Comcot} to finite volume methods \cite{ClawPack,ANUGA} and discontinuous Galerkin methods \cite{Eskilsson}. When producing large scale, parallelised, efficient and robust computational methods the finite volume methods and discontinuous Galerkin methods have distinct advantages over their finite difference method counterparts; they can be readily extended to unstructured meshes and are robust. Unstructured meshes are attractive because they allow for efficient numerical solutions over complex geometries. Furthermore, finite volume methods conserve mass and momentum in their numerical solutions, reproducing the conservative physics at the foundation of the equations. Discontinuous Galerkin methods can also be made conservative under the right conditions, but due to their longer history and simplicity the most popular numerical methods for modelling tsunamis and storm surges are the finite volume methods \cite{ClawPack,ANUGA}.

%even more?

Although great progress has been made in the development of large scale, efficient and robust computational methods for the SWWE \cite{ClawPack,ANUGA} these models suffer from the assumptions made to derive the SWWE. Crucially, the SWWE neglect terms of order $\mathcal{O}\left(\sigma ^2\right)$ in the Euler equations and are therefore unable to model dispersion. Recent research has highlighted the significant impact of dispersion on the evolution of tsunamis \cite{Grue-etal-2008-113,Kirby-etal-2013-39}, driving the development of models based on Boussinesq-type equations. Among these Boussinesq-type equations the Serre equations are the best placed \cite{Bonneton-Lannes-2009-16601}; retaining high-order $\sigma$ terms and allowing arbitrary wave amplitude. Hence, the overarching goal of  this research is the development of large-scale, efficient and robust computational methods for the Serre equations for the purposes of modelling tsunamis and storm surges.

\section{Objectives of the Thesis}
In view of the overarching goal, the primary motivation of this thesis was the development of a numerical method for solving the one-dimensional Serre equations. This method should be robust to steep gradients in the free surface and the inundation of beaches and be extendable to the two-dimensional Serre equations using unstructured meshes. Since many of these aims were achieved for the SWWE using finite volume methods, the focus of this thesis is the extension of finite volume methods for the SWWE to the Serre equations. 

The primary focus of this thesis is the development of the Finite Element Volume Method (FEVM), which satisfies all the initial goals. The FEVM is an improvement of the Finite Difference Volume Methods (FDVM) described by \citet{Zoppou-2014}. The FEVM can adequately handle dry beds and uses a finite element method instead of a finite difference method, making it suitable for unstructured meshes. 

The FEVM was assessed with a linear analysis, a validation against analytic and forced solutions and experimental results. At all stages of this assessment the method is compared to at least one other method to demonstrate its strengths and weaknesses. Overall, the FEVM compared well with the other methods and satisfied all the objectives of the thesis.

\section{Original Contribution of the Thesis}
The research in this thesis made the following original contributions to the field:
\begin{itemize}
	\item Implementation of the third-order FDVM.
	\item Observation and justification of a new structure in the solution of the Serre equations in the presence of steep gradients in the free surface.
	\item Extension of the second-order FDVM to allow for dry beds.
	\item Development and description of the well-balanced second-order FEVM that is capable of modelling flows over dry beds.
	\item A linear analysis of convergence for all developed finite volume based methods as well as some finite difference methods.
	\item A complete linear analysis of the dispersion properties for all developed finite volume based methods as well as some finite difference methods.
	\item Validation of FEVM and the second-order FDVM using forced solutions where all terms of the Serre equations are present for both wet and dry beds.
	\item Comparison of the numerical solutions of FEVM and the second-order FDVM with experimental results in the presence of dry beds and with wave breaking. 
\end{itemize}

\subsection{Publications}
%[][][]
The publications generated from this research are compiled here in chronological order. A brief summary, my contribution to the paper and the relevance of the paper to this thesis are also provided. The publication list is reproduced in Appendix \ref{app:Pub} where the abstracts of the publications are also provided.
\newpage
\begin{center}
	\vbox{
	\textbf{
		\large A Solution of the Conservation Law Form of the Serre Equations}
	
	\vspace*{\baselineskip}
	
	\textit{Australia and New Zealand Industrial and Applied Mathematics Journal (2016)}
	
	{C. Zoppou, S.G. Roberts and J. Pitt}
	\vspace*{0.5\baselineskip}
}
\end{center}

\vbox{
\noindent\textbf{Summary:}

A second-order FDVM for the one-dimensional Serre equations with a horizontal bed is described and validated against an analytic solution. }
\vspace*{\baselineskip}
\vbox{
\noindent\textbf{My Contribution:}

I reproduced the results of my coauthors with my own implementation of the method. This method was consistent with a SWWE solver allowing the computational cost of solving the Serre equations and the SWWE to be compared.}
\vspace*{\baselineskip}
\vbox{
\noindent\textbf{Relevance to Thesis:}
%[][][]

The method described in this paper was extended to allow for varying bathymetry \cite{Zoppou-etal-2017} and dry beds, producing the second-order FDVM. The results of a linear analysis and numerical experiments for the second-order FDVM are reported in this thesis. The linear analysis of the convergence and dispersion properties performed in this thesis assume a completely wet horizontal bed, and therefore the results in this thesis apply to the method described in this paper. The linear analysis results can be found in Chapter \ref{chp:AnalNumMethod} and Appendix \ref{app:LinAnal}. The extended version of the method was validated against analytic and forced solutions in Chapter \ref{chp:NumMethodComp} and compared to experimental results in Chapter \ref{chp:ExpMethodComp}.}

\newpage
\begin{center}
		\vbox{
	\textbf{
		\large Numerical Solution of the Fully Non-Linear Weakly Dispersive
		Serre Equations for Steep Gradient Flows}
	
	\vspace*{\baselineskip}
	
	\textit{Applied Mathematical Modelling (2017)}
	
	{C. Zoppou, J. Pitt and S.G. Roberts}
	\vspace*{0.5\baselineskip}
}
\end{center}


\vbox{
\noindent\textbf{Summary:}

The first-, second-, and third-order FDVM for the one-dimensional Serre equations with variable bathymetry were described. The results of a linear analysis of the dispersion properties of the methods for waves on quiescent water are provided. These methods were then validated against an analytic solution. A well-balanced version of the second-order FDVM is described and further validated against experimental results.}
\vspace*{\baselineskip}

\vbox{
\noindent\textbf{My Contribution:}

The methods, linear dispersion analysis and numerical solutions were primarily produced by me in collaboration with my coauthors who wrote the paper.}
\vspace*{\baselineskip}

\vbox{
\noindent\textbf{Relevance to Thesis:}

The first-, second- and third-order FDVM described in this paper are the methods whose results are reported in this thesis. The results of a linear analysis of the dispersion and convergence properties of all these methods are provided in Chapter \ref{chp:AnalNumMethod} and Appendix \ref{app:LinAnal}. The linear analysis of dispersion presented in this paper was extended to allow for a mean background flow in the water on which the waves occur. The results of the analytic solution validation in this paper are reproduced in Chapter \ref{chp:NumMethodComp} and are extended by studying the convergence and conservation properties of more quantities. The second-order FDVM was then validated against forced solutions in Chapter \ref{chp:NumMethodComp} and experimental results in Chapter \ref{chp:ExpMethodComp}. The results of the second-order FDVM for the negative rectangular wave experiment \cite{Hammack-Segur-1978-337} and the periodic waves over a submerged bar experiment \cite{Beji-Battjes-1994-1} in this paper are reproduced in this thesis.}

\newpage
\begin{center}
		\vbox{
	\textbf{
		\large Importance of Dispersion for Shoaling Waves}
	
	\vspace*{\baselineskip}
	
	\textit{22nd International Congress on Modelling and Simulation (2017)}
	
	{J. Pitt, C. Zoppou and S.G. Roberts}
	\vspace*{0.5\baselineskip}
	}
\end{center}

\vbox{
\noindent\textbf{Summary:}

The numerical solutions of the second-order FDVM \cite{Zoppou-etal-2017} for the Serre equations were compared to the numerical solutions of ANUGA \cite{ANUGA} for the SWWE. By comparing dispersive and non-dispersive water wave models the influence of dispersion on shoaling waves was studied. Two scenarios were investigated, an artificial example of the shoaling of a solitary wave over a fringing reef and the experimental results of \citet{Beji-Battjes-1994-1} studying periodic waves travelling over a submerged bar.}
\vspace*{\baselineskip}

\vbox{
\noindent\textbf{My Contribution:}

 This paper was primarily produced by me in collaboration with my coauthors, based on research that I primarily undertook.} 
\vspace*{\baselineskip}

\vbox{
\noindent\textbf{Relevance to Thesis:}

The experimental results of the second-order FDVM for the periodic waves over a submerged bar experiments of \citet{Beji-Battjes-1994-1} are reproduced in Chapter \ref{chp:ExpMethodComp}.} 

\newpage

 \begin{center}
 		\vbox{
 	\textbf{
 		\large Behaviour of the Serre Equations in the Presence of Steep Gradients Revisited}
 	
 	\vspace*{\baselineskip}
 	
 	\textit{Wave Motion (2018)}
 	
 	{J.P.A. Pitt, C. Zoppou and S.G. Roberts}
 	\vspace*{0.5\baselineskip}
 }
 \end{center}
 
\vbox{
\noindent\textbf{Summary:}
 
The first-, second- and third-order FDVM as well as two second-order finite difference methods for the Serre equations were used to numerically study the behaviour of steep gradients in the water surface. The convergence and conservation properties of many numerical solutions of all these methods were used to justify the observed behaviours. One such behaviour was not previously reported by the literature and was shown to best represent the solution of the Serre equations in the presence of steep gradients in the water surface over short time spans. The effect of the numerical method, the grid resolution and the smoothing of the initial conditions on the observed behaviours was studied. }
 \vspace*{\baselineskip}

\vbox{ 
 \noindent\textbf{My Contribution:}
 
 This paper was primarily produced by me in collaboration with my coauthors, based on research that I primarily undertook.} 
 \vspace*{\baselineskip}
 
\vbox{
 \noindent\textbf{Relevance to Thesis:}
 
 The behaviours observed and justified in this paper are summarised in Chapter \ref{chp:Serreeqns}. The results of this paper demonstrate the utility of using a finite volume based method to solve the Serre equations in the presence of steep gradients, hence the further development of these methods in this thesis.}

 
\newpage

\section{Organisation of the Thesis}
Chapter \ref{chp:Serreeqns} proceeds by presenting the one-dimensional Serre equations in conservation law form with a source term. The dispersion and conservation properties and known analytic solutions of the Serre equations are also presented. The forced Serre equations and the concept of forced solutions are introduced. Finally, the main results of the investigation into the behaviour of the Serre equations in the presence of steep gradients in the free surface \cite{Pitt-2018-61} are summarised.

This is followed by Chapter \ref{chp:HFVMMethod} which provides a detailed description of the FEVM. In this thesis the results of other numerical methods are also provided, descriptions of these methods can be found in the literature \cite{Zoppou-etal-2017,Pitt-2018-61}. 

Chapter \ref{chp:AnalNumMethod} provides a linear analysis of the convergence and dispersion properties of the FEVM in detail. The analysis begins with the linearised Serre equations over a horizontal bed and then derives the evolution matrix; through which the convergence and dispersion properties can be studied. The results of the linear analysis are also provided for all the methods used by \citet{Pitt-2018-61} as a comparison.

The convergence and conservation properties of the FEVM and all the numerical methods used by \citet{Pitt-2018-61} are then assessed in Chapter \ref{chp:NumMethodComp} using analytic and forced solutions of the Serre equations. This is followed by an experimental validation of the FEVM and the second-order FDVM in Chapter \ref{chp:ExpMethodComp}.

%%%[][][][]
Chapter \ref{chp:Conclusion} summarises the major contributions and findings of this thesis and provides ideas for future work.

Appendix \ref{app:ConQuant} provides expressions for the total amount of the conserved quantities over any domain for all of the analytic solutions described in this thesis. 

The basis functions used by the FEVM and the precise definition of the function spaces mentioned in this thesis are provided in Appendix \ref{app:FEMIntegrals}. 

Appendix \ref{app:LinAnal} provides the evolution matrix produced by the linear analysis for the first-, second- and third-order FDVM and two second-order finite difference methods. It also includes the consistency results for these methods, as they were omitted from Chapter \ref{chp:AnalNumMethod}.

Finally the publications I contributed to throughout the research underpinning this thesis and their abstracts are provided in chronological order in Appendix \ref{app:Pub}. 