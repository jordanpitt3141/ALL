\documentclass[12pt]{article}
\usepackage{amsmath}
\usepackage{ amssymb }
\usepackage{breqn}
\begin{document}
\section{Finite Element}

\[G = uh - \frac{\partial}{\partial x}\left(\frac{h^3}{3}u_x\right)\]

To do so we begin by first multiplying by an arbitrary test function $v$ so that

\[Gv = uhv  - \frac{\partial}{\partial x}\left(\frac{h^3}{3}u_x\right)v\]

and then we integrate over the entire domain to get 
\[\int_\Omega Gv dx = \int_\Omega uhv dx - \int_\Omega \frac{\partial}{\partial x}\left(\frac{h^3}{3}u_x\right)vdx\]

for all $v$

We then make use of integration by parts, with Dirchlet boundaries to get

\[\int_\Omega Gv dx = \int_\Omega uhv dx + \int_\Omega\frac{h^3}{3}u_{x}v_xdx\]

For $u$ we are going to use $x_{j - 1/2}$, $x_{j}$ and $x_{j + 1/2}$ as the nodes, which generate the basis functions $\phi_{j \pm 1/2}$ and $\phi_{j}$ , which for us will be the space of continuous quadratic elements. 

While for $G$  and $h$ we will choose basis functions $w$ that are linear from $[x_{j-1/2}, x_{j+1/2}]$ but discontinuous at the edges.

There are two types of basis functions in this set up the $\phi_{j}$ which are non-zero on
$[x_{j-1/2}, x_{j+1/2}]$ and the $\phi_{j \pm 1/2}$, which we can reduce to just doing it once, but with a translation, so we focus on the $\phi_{j + 1/2}$  which is non-zero on  $[x_{j-1/2}, x_{j+3/2}]$
\section{$\phi_{j}$}
In this section we focus on the test function $v =\phi_{j}$ and thus we focus on the integrals on $[x_{j-1/2}, x_{j+1/2}]$ as 

\[\int_\Omega Gv dx = \int_\Omega uhv dx + \int_\Omega\frac{h^3}{3}u_{x}v_xdx\]

is 

\[\sum_{j}\int_{x_{j-1/2}}^{x_{j+1/2}} G\phi_{j} dx = \sum_{j}\int_{x_{j-1/2}}^{x_{j+1/2}}  uh \phi_{j} dx + \sum_{j}\int_{x_{j-1/2}}^{x_{j+1/2}} \frac{h^3}{3}u_{x}{(\phi_{j})}_{x}dx\]

$$x = \frac{1}{2}\xi\Delta x + x_{j}$$

Taking the derivatives we see


$dx = d\frac{\xi}{2}\Delta x$ , $\frac{dx}{d\xi} = \frac{\Delta x}{2}$ , $\frac{d\xi}{dx} = \frac{2}{\Delta x}$ . \\ \\ We can describe the basis functions in the $\xi$ space, where they are non-zero

\begin{equation}
\phi_{j} = 1 - \xi^2
\end{equation}

\begin{equation}
\phi_{j - 1/2} = \frac{1}{2}\left(\xi^2 - \xi\right)
\end{equation}

\begin{equation}
\phi_{j + 1/2} = \frac{1}{2}\left(\xi^2 + \xi\right)
\end{equation}

\begin{equation}
\phi'_{j} =  - 2\xi
\end{equation}

\begin{equation}
\phi'_{j - 1/2} = \frac{1}{2}\left(2\xi - 1\right)
\end{equation}

\begin{equation}
\phi'_{j + 1/2} = \frac{1}{2}\left(2\xi + 1\right)
\end{equation}

\begin{equation}
w^+_{j - 1/2} = \frac{1}{2}\left(1 - \xi\right)
\end{equation}

\begin{equation}
w^-_{j - 1/2} = \frac{1}{2}\left(1 + \xi\right)
\end{equation}

\[G \approx G' = \sum_{j}G_{j+1/2}w_{j+1/2}\]
\[u \approx u' = \sum_{j} \left[u_{j-1/2}\phi_{j-1/2} + u_{j}\phi_{j} + u_{j+1/2}\phi_{j+1/2}\right] \]
\[h \approx h' = \sum_{j}h_{j+1/2}w_{j+1/2}\]

\subsection{First Integral}
So now we do the substitution for all integrals firstly we do

\[\int_{x_{j-1/2}}^{x_{j+1/2}} G\phi_{j} dx = \int_{-1}^{1} G'(\xi)\phi_{j}(\xi) \frac{d x}{d\xi}d\xi\]

\[ = \frac{\Delta x}{2} \int_{-1}^{1} G'(\xi)\phi_{j}(\xi)d\xi\]

So we focus in on the integral

\[\int_{-1}^{1} G'(\xi)\phi_{j}(\xi)d\xi = \int_{-1}^{1} \left( G^+_{j-1/2} w^+_{j-1/2} +  G^-_{j+1/2} w^-_{j+1/2}  \right)\phi_{j}d\xi \]

\[= G^+_{j-1/2}\int_{-1}^{1}  w^+_{j-1/2}\phi_{j}d\xi  + G^-_{j+1/2}  \int_{-1}^{1}  w^-_{j+1/2} \phi_{j}d\xi\]

we have 
\[\int_{-1}^{1}  w^+_{j-1/2}\phi_{j}d\xi = \int_{-1}^{1} \frac{1}{2}\left(1 - \xi\right)\left(1 - \xi^2\right)d\xi = \frac{1}{2} \frac{4}{3} = \frac{2}{3}\]

\[\int_{-1}^{1}  w^-_{j+1/2} \phi_{j}d\xi = \int_{-1}^{1} \frac{1}{2}\left(1 + \xi\right)\left(1 - \xi^2\right)d\xi = \frac{1}{2} \frac{4}{3} = \frac{2}{3} \]

so then 

\[\int_{-1}^{1} G'(\xi)\phi_{j}(\xi)d\xi = \frac{2}{3} G^+_{j-1/2}  + \frac{2}{3}G^-_{j+1/2}  \]

so we have 

\[\int_{x_{j-1/2}}^{x_{j+1/2}} G\phi_{j} dx = \frac{\Delta x}{2} \left[\frac{2}{3} G^+_{j-1/2}  + \frac{2}{3}G^-_{j+1/2}\right]  =  \frac{\Delta x}{3} \left[ G^+_{j-1/2}  + G^-_{j+1/2}\right]\]

\subsection{Second Integral}

\[\int_{x_{j-1/2}}^{x_{j+1/2}}  uh \phi_{j} dx =  \int_{-1}^{1}  u(\xi)h(\xi) \phi_{j}(\xi) \frac{d x}{d\xi}d\xi = \frac{\Delta x}{2}\int_{-1}^{1}  u'h' \phi_{j} d\xi\]

focusing on the integral 
\[\int_{-1}^{1}  u'h' \phi_{j} d\xi = \int_{-1}^{1} \left(u_{j-1/2}\phi_{j-1/2} + u_{j}\phi_{j} + u_{j+1/2}\phi_{j+1/2}\right) \left(h^+_{j-1/2} w^+_{j-1/2} +  h^-_{j+1/2} w^-_{j+1/2} \right) \phi_{j} d\xi\]

\begin{multline*}
= \int_{-1}^{1}  \Bigg(u_{j-1/2}h^+_{j-1/2} w^+_{j-1/2}\phi_{j-1/2} + u_{j}h^+_{j-1/2} w^+_{j-1/2}\phi_{j} + u_{j+1/2}h^+_{j-1/2} w^+_{j-1/2}\phi_{j+1/2}  \\ + u_{j-1/2}h^-_{j+1/2} w^-_{j+1/2}\phi_{j-1/2} + u_{j}h^-_{j+1/2} w^-_{j+1/2}\phi_{j} + u_{j+1/2}h^-_{j+1/2} w^-_{j+1/2}\phi_{j+1/2}\Bigg) \phi_{j} d\xi
\end{multline*}

\begin{multline*}
= u_{j-1/2}h^+_{j-1/2}  \int_{-1}^{1}w^+_{j-1/2}\phi_{j-1/2} \phi_{j} d\xi + 
 u_{j}h^+_{j-1/2} \int_{-1}^{1} w^+_{j-1/2}\phi_{j} \phi_{j} d\xi  \\+
 u_{j+1/2}h^+_{j-1/2} \int_{-1}^{1} w^+_{j-1/2}\phi_{j+1/2} \phi_{j} d\xi \\+
 u_{j-1/2}h^-_{j+1/2}  \int_{-1}^{1}w^-_{j+1/2}\phi_{j-1/2} \phi_{j} d\xi + 
  u_{j}h^-_{j+1/2} \int_{-1}^{1} w^-_{j+1/2}\phi_{j} \phi_{j} d\xi  \\+
  u_{j+1/2}h^-_{j+1/2} \int_{-1}^{1} w^-_{j+1/2}\phi_{j+1/2} \phi_{j} d\xi
\end{multline*}


Now we calculate the integrals

\[\int_{-1}^{1}w^+_{j-1/2}\phi_{j-1/2} \phi_{j} d\xi = \int_{-1}^{1}\frac{1}{2}\left(1 - \xi \right) \frac{1}{2}\left(\xi^2 - \xi\right) \left(1 - \xi^2\right) d\xi  = \frac{1}{4}\left[\frac{8}{15}\right] = \frac{2}{15}\]

\[\int_{-1}^{1} w^+_{j-1/2}\phi_{j} \phi_{j} d\xi =   \int_{-1}^{1}\frac{1}{2}\left(1 - \xi \right) \left(1 - \xi^2\right) \left(1 - \xi^2\right) d\xi = \frac{1}{2}\left[\frac{16}{15}\right] = \frac{8}{15}\]

\[\int_{-1}^{1} w^+_{j-1/2}\phi_{j+1/2} \phi_{j} d\xi = \int_{-1}^{1}\frac{1}{2}\left(1 - \xi \right) \frac{1}{2}\left(\xi^2 + \xi\right) \left(1 - \xi^2\right) = \frac{1}{4} \times  0 = 0\]

\[\int_{-1}^{1}w^-_{j+1/2}\phi_{j-1/2} \phi_{j} d\xi = \int_{-1}^{1}\frac{1}{2}\left(\xi  + 1 \right) \frac{1}{2}\left(\xi^2 - \xi\right) \left(1 - \xi^2\right) d\xi  = \frac{1}{4} \times  0 = 0 \]

\[\int_{-1}^{1} w^-_{j+1/2}\phi_{j} \phi_{j} d\xi =  \int_{-1}^{1}\frac{1}{2}\left(\xi + 1 \right) \left(1 - \xi^2\right) \left(1 - \xi^2\right) d\xi  = \frac{1}{2}\left[\frac{16}{15}\right] = \frac{8}{15}\]

\[\int_{-1}^{1} w^-_{j+1/2}\phi_{j+1/2} \phi_{j} d\xi =  \int_{-1}^{1}\frac{1}{2}\left(\xi  + 1 \right) \frac{1}{2}\left(\xi^2 + \xi\right) \left(1 - \xi^2\right) = \frac{1}{4}\left[\frac{8}{15}\right] = \frac{2}{15}\]

So we have 

\begin{multline*}
u_{j-1/2}h^+_{j-1/2}  \int_{-1}^{1}w^+_{j-1/2}\phi_{j-1/2} \phi_{j} d\xi + 
 u_{j}h^+_{j-1/2} \int_{-1}^{1} w^+_{j-1/2}\phi_{j} \phi_{j} d\xi  \\+
 u_{j+1/2}h^+_{j-1/2} \int_{-1}^{1} w^+_{j-1/2}\phi_{j+1/2} \phi_{j} d\xi \\+
 u_{j-1/2}h^-_{j+1/2}  \int_{-1}^{1}w^-_{j+1/2}\phi_{j-1/2} \phi_{j} d\xi + 
  u_{j}h^-_{j+1/2} \int_{-1}^{1} w^-_{j+1/2}\phi_{j} \phi_{j} d\xi  \\+
  u_{j+1/2}h^-_{j+1/2} \int_{-1}^{1} w^-_{j+1/2}\phi_{j+1/2} \phi_{j} d\xi
\end{multline*}


\begin{multline*}
 = \frac{2}{15} h^+_{j-1/2} u_{j-1/2}  +  \frac{8}{15}h^+_{j-1/2}u_{j}  +
  \frac{8}{15} h^-_{j+1/2} u_{j}  + \frac{2}{15}h^-_{j+1/2} u_{j+1/2} 
\end{multline*}

So 
\begin{multline*}
\int_{x_{j-1/2}}^{x_{j+1/2}}  uh \phi_{j} dx =  \\
\frac{\Delta x}{2} \left[\frac{2}{15} h^+_{j-1/2} u_{j-1/2}  +  \frac{8}{15}h^+_{j-1/2}u_{j}  +
  \frac{8}{15} h^-_{j+1/2} u_{j}  + \frac{2}{15}h^-_{j+1/2} u_{j+1/2}\right] \\ =
\frac{\Delta x}{15} \left[ h^+_{j-1/2} u_{j-1/2}  +  4h^+_{j-1/2}u_{j}  +
  4 h^-_{j+1/2} u_{j}  + h^-_{j+1/2} u_{j+1/2}\right]  
\end{multline*}

\subsection{Third Integral}
Lastly we have

\begin{multline}
\int_{x_{j-1/2}}^{x_{j+1/2}} \frac{h^3}{3}u_{x}{(\phi_{j})}_{x}dx = \frac{1}{3\Delta x}\int_{-1}^{1} {h^3}u_{\xi}{(\phi_{j})}_{\xi}  d\xi = \frac{1}{3\Delta x}\int_{-1}^{1} h^3u_{\xi}{(\phi_{j})}_{\xi}  d\xi 
\end{multline}

\begin{multline}
\int_{-1}^{1} h^3u_{\xi}{(\phi_{j})}_{\xi}  d\xi  = \int_{-1}^{1} \left(h^+_{j-1/2} w^+_{j-1/2} +  h^-_{j+1/2} w^-_{j+1/2} \right)^3 \times\\ \left(u_{j-1/2}\phi'_{j-1/2} + u_{j}\phi'_{j} + u_{j+1/2}\phi'_{j+1/2}\right)(\phi'_{j}) d\xi 
\end{multline}

\begin{multline}
\int_{-1}^{1} \bigg[\left(h^+_{j-1/2} w^+_{j-1/2}\right)^3 + 3\left(h^+_{j-1/2} w^+_{j-1/2}\right)^2\left(h^-_{j+1/2} w^-_{j+1/2}\right) + 3\left(h^+_{j-1/2} w^+_{j-1/2}\right)\left(h^-_{j+1/2} w^-_{j+1/2}\right)^2  \\+  \left(h^-_{j+1/2} w^-_{j+1/2}\right)^3 \bigg] \times\\ \left(u_{j-1/2}\phi'_{j-1/2}\phi'_{j} + u_{j}\phi'_{j}\phi'_{j} + u_{j+1/2}\phi'_{j+1/2}\phi'_{j}\right) d\xi 
\end{multline}

\begin{multline}
\int_{-1}^{1} \bigg[\left(h^+_{j-1/2} w^+_{j-1/2}\right)^3 + 3\left(h^+_{j-1/2} w^+_{j-1/2}\right)^2\left(h^-_{j+1/2} w^-_{j+1/2}\right) + 3\left(h^+_{j-1/2} w^+_{j-1/2}\right)\left(h^-_{j+1/2} w^-_{j+1/2}\right)^2  \\+  \left(h^-_{j+1/2} w^-_{j+1/2}\right)^3 \bigg]u_{j-1/2}\phi'_{j-1/2}\phi'_{j} \\ +
\bigg[\left(h^+_{j-1/2} w^+_{j-1/2}\right)^3 + 3\left(h^+_{j-1/2} w^+_{j-1/2}\right)^2\left(h^-_{j+1/2} w^-_{j+1/2}\right) + 3\left(h^+_{j-1/2} w^+_{j-1/2}\right)\left(h^-_{j+1/2} w^-_{j+1/2}\right)^2  \\+  \left(h^-_{j+1/2} w^-_{j+1/2}\right)^3 \bigg]u_{j}\phi'_{j}\phi'_{j} \\ +
\bigg[\left(h^+_{j-1/2} w^+_{j-1/2}\right)^3 + 3\left(h^+_{j-1/2} w^+_{j-1/2}\right)^2\left(h^-_{j+1/2} w^-_{j+1/2}\right) + 3\left(h^+_{j-1/2} w^+_{j-1/2}\right)\left(h^-_{j+1/2} w^-_{j+1/2}\right)^2  \\+  \left(h^-_{j+1/2} w^-_{j+1/2}\right)^3 \bigg]u_{j+1/2}\phi'_{j+1/2}\phi'_{j} d\xi
\end{multline}

\begin{multline}
\int_{-1}^{1} u_{j-1/2}\left(h^+_{j-1/2}\right)^3 \left(w^+_{j-1/2}\right)^3 \phi'_{j-1/2}\phi'_{j} \\
+ 3u_{j-1/2}\left(h^+_{j-1/2}\right)^2 h^-_{j+1/2} \left(w^+_{j-1/2}\right)^2\left( w^-_{j+1/2}\right)\phi'_{j-1/2}\phi'_{j} \\
+ 3u_{j-1/2} h^+_{j-1/2}\left(h^-_{j+1/2}\right)^2  w^+_{j-1/2}\left(w^-_{j+1/2}\right)^2\phi'_{j-1/2}\phi'_{j}  \\
+  u_{j-1/2}\left(h^-_{j+1/2}\right)^3 \left(w^-_{j+1/2}\right)^3 \phi'_{j-1/2}\phi'_{j} \\ +
u_{j}\left(h^+_{j-1/2}\right)^3 \left(w^+_{j-1/2}\right)^3 \phi'_{j}\phi'_{j} \\
+ 3u_{j}\left(h^+_{j-1/2}\right)^2 h^-_{j+1/2} \left(w^+_{j-1/2}\right)^2\left( w^-_{j+1/2}\right)\phi'_{j}\phi'_{j} \\
+ 3u_{j} h^+_{j-1/2}\left(h^-_{j+1/2}\right)^2  w^+_{j-1/2}\left(w^-_{j+1/2}\right)^2\phi'_{j}\phi'_{j}  \\
+  u_{j}\left(h^-_{j+1/2}\right)^3 \left(w^-_{j+1/2}\right)^3 \phi'_{j}\phi'_{j} \\ +
u_{j+1/2}\left(h^+_{j-1/2}\right)^3 \left(w^+_{j-1/2}\right)^3 \phi'_{j+1/2}\phi'_{j} \\
+ 3u_{j+1/2}\left(h^+_{j-1/2}\right)^2 h^-_{j+1/2} \left(w^+_{j-1/2}\right)^2\left( w^-_{j+1/2}\right)\phi'_{j+1/2}\phi'_{j} \\
+ 3u_{j+1/2} h^+_{j-1/2}\left(h^-_{j+1/2}\right)^2  w^+_{j-1/2}\left(w^-_{j+1/2}\right)^2\phi'_{j+1/2}\phi'_{j}  \\
+  u_{j+1/2}\left(h^-_{j+1/2}\right)^3 \left(w^-_{j+1/2}\right)^3 \phi'_{j+1/2}\phi'_{j}  d\xi
\end{multline}

\[\int_{-1}^{1} \left(w^+_{j-1/2}\right)^3 \phi'_{j-1/2}\phi'_{j} d\xi\]

etc (print out reference for this)
 



\section{$\phi_{j+ 1/2} $}

In this section we focus on the test function $v =\phi_{j + 1/2}$ and thus we focus on the integrals on $[x_{j-1/2}, x_{j+3/2}]$ as 

\[\int_\Omega Gv dx = \int_\Omega uhv dx + \int_\Omega\frac{h^3}{3}u_{x}v_xdx\]

is 

\[\sum_{j}\int_{x_{j-1/2}}^{x_{j+3/2}} G\phi_{j+ 1/2} dx = \sum_{j}\int_{x_{j-1/2}}^{x_{j+3/2}}  uh \phi_{j + 1/2} dx + \sum_{j}\int_{x_{j-1/2}}^{x_{j+3/2}} \frac{h^3}{3}u_{x}{(\phi_{j+ 1/2})}_{x}dx\]

For our interval

\section{$\phi_{j -  1/2} $}


\section{Combination}


\end{document}
