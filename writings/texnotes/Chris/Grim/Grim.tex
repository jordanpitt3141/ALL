\documentclass[12pt]{article}
\usepackage{amsmath}
\usepackage{ amssymb }
\usepackage{breqn}
\begin{document}



\section{Grimshaw}
Undular bores for the one dimensional Serre equations were analysed by \citeN{El-etal-2006} and an expression for the long time asymptotic amplitude of an advancing front $a^+$ of a bore initially $h_1 \,m$ deep traveling into still water $1m$ deep was given
\begin{linenomath*}
\begin{gather}
\frac{\Delta}{\left(a^+ + 1\right)^{1/4}} - \left(\frac{3}{4 - \sqrt{a^+ + 1}}\right)^{21/10}\left(\frac{2}{1 + \sqrt{a^+ + 1}}\right)^{2/5} = 0
\end{gather}
\end{linenomath*}
 with $\Delta = h_1$. We note that for the dambreak problem one does not use the depth of the water on the left as $\Delta$ but rather calculates the bore height generated by the dam break as in \citeN{El-etal-2006} so that $\Delta =  \left(\sqrt{h_1} + 1\right)^2/4$. This asymptotic result was found by \citeN{El-etal-2006} to be applicable up to $\Delta \approx 1.43$. 



\end{document}
