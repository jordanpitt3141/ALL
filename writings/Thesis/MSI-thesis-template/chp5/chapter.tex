
\chapter{Numerical Validation}
\label{chp:NumMethodComp}

\section{Measuring Convergence and Conservation}
The numerical methods are assessed in this chapter by investigating their convergence and conservation properties. To measure the convergence of these numerical methods we study the difference between numerical and analytic solutions. While conservation is investigated by measuring the total of a conserved quantity in a numerical solution and comparing it to the total of that quantity present in the initial conditions. We introduce notation for these measures and describe their calculation here, beginning with convergence.

\subsection{Measures of Convergence}
By measuring the relative difference between the numerical and analytic solutions as $\Delta x$ varies, the convergence of the numerical methods can be investigated. To measure the relative difference we use the $L_1$ vector norm; to compare the numerical and analytic solutions at the numerical grid locations $x_j$. For a quantity $q$, the vector of its values $\vecn{q}$ at the grid locations $x_j$ and the corresponding numerical solution at those locations $\vecn{q^*}$; the $L_1$ norm is
\begin{equation}
L_1(\vecn{q},\vecn{q^*}) =  \left\lbrace \begin{array}{c r} 
\dfrac{||\vecn{q^*} - \vecn{q}||_{1}}{||\vecn{q}||_{1}} & ||\vecn{q}||_{1} > 0 \\ \\
{||\vecn{q^*}||_{1}} & ||\vecn{q}||_{1} = 0  \end{array}\right. . 
\label{eqn:L1qdef} 
\end{equation}


When no analytic solution is present, we can compare the distance between numerical solutions to gain some insight into how a sequence of numerical solutions is behaving. This allows us to demonstrate that a sequence of numerical solutions is convergent to some solution without possessing an analytic solution. To do this the $L_1$ vector norm is again used as in \eqref{eqn:L1qdef} except now both vectors are numerical solutions. Since both numerical solutions will have different grid locations, we only take the difference between the two at the common grid points. We have constructed our grids to accommodate for this, varying $\Delta x$ by successively dividing by $2$. This ensures that the grid locations generated by the larger $\Delta x$ value are all in the grid generated by the smaller $\Delta x$ value, and so we can compare both numerical solutions at the grid points generated by the larger $\Delta x$ value.  


\subsection{Measures of Conservation}
The conservation properties of the methods can be observed by calculating the total of a conserved quantity in the numerical solution $\mathcal{C}^*\left({\vecn{q^*}}\right)$ and comparing it to the total of that quantity for the initial conditions $\mathcal{C}\left({q(x,0)} \right)$, derived analytically. We do this again using a relative measure, 
\begin{equation}
C_1(q,\vecn{q^*}) =  \left\lbrace \begin{array}{c r} 
\dfrac{|\mathcal{C}^*\left({\vecn{q^*}}\right) - \mathcal{C}\left({q(x,0)} \right)| }{|\mathcal{C}\left({q(x,0)} \right)|} & |\mathcal{C}\left({q(x,0)} \right)| > 0 \\ \\
|\mathcal{C}^*\left({\vecn{q^*}}\right)| & |\mathcal{C}\left({q(x,0)} \right)| = 0  \end{array}\right. . 
\label{eqn:C1qdef} 
\end{equation}
$\mathcal{C}^*\left({\vecn{q^*}}\right)$ was calculated using 3 point Gaussian quadrature over the $j^{th}$ cell and summing these cell integrals for all $j$. The three points needed to perform the Gaussian quadrature were calculated by interpolating the $j^{th}$ cell using a quartic that fits the nodal values $q_{j-2}$, $q_{j-1}$, $q_{j}$, $q_{j+1}$ and $q_{j+2}$. The Gaussian quadrature using three points is $5^{th}$ order accurate and interpolation by quartics is $5^{th}$ order accurate for the quantity $q$ and $4^{th}$ order accurate for its spatial derivative $\partial q /  \partial x$. Since all methods are third-order accurate or less the error introduced by the calculation of $\mathcal{C}^*\left({\vecn{q^*}}\right)$ for the mass, momentum, $G$ and $\mathcal{H}$ will be dominated by the error introduced by the numerical solvers.

In some cases $\mathcal{C}\left({q(x,0)} \right)$ may be difficult to derive analytically. In this case we compare $\mathcal{C}^*\left(\vecn{q^*}\right)$ with $\mathcal{C}^*\left(\vecn{q}^0\right)$; where $\vecn{q}^0$ is the vector of the quantity at the grid locations used as the initial conditions of our numerical method. Comparing these we get 
\begin{equation}
C^*_1({\vecn{q}^0},\vecn{q^*}) =  \left\lbrace \begin{array}{c r} 
\dfrac{|\mathcal{C}^*\left({\vecn{q^*}}\right) - \mathcal{C}^*\left({\vecn{q}^0}\right)| }{|\mathcal{C}^*\left({\vecn{q}^0}\right)|} & |\mathcal{C}^*\left({\vecn{q}^0}\right)| > 0 \\ \\
|\mathcal{C}^*\left({\vecn{q^*}}\right)| & |\mathcal{C}^*\left({\vecn{q}^0}\right)| = 0  \end{array}\right. . 
\label{eqn:C*1qdef} 
\end{equation}


\section{Analytic Solutions}
The first test of a numerical method is its ability to accurately reproduce the analytic solutions of the governing partial differential equations. For the Serre equations there are only two known analytic solutions; cnoidal waves and a lake at rest. We will use both to assess our numerical methods, using the solitary cnoidal wave solution to assess all methods and a lake at rest solution to assess $\text{FEVM}_2$ and $\text{FDVM}_2$ as only these methods allow for variable bathymetery. 

\subsection{Solitary Cnoidal Wave Solution}
To assess the numerical schemes ability to reproduce the solitary cnoidal wave solution we used \eqref{eqn:Solitondefhub} with $a_0 = 1m$ , $a_0 = 0.7m$ and $g= 9.81m/s^2$ at $t=0s$ as initial conditions in our numerical methods. The spatial domain was $[-250m,250m]$ and the problem was solved until $t= 50s$. This was done for a range of $\Delta x$ values that had the following form; $\Delta x = 100 / 2^k m$ with $k \in  \left[6,7, \dots,19\right]$. We satisfied the CFL condition with a CFL number of $Cr = 0.5$ by setting  $\Delta t = Cr / \sqrt{g\left(a_0 + a_1\right)}$. For $\text{FDVM}_2$ and $\text{FEVM}_2$ we used $\theta  = 1.2$ as the limiting parameter in the generalised minmod limiter. 

The parameters $a_0 = 1m$ and $a_0 = 0.7m$ were chosen as for these values the non-linearity parameter $\epsilon = a_1 / a_0 = 0.7$ is beneath the breaking threshold for water waves []. The spatial domain was chosen so that there were no effects from the boundary throughout the numerical solution and the final time allowed for many time-steps to be performed. 

\subsubsection{Results}
%mention FEVM first?
An example numerical solution with $\Delta x = {100} / {2^{11}}m$ from all methods was plotted in Figure \ref{fig:SolitonExAll} against the analytic solution at $t= 50s$. We have only plotted an illustrative amount of the points in the numerical solution. From these plots it is clear that $\text{FDVM}_1$ performs significantly worse than the higher-order methods at reproducing the analytic solution, even for relatively fine grids where the wave is captured by more than $200$ cells. This is primarily due to the numerical diffusion introduced by the method, which has caused the wave in the numerical solution to decrease in amplitude and widen significantly. The higher-order numerical methods all accurately replicate the analytic solution, with insignificant differences in these plots.

\begin{figure}
	\centering
	\begin{subfigure}{0.5\textwidth}
		\includegraphics[width=\textwidth]{./chp5/figures/Analytic/Soliton/Example/FDVM1.pdf}
		\subcaption{$\text{FDVM}_1$}
		\vspace{0.5cm}
	\end{subfigure}%
	\begin{subfigure}{0.5\textwidth}
		\includegraphics[width=\textwidth]{./chp5/figures/Analytic/Soliton/Example/FDVM2.pdf}
		\subcaption{$\text{FDVM}_2$}
		\vspace{0.5cm}
	\end{subfigure}
	\begin{subfigure}{0.5\textwidth}
		\includegraphics[width=\textwidth]{./chp5/figures/Analytic/Soliton/Example/FEVM2.pdf}
		\subcaption{$\text{FEVM}_2$}
		\vspace{0.5cm}
	\end{subfigure}%
	\begin{subfigure}{0.5\textwidth}
		\includegraphics[width=\textwidth]{./chp5/figures/Analytic/Soliton/Example/FDVM3.pdf}
		\subcaption{$\text{FDVM}_3$}
		\vspace{0.5cm}
	\end{subfigure}
	\begin{subfigure}{0.5\textwidth}
		\includegraphics[width=\textwidth]{./chp5/figures/Analytic/Soliton/Example/D.pdf}
		\subcaption{$\mathcal{D}$}
		\vspace{0.5cm}
	\end{subfigure}%
	\begin{subfigure}{0.5\textwidth}
		\includegraphics[width=\textwidth]{./chp5/figures/Analytic/Soliton/Example/W.pdf}
		\subcaption{$\mathcal{W}$}
		\vspace{0.5cm}
	\end{subfigure}
	\caption{Comparison of the analytic solution ({\color{blue} \solidrule}) and numerical solution with $\Delta x = {100} / {2^{11}}m$ ({\color{red} $\bullet$}) for the soliton problem at $t=50s$ for all methods.}
	\label{fig:SolitonExAll}
\end{figure}


The $L_1$ norm was calculated for $h$, $u$ and $G$ for all numerical solutions and was plotted against $\Delta x$ for all numerical methods in Figure \ref{fig:SolitonL1All}. All numerical methods demonstrate convergence for all quantities. Therefore all methods are appropriately reproducing the solitary cnoidal wave solution of the Serre equations. The rate at which the numerical solutions converge to the analytic solution over $\Delta x$ is determined by the order of accuracy of the numerical scheme. All methods demonstrate the expected order of accuracy given the order of accuracy of the approximations used and their order of accuracy from the linear analysis in Chapter \ref{chp:AnalNumMethod}.  

All methods more accurately reproduced the analytic solution for $h$ then either $G$ or $u$ across all $\Delta x$ values. This is due to the simplicity of the continuity equation \eqref{eqn:FullSerreConMass} compared to the irrotationality equation \eqref{eqn:Serreconsconmom} and the error in $u$ being dominated by the error in $G$. 

Increasing the order of accuracy of our numerical methods leads to smaller errors when comparing two methods for the same $\Delta x$ value, as Figure \ref{fig:SolitonL1All} clearly demonstrates. This is consistent with the example numerical solution in Figure \ref{fig:SolitonExAll}, where the lowest order accuracy scheme, $\text{FDVM}_1$ had the poorest reproduction of the analytic solution. However, even though the third-order accurate $\text{FDVM}_3$ is an improvement over its second-order counterparts, this improvement is less pronounced than the improvement between first and second-order methods.

%Difference between FEVM and W
For the second-order methods we find that $\text{FDVM}_2$ consistently produces the smallest $L_1$ error followed by $\text{FEVM}_2$, $\mathcal{W}$ and $\mathcal{D}$. The difference between the $\text{FDVM}_2$ and $\text{FEVM}_2$ is significant with errors of $\text{FEVM}_2$ being $2$ to $4$ times larger than $\text{FDVM}_2$. Both finite difference methods produce very similar errors which are about twice as large as the errors of $\text{FEVM}_2$. 

\begin{figure}
	\centering
	\begin{subfigure}{0.5\textwidth}
		\includegraphics[width=\textwidth]{./chp5/figures/Analytic/Soliton/L1/FDVM1.pdf}
		\subcaption{$\text{FDVM}_1$}
		\vspace{0.5cm}
	\end{subfigure}%
	\begin{subfigure}{0.5\textwidth}
		\includegraphics[width=\textwidth]{./chp5/figures/Analytic/Soliton/L1/FDVM2.pdf}
		\subcaption{$\text{FDVM}_2$}
		\vspace{0.5cm}
	\end{subfigure}
	\begin{subfigure}{0.5\textwidth}
		\includegraphics[width=\textwidth]{./chp5/figures/Analytic/Soliton/L1/FEVM2.pdf}
		\subcaption{$\text{FEVM}_2$}
		\vspace{0.5cm}
	\end{subfigure}%
	\begin{subfigure}{0.5\textwidth}
		\includegraphics[width=\textwidth]{./chp5/figures/Analytic/Soliton/L1/FDVM3.pdf}
		\subcaption{$\text{FDVM}_3$}
		\vspace{0.5cm}
	\end{subfigure}
	\begin{subfigure}{0.5\textwidth}
		\includegraphics[width=\textwidth]{./chp5/figures/Analytic/Soliton/L1/D.pdf}
		\subcaption{$\mathcal{D}$}
		\vspace{0.5cm}
	\end{subfigure}%
	\begin{subfigure}{0.5\textwidth}
		\includegraphics[width=\textwidth]{./chp5/figures/Analytic/Soliton/L1/W.pdf}
		\subcaption{$\mathcal{W}$}
		\vspace{0.5cm}
	\end{subfigure}
	\caption{Convergence plots as measured by the $L_1$ norm for $h$ (\trianglet{blue}), $u$ (\squaret{black}) and $G$ (\diamondt{red}) for the soliton problem for all methods.}
	\label{fig:SolitonL1All}
\end{figure}
%
The $C_1$ norm was measured for mass ($h$), momentum ($uh$), $G$ and $\mathcal{H}$ using the analytic values [] and plotted in Figure \ref{fig:SolitonC1All}. From these plots it can be seen that all methods conserve mass at round-off error as long as $\Delta x$ is not too large, in this case requiring $\Delta x < 0.5m$. Since the FD methods perform just as well at conserving mass as the FDVM and FEVM, this suggests that using a finite volume method for the continuity equation \eqref{eqn:FullSerreConMass} is not necessary to conserve mass for this analytic solution. The $C_1$ norm suggests that $\text{FDVM}_3$ is the worst method for conservation of mass at lower $\Delta x$ values. However, this is due to the larger number of calculations required by this method leading to large accumulation of round-off errors, rather than a particular deficiency in $\text{FDVM}_3$ . 

The conservation of momentum is significantly above round-off error, and for all numerical methods decreases at the rate determined by the order of accuracy or better. For the FDVM and the FEVM this is not surprising as $u$ is calculated from the elliptic equation, which will not necessarily conserve momentum in the system. While the FD methods which solve the momentum equation directly do not conserve momentum as finite difference methods are not necessarily conservative. Given the results for the $L_1$ norm, one might expect that the $C_1$ norm would decrease at the order of accuracy of the scheme. However, given that a quantity can be conserved and possess a large $L_1$ error, for instance translations of the solution, the fact that $C_1$ decreases faster than the order of accuracy is not surprising. 

The second-order $\text{FEVM}_2$ performs the worst of all the methods for the conservation of momentum, even compared to the first-order $\text{FDVM}_1$. The superiority of the FDVM over the FEVM for conservation of momentum appears to be due to the use of the finite difference method to solve the elliptic equation, as both $\text{FDVM}_1$ and $\text{FDVM}_2$ use the same elliptic solver and both conserve momentum better than $\text{FEVM}_2$. 
%momentum at point not an integral??

For the conservation of $G$ we see that the case is very similar to the conservation of momentum, except for $\text{FEVM}_2$ which conserves $G$ at round-off error as long as $\Delta x$ isn't large. This implies that the FD used to solve the elliptic equation while resulting in smaller $L_1$ errors and better conservation of momentum, performs worse for the conservation of $G$. Therefore, we require a FEM to solve the elliptic equation so that our scheme conserves the conservative quantities $h$ and $G$. There appears to be some trade-off here between the conservation of $G$ and the conservation of momentum, with the $\text{FEVM}_2$ improving its conservation of $G$ by worsening its conservation of momentum. 

The FD methods conserve $G$ better than they conserve momentum, this is surprising given that their governing partial differential equations are the conservation of mass and momentum equations \eqref{eqn:FullSerreNonCon}. 

The conservation of the Hamiltonian decreases at the rate given by the order of accuracy of the method or better for all methods. Since no methods were designed using the conservation of Hamiltonian equation, its conservation is a good indication that these numerical method are appropriate for the Serre equations.

Surprisingly the two FD methods conserve the Hamiltonian best, followed by $\text{FDVM}_2$, $\text{FEVM}_2$, $\text{FDVM}_3$ and $\text{FDVM}_1$.  All second-order methods conserve the Hamiltonian with an order of accuracy greater than $2$ while the first and third-order methods conserve the Hamiltonian at their respective order of accuracy. This suggests that even-order schemes have some advantage over odd-order schemes for the conservation of the Hamiltonian.

\begin{figure}
	\centering
	\begin{subfigure}{0.5\textwidth}
		\includegraphics[width=\textwidth]{./chp5/figures/Analytic/Soliton/C1/FDVM1.pdf}
		\subcaption{$\text{FDVM}_1$}
		\vspace{0.5cm}
	\end{subfigure}%
	\begin{subfigure}{0.5\textwidth}
		\includegraphics[width=\textwidth]{./chp5/figures/Analytic/Soliton/C1/FDVM2.pdf}
		\subcaption{$\text{FDVM}_2$}
		\vspace{0.5cm}
	\end{subfigure}
	\begin{subfigure}{0.5\textwidth}
		\includegraphics[width=\textwidth]{./chp5/figures/Analytic/Soliton/C1/FEVM2.pdf}
		\subcaption{$\text{FEVM}_2$}
		\vspace{0.5cm}
	\end{subfigure}%
	\begin{subfigure}{0.5\textwidth}
		\includegraphics[width=\textwidth]{./chp5/figures/Analytic/Soliton/C1/FDVM3.pdf}
		\subcaption{$\text{FDVM}_3$}
		\vspace{0.5cm}
	\end{subfigure}
	\begin{subfigure}{0.5\textwidth}
		\includegraphics[width=\textwidth]{./chp5/figures/Analytic/Soliton/C1/D.pdf}
		\subcaption{$\mathcal{D}$}
		\vspace{0.5cm}
	\end{subfigure}%
	\begin{subfigure}{0.5\textwidth}
		\includegraphics[width=\textwidth]{./chp5/figures/Analytic/Soliton/C1/W.pdf}
		\subcaption{$\mathcal{W}$}
		\vspace{0.5cm}
	\end{subfigure}
	\caption{Conservation plots as measured by $C_1$ for $h$ (\trianglet{blue}), $uh$ (\squaret{black}), $G$ (\diamondt{red}) and $\mathcal{H}$ (\circlet{green!60!black}) for the soliton problem for all methods.}
	\label{fig:SolitonC1All}
\end{figure}

\subsection{Lake at Rest}
%C1, H1 and L1

\section{Forced Solution}
%only L1

\subsection{Travelling Gaussian}


%With $a_0 = 1$ or $a_0 = 0$.
%$a_1 = 0.5$
%$a_2 = 1$
%$a3 = -20$
%$a4 = 1$
%$a5 = a1$
%$a6 = 1$
%$a7 = 0.1$
%$x \in \left[- \frac{\pi}{a_9} , 0\right]$
%$Cr = 0.5$
%$\Delta t = \frac{Cr}{a_5 + \sqrt{g \left(a_0 + a_1\right)}} \Delta x$
%$et = 1$

% Give the source terms for these functions

%Equation for G

% Give the source terms for these functions

%ht

%Gt

%fluxh

%fluxG

%Source G

\begin{figure}
	\centering
	\begin{subfigure}{0.5\textwidth}
		\includegraphics[width=\textwidth]{./chp5/figures/Forced/Dry/P2P/FEVMExw.pdf}
		\subcaption{$w$ and $b$ ({\color{green!60!black} \solidrule})}
		\vspace{0.5cm}
	\end{subfigure}%
	\begin{subfigure}{0.5\textwidth}
		\includegraphics[width=\textwidth]{./chp5/figures/Forced/Dry/P2P/FEVMExh.pdf}
		\subcaption{$h$}
		\vspace{0.5cm}
	\end{subfigure}
	\begin{subfigure}{0.5\textwidth}
		\includegraphics[width=\textwidth]{./chp5/figures/Forced/Dry/P2P/FEVMExG.pdf}
		\subcaption{$G$}
		\vspace{0.5cm}
	\end{subfigure}%
	\begin{subfigure}{0.5\textwidth}
		\includegraphics[width=\textwidth]{./chp5/figures/Forced/Dry/P2P/FEVMExu.pdf}
		\subcaption{$u$}
		\vspace{0.5cm}
	\end{subfigure}
	\caption{Plots of various quantities for the Forced solution at $0s$ ({\color{blue} \solidrule}), $2.5s$ ({\color{red} \solidrule}), $5.0s$ ({\color{violet!80!white} \solidrule}), $7.5s$ ({\color{orange} \solidrule}), $10.0s$ ({\color{black} \solidrule})  .}
	\label{fig:ForcedP2PExAll}
\end{figure}


\begin{figure}
	\centering
	\begin{subfigure}{0.5\textwidth}
		\includegraphics[width=\textwidth]{./chp5/figures/Forced/Dry/P2P/FEVML1.pdf}
		\subcaption{$\text{FEVM}_2$}
		\vspace{0.5cm}
	\end{subfigure}%
	%[]!!---!![]
	\begin{subfigure}{0.5\textwidth}
		\includegraphics[width=\textwidth]{./chp5/figures/Forced/Dry/P2P/FEVML1.pdf}
		\subcaption{$\text{FDVM}_2$}
		\vspace{0.5cm}
	\end{subfigure}
	\caption{Convergence plots as measured by the $L_1$ norm for $h$ (\trianglet{blue}), $u$ (\squaret{black}),  $uh$ ({\color{green!60!black}$\times$})  and $G$ (\circlet{red}) for the forced solution problem for FEVM and FDVM at $t=10s$.}
\end{figure}

\begin{figure}
	\centering
	\begin{subfigure}{0.5\textwidth}
		\includegraphics[width=\textwidth]{./chp5/figures/Forced/Dry/SingleTimeStep/FEVML1.pdf}
		\subcaption{$\text{FEVM}_2$}
		\vspace{0.5cm}
	\end{subfigure}%
	%[]!!---!![]
	\begin{subfigure}{0.5\textwidth}
		\includegraphics[width=\textwidth]{./chp5/figures/Forced/Dry/SingleTimeStep/FEVML1.pdf}
		\subcaption{$\text{FDVM}_2$}
		\vspace{0.5cm}
	\end{subfigure}
	\caption{Convergence plots as measured by the $L_1$ norm for $h$ (\trianglet{blue}), $u$ (\squaret{black}) and $G$ (\circlet{red}) for the forced solution problem for FEVM and FDVM after a single time step.}
\end{figure}

\begin{figure}
	\centering
	\begin{subfigure}{0.5\textwidth}
		\includegraphics[width=\textwidth]{./chp5/figures/Forced/Dry/SingleTimeStep/FEVML1.pdf}
		\subcaption{$\text{FEVM}_2$}
		\vspace{0.5cm}
	\end{subfigure}%
	%[]!!---!![]
	\begin{subfigure}{0.5\textwidth}
		\includegraphics[width=\textwidth]{./chp5/figures/Forced/Dry/SingleTimeStep/FEVML1.pdf}
		\subcaption{$\text{FDVM}_2$}
		\vspace{0.5cm}
	\end{subfigure}
	\caption{Convergence plots as measured by the $L_1$ norm for $h$ (\trianglet{blue}), $u$ (\squaret{black}) and $G$ (\circlet{red}) for the forced solution problem for FEVM and FDVM after a single time step.}
\end{figure}

\begin{figure}
	\centering
	\begin{subfigure}{0.5\textwidth}
		\includegraphics[width=\textwidth]{./chp5/figures/Forced/Dry/uSolve/FEVML1.pdf}
		\subcaption{$\text{FEVM}_2$}
		\vspace{0.5cm}
	\end{subfigure}%
	%[]!!---!![]
	\begin{subfigure}{0.5\textwidth}
		\includegraphics[width=\textwidth]{./chp5/figures/Forced/Dry/uSolve/FEVML1.pdf}
		\subcaption{$\text{FDVM}_2$}
		\vspace{0.5cm}
	\end{subfigure}
	\caption{Convergence plots as measured by the $L_1$ norm for $h$ (\trianglet{blue}), $u$ (\squaret{black}) and $G$ (\circlet{red}) for the for FEVM and FDVM after a single time step.}
\end{figure}


