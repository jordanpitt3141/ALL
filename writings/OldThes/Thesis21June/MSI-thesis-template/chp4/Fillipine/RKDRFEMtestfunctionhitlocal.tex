\documentclass[12pt]{article}
\usepackage{amsmath}
\usepackage{ amssymb }
\usepackage{breqn}
\begin{document}

\section{Elliptic Equation}
The linearised elliptic equation is

\[G = Hu - \frac{H^3}{3}u_{xx}\]

Want to find out the FEM approximation to this $G'$ such that

\[G' = \mathcal{G}_{FE_1} u\]

for $P^1$ FEM

\[G' = \mathcal{G}_{FE_2} u\]

for $P^2$ FEM.

To do so we begin by first multiplying by an arbitrary test function $v$ so that

\[Gv = Huv - \frac{H^3}{3}u_{xx}v\]

and then we integrate over the entire domain to get 
\[\int_\Omega Gv dx = \int_\Omega Huv dx - \int_\Omega \frac{H^3}{3}u_{xx}vdx\]

for all $v$

We then make use of integration by parts, with Dirchlet boundaries to get

\[\int_\Omega Gv dx = \int_\Omega Huv dx + \int_\Omega \frac{H^3}{3}u_{x}v_xdx\]

Our FVM discretisation already has a natrual structure with linear functions intervals of $[x_{j- 1/2} , x_{j+1/2}]$, to achieve this in $P^1$ we have our nodes at the boundaries, thus

So we can reformulate this as 

\[\sum_{j}\int_{x_{j-1/2}}^{x_{j+3/2}} Gv dx = \sum_{j}\int_{x_{j-1/2}}^{x_{j+3/2}} Huv dx + \sum_{j}\int_{x_{j-1/2}}^{x_{j+3/2}} \frac{H^3}{3}u_{x}v_{x}dx\]

or more aptly

\[\sum_{j}\int_{x_{j-1/2}}^{x_{j+3/2}} Gv dx - \int_{x_{j-1/2}}^{x_{j+3/2}} Huv dx - \int_{x_{j-1/2}}^{x_{j+3/2}} \frac{H^3}{3}u_{x}v_{x}dx = 0 \]

for all v

\[\sum_{j}\int_{x_{j-1/2}}^{x_{j+3/2}} Gv dx - H\int_{x_{j-1/2}}^{x_{j+3/2}} uv dx -  \frac{H^3}{3}\int_{x_{j-1/2}}^{x_{j+3/2}}u_{x}v_{x}dx = 0 \]

\section{P1 FEM}
We are going to corodainte tranform from x space the interval $[x_{i-1/2},x_{i+1/2} ,x_{i+3/2}]$ to the $\xi$ space interval $[-1,0,1]$. To accomplish this we have the following relation

$$x = \xi\Delta x + x_{j+1/2}$$

Taking the derivative we see

$$dx = d\xi\Delta x$$


For this FEM we are intereseted in $G_{i+1/2}$ and then we can just get a shift operator to get the otherones. For FEM we replace the functions by their P1 approximations so

\[G \approx G' = \sum^{j}G_{j+ 1/2}v_{j+ 1/2}\]
\[u \approx u' = \sum^{j}u_{j+ 1/2}v_{j+ 1/2}\]

\[\sum_{j}\int_{x_{j-1/2}}^{x_{j+3/2}} G'v dx - H\int_{x_{j-1/2}}^{x_{j+3/2}} u'v dx -  \frac{H^3}{3}\int_{x_{j-1/2}}^{x_{j+3/2}}u'_{x}v_{x}dx = 0 \]

We break this up into the integrals because of the domain of dependence of the basis functions is covered. We also just use a particular basis function as the test function, in particular we choose $v_{j + 1/2}$

\[\int_{x_{j-1/2}}^{x_{j+3/2}} G'(x)v_{j + 1/2} dx = \int_{-1}^{1} G'(\xi)v_{j + 1/2}(\xi) \frac{d \xi}{dx}d\xi\]

\[= \Delta x \int_{-1}^{1} G'(\xi)v_{j + 1/2}(\xi)d\xi\]
\[= \Delta x \int_{-1}^{1} \left(G_{j -1/2}v_{j - 1/2} + G_{j +1/2}v_{j + 1/2} + G_{j +3/2}v_{j +3/2}\right)v_{j + 1/2}d\xi\]

\[= \Delta x\left[G_{j -1/2} \int_{-1}^{1}v_{j - 1/2}v_{j + 1/2}d\xi + G_{j +1/2}\int_{-1}^{1}v_{j + 1/2}^2d\xi + G_{j +3/2}\int_{-1}^{1}v_{j +3/2}v_{j + 1/2}d\xi\right]\]

For linear elements it can be easily shown that 

\[\int_{-1}^{1}v_{j + 1/2}^2d\xi = \frac{2}{3}\]

\[\int_{-1}^{1}v_{j - 1/2}v_{j + 1/2}d\xi = \frac{1}{6} = \int_{-1}^{1}v_{j +3/2}v_{j + 1/2}d\xi \]

So 

\[= \Delta x\left[G_{j -1/2} \frac{1}{6}  + G_{j +1/2}\frac{2}{3} + G_{j +3/2}\frac{1}{6} \right]\]

\[= \frac{\Delta x}{6}\left[G_{j -1/2}  + 4G_{j +1/2} + G_{j +3/2} \right]\]

\[ - H\int_{x_{j-1/2}}^{x_{j+3/2}} u'v dx  = - H\Delta x\int_{-1}^{1} u'v d\xi \]
\[ =  \Delta x\left[u_{j -1/2} \int_{-1}^{1}v_{j - 1/2}v_{j + 1/2}d\xi + u_{j +1/2}\int_{-1}^{1}v_{j + 1/2}^2d\xi + u_{j +3/2}\int_{-1}^{1}v_{j +3/2}v_{j + 1/2}d\xi\right]\]

\[= -H\frac{\Delta x}{6}\left[u_{j -1/2}  + 4u_{j +1/2} + u_{j +3/2} \right]\]

Also 

\[ - \frac{H^3}{3}\int_{x_{j-1/2}}^{x_{j+3/2}}u'_{x}v_{x}dx = \Delta x \frac{H^3}{3}\int_{-1}^{1 }u'_{\xi}v_{\xi} (\frac{d\xi}{dx})^2d\xi \]

\[ = -\frac{1}{\Delta x} \frac{H^3}{3}\left[u_{j -1/2} \int_{-1}^{1}(v_{j - 1/2})_\xi (v_{j + 1/2})_\xi d\xi + u_{j +1/2}\int_{-1}^{1}(v_{j + 1/2})_\xi^2d\xi + u_{j +3/2}\int_{-1}^{1}(v_{j +3/2}))_\xi (v_{j + 1/2}))_\xi d\xi\right]\]

It can be easily shown that 

\[\int_{-1}^{1}(v_{j + 1/2})_\xi^2d\xi = 2\]

\[\int_{-1}^{1}(v_{j - 1/2})_\xi (v_{j + 1/2})_\xi d\xi = -1 = \int_{-1}^{1}(v_{j +3/2}))_\xi (v_{j + 1/2}))_\xi \]

\[ = - \frac{H^3}{3 \Delta x}\left[ -u_{j -1/2} + 2u_{j +1/2} - u_{j +3/2}\right]\]

So we have

\begin{multline}
\Delta x\left[G_{j -1/2} \frac{1}{6}  + G_{j +1/2}\frac{2}{3} + G_{j +3/2}\frac{1}{6} \right] = \\H\frac{\Delta x}{6}\left[u_{j -1/2}  + 2u_{j +1/2} + u_{j +3/2} \right] + \frac{H^3}{3 \Delta x}\left[ -u_{j -1/2} + 2u_{j +1/2} - u_{j +3/2}\right]
\end{multline}

\begin{multline}
\left[G_{j -1/2}  + 4G_{j +1/2} + G_{j +3/2}\right] = \\H\left[u_{j -1/2}  + 4u_{j +1/2} + u_{j +3/2} \right] + \frac{2H^3}{\Delta x^2}\left[ -u_{j -1/2} + 2u_{j +1/2} - u_{j +3/2}\right]
\end{multline}



\end{document}
