\documentclass[12pt]{article}
\usepackage{amsmath}
\usepackage{ amssymb }
\usepackage{breqn}
\begin{document}

\section{Elliptic Equation}
The linearised elliptic equation is

\[G = Hu - \frac{H^3}{3}u_{xx}\]

Want to find out the FEM approximation factor $\mathcal{G}_{FE_1}$ such that

\[G = \mathcal{G}_{FE_1} u\]

To do so we begin by first multiplying by an arbitrary test function $v$ so that

\[Gv = Huv - \frac{H^3}{3}u_{xx}v\]

and then we integrate over the entire domain to get 
\[\int_\Omega Gv dx = \int_\Omega Huv dx - \int_\Omega \frac{H^3}{3}u_{xx}vdx\]

for all $v$

We then make use of integration by parts, with Dirchlet boundaries to get

\[\int_\Omega Gv dx = \int_\Omega Huv dx + \int_\Omega \frac{H^3}{3}u_{x}v_xdx\]

We are going to use $x_j$ as the nodes, which generate the basis functions $\phi_j$, which for us will be the space of continuous linear elements. These are such that $\phi_j(x) \neq 0$ when $x_{j-1} < x< x_{j+1}$ and are the usual hat functions centered at $x_j$.
So we can reformulate this as 

\[\sum_{j}\int_{x_{j-1}}^{x_{j+1}} Gv dx = \sum_{j}\int_{x_{j-1}}^{x_{j+1}}  Huv dx + \sum_{j}\int_{x_{j-1}}^{x_{j+1}}  \frac{H^3}{3}u_{x}v_{x}dx\]


for all v

\section{P1 FEM}
We are going to coordinate tranform from x space the interval $[x_{j-1},x_{j} ,x_{j+1}]$ to the $\xi$ space interval $[-1,0,1]$. To accomplish this we have the following relation

$$x = \xi\Delta x + x_{j}$$

Taking the derivatives we see


$dx = d\xi\Delta x$ , $\frac{dx}{d\xi} = \Delta x$ , $\frac{d\xi}{dx} = \frac{1}{\Delta x}$. \\

Our $\phi_j$ can be described in $\xi$ space as

\begin{equation}
\phi_j = \left\lbrace \begin{array}{c c}
1 + \xi & \xi < 0 \\
1 - \xi & \xi > 0\\
0 & \text{otherwise}
\end{array} 
\right.
\end{equation}

\begin{equation}
\phi_{j-1} = \left\lbrace \begin{array}{c c}
-\xi & \xi < 0 \\
0 & \text{otherwise}
\end{array} 
\right.
\end{equation}

\begin{equation}
\phi_{j+1} = \left\lbrace \begin{array}{c c}
\xi & \xi > 0 \\
0 & \text{otherwise}
\end{array} 
\right.
\end{equation}

For FEM we replace the functions by their P1 approximations so

\[G \approx G' = \sum_{j}G_{j}\phi_{j}\]
\[u \approx u' = \sum_{j}u_{j}\phi_{j}\]

\[\sum_{j}\int_{x_{j-1}}^{x_{j+1}} G'\phi_{j} dx - H\int_{x_{j-1}}^{x_{j+1}} u'\phi_{j} dx -  \frac{H^3}{3}\int_{x_{j-1}}^{x_{j+1}}u'_{x}(\phi_{x})_{j}dx = 0 \]

For all $\phi_j$. For this analysis we choose a particular basis function $\phi_j$ and we look at all the integrals. Beginning with the first one:

\[\int_{x_{j-1}}^{x_{j+1}} G'(x)v_{j} dx = \int_{-1}^{1} G'(\xi)v_{j}(\xi) \frac{d x}{d\xi}d\xi\]

\[= \Delta x \int_{-1}^{1} \left(G_{j- 1}v_{j - 1} + G_{j}v_{j} +G_{j+ 1}v_{j+ 1} \right)v_{j} d\xi\]

\[= \Delta x \left[G_{j- 1} \int_{-1}^{1} v_{j - 1}v_{j} d\xi + G_{j} \int_{-1}^{1} v_{j}v_{j} d\xi +  G_{j+1} \int_{-1}^{1} v_{j + 1}v_{j} d\xi\right]\]

We have that

\[\int_{-1}^{1} v_{j - 1}v_{j} d\xi =  \int_{-1}^{1} v_{j + 1}v_{j} d\xi = \frac{1}{6}\]

\[\int_{-1}^{1} v_{j}v_{j} d\xi = \frac{2}{3}\]

So

\[= \Delta x \left[G_{j- 1} \frac{1}{6} + G_{j} \frac{2}{3} +  G_{j+1} \frac{1}{6}\right]\]

\[=  \frac{\Delta x}{6} \left[G_{j- 1} + 4G_{j} +  G_{j+1} \right]\]

Similarly we have

\[- H\int_{x_{j-1}}^{x_{j+1}} u'v_{j} dx = -\frac{H\Delta x}{6} \left[u_{j- 1} + 4u_{j} +  u_{j+1} \right] \]



\[-\frac{H^3}{3}\int_{x_{j-1}}^{x_{j+1}}u'_{x}(v_{j})_{x}dx = -\frac{H^3}{3}\int_{-1}^{1}u'_{\xi}\frac{d \xi }{dx}(v_{\xi})_{j}\frac{d \xi }{dx} \frac{d x}{d\xi} d\xi   \]

\[= -\frac{H^3}{3\Delta x}\int_{-1}^{1}u'_{\xi}(v_{\xi})_{j} d\xi   \]

We will now use $'$ to denote derivative
\[= -\frac{H^3}{3\Delta x}\int_{-1}^{1}\left(u_{j- 1}'v_{j - 1}' + u_{j}'v_{j}' +u_{j+ 1}'v_{j+ 1}' \right)v_{j}' d\xi   \]

\[= -\frac{H^3}{3\Delta x}\left[u_{j- 2} \int_{-1}^{1} v_{j - 1}'v_{j}' d\xi + u_{j} \int_{-1}^{1} v_{j}'v_{j}' d\xi +  u_{j+1} \int_{-1}^{1} v_{j + 1}'v_{j}' d\xi\right]   \]

We have that 

\[\int_{-1}^{1} v_{j-1}'v_{j}' d\xi = -1 = \int_{-1}^{1} v_{j+1}'v_{j}' d\xi \]

\[\int_{-1}^{1} v_{j}'v_{j}' d\xi = 2\]

Therefore

\[= -\frac{H^3}{3\Delta x}\left[-u_{j- 1}  + 2u_{j} -  u_{j+1} \right]   \]

Then our equation becomes

\begin{multline}
\frac{\Delta x}{6} \left[G_{j- 1} + 4G_{j} +  G_{j+1} \right] = \\ \frac{H\Delta x}{6} \left[u_{j} + 4u_{j} +  u_{j+1} \right] +  \frac{H^3}{3\Delta x}\left[-u_{j- 1}  + 2u_{j} -  u_{j+1} \right]
\end{multline}

\begin{multline}
\left[G_{j-1} + 4G_{j} +  G_{j+1} \right] = \\ H \left[u_{j- 1} + 4u_{j} +  u_{j+1} \right] +  \frac{2H^3}{\Delta x^2}\left[-u_{j- 1}  + 2u_{j} -  u_{j+1} \right]
\end{multline}

This formula is correct for $u= 1,x,x^2$
 

We now assume the following form for $u$ and $G$.

Let quantity $q$ is given by so that
$q(x,t) = q(0,0) e^{i\left(\omega t + kx\right)}$. The use of this comes when we use our uniform grid in space, so that $q(x_j,t) = q_j$ then $q_{j \pm l} = q_j e^{\pm ik l\Delta x} $

Then we have 

\begin{multline}
\left[G_j e^{- ik\Delta x} + 4G_j +  G_j e^{ik\Delta x}  \right] = \\ H \left[u_j e^{- ik \Delta x} + 4u_j  +  u_j e^{ik\Delta x}  \right] +  \frac{2H^3}{\Delta x^2}\left[-u_j e^{- ik \Delta x}  + 2u_j  -  u_j e^{ik \Delta x} \right]
\end{multline}

\begin{multline}
G_j\left[e^{- ik\Delta x} + 4  +  e^{ik\Delta x}  \right] = \\ \left( H \left[e^{- ik \Delta x} + 4   +  e^{ik \Delta x}  \right] +  \frac{2H^3}{\Delta x^2}\left[- e^{- ik \Delta x}  + 2 -  e^{ik \Delta x} \right] \right)u_j
\end{multline}

\begin{multline}
G_j= \\ \left( H  +  \frac{2H^3}{\Delta x^2}\frac{\left[- e^{- ik\Delta x}  + 2  -  e^{ik \Delta x} \right]}{\left[e^{- ik\Delta x} + 4  +  e^{ik\Delta x}  \right]} \right)u_j
\end{multline}

\begin{multline}
G_j=  \left( H  +  \frac{2H^3}{\Delta x^2}\frac{ 2  -2\cos\left(k \Delta x\right) }{4   +2\cos\left(k \Delta x\right) } \right)u_j
\end{multline}

\[
G_j=  \left( H  +  \frac{2H^3}{\Delta x^2}\frac{1  -\cos\left(k \Delta x\right) }{2   +\cos\left(k \Delta x\right) } \right)u_j\]

We want something like

\[\frac{k^2}{3} \approx  \frac{2}{\Delta x^2}\frac{1  -\cos\left(k \Delta x\right) }{2   +\cos\left(k \Delta x\right) }\]

and we want to compare it to the FD approximation

\[\frac{k^2}{3} \approx  \frac{2}{3\Delta x^2}\left(1  -\cos\left(k \Delta x\right)\right)\]

For the FEM we have the taylor series

\begin{multline}
 \frac{2}{\Delta x^2}\frac{1  -\cos\left(k \Delta x\right) }{2   +\cos\left(k \Delta x\right) } = \\ \frac{k^2}{3} + \frac{k^4 \Delta x^2}{36} + \frac{k^6  \Delta x^4}{1080} - \frac{17k^8 \Delta x^6}{181440} - \frac{11k^{10} \Delta x^{8}}{604800} + O(\Delta x^{10})
\end{multline}

\begin{multline}
\frac{2}{3\Delta x^2}\left(1  -\cos\left(k \Delta x\right)\right) = \\ \frac{k^2}{3} - \frac{k^4 \Delta x^2}{36} + \frac{k^6 \Delta x^4}{1080} - \frac{k^8 \Delta x^6}{60480} - \frac{k^{10} \Delta x^{8}}{5443200} + O(\Delta x^{10})
\end{multline}

We can see that because the FD error alternates earlier its error is actualy slightly smaler than the FEM error, hence why it is worse. Although this is a better approximationg than the discontinuous edges one. 






\end{document}
