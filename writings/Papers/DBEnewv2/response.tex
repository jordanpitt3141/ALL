\documentclass[times]{article}
\usepackage{color}
\begin{document}
	\section{Reviewer 1}
	 The authors have improved their manuscript and I find it appropriate for publication.
	
	\section{Reviewer 2}
	The authors have revised the manuscript, with careful attention being paid to my list of
	comments in my original review. I now have a much better understanding of the authors’ reasons for undertaking this work. I think that this now comes across in the revised manuscript.
	I am now happy to recommend that the manuscript be accepted for publication in Wave
	Motion. But I have one small comment on Section 5.2, given below, which I think that the
	authors should note.
	
	A general comment about Section 5.2 and the comparisons of the numerical solutions and
	the solutions of the shallow water equations for the intermediate heights $h_2$ and $u_2$ is that the
	derivation of the dispersive shock fitting method outlined in the review article [7] shows that
	the left going characteristics of the non-dispersive equations (in this case the shallow water
	equations) carry a Riemann invariant which is conserved across the undular bore (dispersive
	shock wave). Hence, the means $h_2$ and $u_2$ should be given by shallow water theory. This also
	relates to point 4 in the authors’ reply in the ``case for publication''.
		\\ \\
		{\color{blue} Again we would like to thank this Reviewer for improving the theoretical aspects of this paper as well as our understanding. We have updated the paper to make it clear that the undulations of the undular bore of the Serre equations oscillating around the bore of the shallow water wave equations is an expected result given the asymptotic theory in [7]. Consequently this paper provides a confirmation that this equivalence extends well to short term numerical solutions for a range of dam-break height ratios, extending previous comparisons to numerical solutions made in the literature.} \\ \\
\end{document}