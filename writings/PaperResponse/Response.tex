\documentclass[times]{article}
\usepackage{color}


\begin{document}
	
\section{Reviewer 1}

The authors of the manuscript ``Behaviour of the Serre equations in the presence of steep gradients revisited" study numerically the structure of undular bores generated by initial conditions with steep gradients using four different finite different (finite volume) methods. Two of the methods were second-order accurate, and the other two were first and third order accurate. The main subject of this manuscript is the different structures and the dynamics of the waves generated in the dam-break problem for the fully-nonlinear and weakly-dispersive shallow water equations (Serre equations).

The paper is very well written and their discoveries are really interesting and new. Initially, the reader may be cautious because it is well-known that finite-difference and finite-volume methods can modify the dynamics of the solutions due to its dissipative nature. Moreover, the linear theory and (as the authors mention) the asymptotics cannot predict the new structures found in this paper. I also had an experience in a conference where one speaker insisted that the tails should separate. But there is no doubt that these structures exist since I repeated the experiments presented in this paper using the  4th-order accurate methods of [8] and of the paper
D. Mitsotakis, C. Synolakis and M. McGuinness, A Modified Galerkin / Finite Element Method for the numerical solution of the Serre-Green-Naghdi system., Int. J. Numer. Meth. Fluids, 83(2017), 755-778 and I confirm that all the results presented in this manuscript are accurate and new. These new discoveries,  are also important because they offer a new problem to the asymptotics community to work with.  I am sure it will attract the interest of many scientists.


For this reason I suggest the publication of this paper.

I would like to suggest the authors to consider including a small paragraph in their conclusions (or anywhere they believe it fits better) mentioning that although the linear theory of V.  Dougalis, A. Duran, M. A. Lopez-Marcos, D. Mitsotakis,  A numerical study of the stability of solitary waves of the Bona-Smith family of Boussinesq systems, J. Nonlinear Sci., 17(2007), 569-607 applied to the linearised equations predicts the separation of the dispersive tails the findings of these paper are due to the nonlinear nature of the equations.
	\\ \\
	{\color{blue} This has been added.} \\ \\

The authors may also find useful to mention that these results can be obtained with other methods such as Galerkin methods.
	\\ \\
	{\color{blue} This has been added. } \\ \\

I leave the inclusion of the previous information at the authors disposal and I only suggest the publication of this manuscript.
	
\section{Reviewer 2}
{\Large \textbf{Review of “Behaviour of the Serre equations in the presence of steep gradients revisited” by J.P.A. Pitt, C. Zoppou and S.G. Roberts}}

	This manuscript deals with numerical solutions of the Serre equations, also termed the	Su-Gardner system, for smoothed step initial conditions. The influence of the amount of smoothing of the jump on the solution is investigated. Five different numerical schemes are used and it is assumed that when these agree they represent an accurate solution of the
	equations. It is not entirely clear to me what the point of this work is. On the one hand, it confirms that as the initial condition becomes steeper the solution converges to what is predicted by Whitham modulation theory. But this is expected. The main issue discussed in	the manuscript seems to be the difference between cases $S_3$ and $S_4$ in terms of the wavetrains on the intermediate level and the influence of the numerical method on this. 	My understanding is that as the initial condition becomes steeper (and the numerical space step smaller), the solution goes from case $S_3$ to case $S_4$. This results in a larger wavetrain
	on the intermediate level which interacts with the trailing, linear edge of the bore and gives a two phase wavetrain in the interaction region. In the second last paragraph of Section 5.1 on page 17 it is stated that as the initial condition becomes steeper, the amplitudes of these wavetrains (around $x_{u_2}$) grow. It is also stated that the numerical methods give different structures for these wavetrains. At the bottom of page 15 it is stated that Figure 11 shows
	that the $S_4$ structure can only be found for very fine grids and that the numerical solution shows poor convergence around $x_{u_2}$. So the natural question is whether the structure around	$x_{u_2}$ is real or a numerical artefact?
	\\ \\
	{\color{blue} Using numerical solutions always presents this problem, but given that there are no analytic solutions, this paper presents a good argument for why one should believe these are not numerical artefacts.
		
	In particular since all five methods in this paper exhibit and converge (apart from around $x_{u_2}$) to the $\mathcal{S}_4$ structure with good conservation of the mass, momentum and energy over the entire domain, this behaviour is not a numerical artefact as such artefacts have different structures for different methods and will lead to large errors in conservation. The first reviewer, independently confirmed the $\mathcal{S}_4$ structure using a fourth-order finite element method.
	
	Furthermore these results demonstrate that if a given numerical method does not observe the $S_4$ structure for the smoothed dam-break problem (or the dam-break problem) then either it was not a high enough resolution grid (the order of accuracy of the method allows lower resolution grids), or the initial conditions were not steep enough (large $\alpha$ values for the smoothed dam-break problem). } \\ \\
	
	As the initial condition is steepened, higher frequency	modes will be generated. This could go some way to explaining the increase in the small amplitude waves on the intermediate level.	The waves on the intermediate shelf coming from the left are due to the smoothing of the	discontinuity in the non-dispersive solution where the intermediate level meets the backward	propagating expansion fan. These wavetrains due to this smoothing of discontinuities in the	non-dispersive solution are higher order than standard Whitham modulation theory, but can be determined using higher order perturbation expansions, see [1] for these details for the KdV
	equation. Whether these waves form on the intermediate level or the upstream level depends	on the sign of the dispersion. Furthermore, it is stated at the bottom of page 20 and the top	of page 21 that the $S_4$ structure decays in time and so is a transient. The dominant long term structure is then the bore, as expected. So I am puzzled as to the point of this manuscript.	Much of it concentrates on this transient wavetrain and studies the solution at short times
	so that its amplitude is reasonably large. \\ \\
	
	{\color{blue} We do focus on this behaviour because it is absent from the literature and unexpected for these equations as mentioned by the first reviewer. We observed this structure for numerical solutions of the dam-break problem on high resolution grids, which was not reported in the literature and so our first thought was that this was some error or numerical artefact. The focus on different structures obtained from different smoothings and different resolutions is then an attempt to show that this behaviour is consistent with other numerical solutions and give some reason as to why other numerical solutions had not observed this behaviour in the literature.
	
	The first reviewer confirms the existence of this structure, and notes that this structure is contrary to the linear theory for dispersive tails. The first reviewer also considers these observations to be important and of interest to our fellow researchers.
	} \\ \\
	
	The other main point seems to be comparisons between various numerical schemes to find which picks up these transients better. The larger issue of the bore for the Serre equations is well known from previous studies. So in my opinion the authors need to make a much better case for the importance of their work sincethe leading and trailing edges of the asymptotic structure, the bore, are known from previous work, before I could recommend publication in Wave Motion. 
	\\ \\
	{\color{blue} 
		This paper focused on numerical methods and numerical results,, as there are no analytic solutions to the Serre equations. The main point of using various numerical schemes is not to compare how well they pick up these transients (although it does serve this purpose), but to demonstrate that this structure is not the numerical artefact of a particular method. Furthermore since some of these methods are the same methods as those in previous papers, it allows us to demonstrate why such behaviour has not been observed in the literature (grid resolution not high enough, $\alpha$ too large, or method is too diffusive). \\
		
		We see the case for publication as follows:\\
		
		1. We observe a new structure in numerical solutions in the short term which was unexpected given the literature. These results demonstrate that one should see this structure for the solution of the Serre equations for the smoothed dam-break and the dam-break in the short term as well. While demonstrating that it is consistent with the expected behaviour in the long term. 
		
		2. We examine the effect of different $\alpha$ values on the structure of the solution of the smoothed dam-break problem.
		
		3. We examine the effect of different methods (orders of accuracy) on the numerical solutions for the smoothed dam-break. With all higher-order method reproducing the $\mathcal{S}_4$ structure.
		
		4. The analytic solution of the Shallow Water Wave is confirmed to be good approximation for the mean velocity and height in Regions III and IV for a range of aspect ratios. These results justify their use in previous papers, and demonstrate their limitations.
		
		5. Whitham modulation results are shown to be an underestimate for the leading wave speed and amplitude even when $\Delta < 1.43$ ($h_0/h_1 < 1.94$). This was modified in response to the review.
		
		Although this paper has a particular focus, we chose this journal because it has a mixed analytical, numerical and experimental audience and so your review is very much appreciated. We believe that these results will be of interest to all these communities (particularly given point 1., which we go to great lengths to justify as you point out). The first reviewer also holds this opinion.
			}
			 \\ \\
	
	In addition, I think that there	is too much detail about basic issues given in the manuscript and it could be shortened. For instance, the start of Section 3 and Sections 3.1 to 3.1.2. It is clear that (3a) will smooth
	the jump and that $\alpha$ measures the steepness of the initial condition, with it becoming steeper as $\alpha$ decreases. It is also clear how the errors in the numerical evaluation of the conserved quantities are calculated and what they mean. \\ \\
		{\color{blue} I have resolved these issues for the mentioned sections and modified the paper according to the general principle of this comment. } \\ \\	
	This is a minor point, but equations (1a) and (1b) could be non-dimensionalised so that dimensional values in meters etc are not needed.
	\\ \\
	{\color{blue} Here we are following the paper that describes th finite difference-volume methods in detail , reference [4] in this paper } \\ \\
	
	Comments on detailed points in the manuscript are 
	
	\begin{enumerate}
		\item Last paragraph of page 6: “Regions III and IV” do not “represent the shock”. The	shock is the discontinuous jump between states IV and V. States III and IV are a partial expansion between the initial levels ahead and behind. This is the same as for the shock tube problem of compressible flow [2]. \\ \\
		{\color{blue} This statement has been corrected.} \\ \\

		\item Second paragraph of Section 5.1 on page 9: The variable $x_{u_2}$ should be defined. It is clear that it is the position at which the wavetrains from the two edges of the intermediate state meet, but this should be stated. \\ \\
		{\color{blue} $x_{u_2}$ was defined earlier, but we have modified this section to remind the reader of its definition.} \\ \\
		
		
		\item The manuscript discusses the growth around $x_{u_2}$ of the intermediate level wavetrains for case $S_4$ , with detailed views shown in Figures 11 and 12. It is clear from Figures 11 and 12 that this region is a two phase wavetrain and the interaction between these two wavetrains is what is causing the rise discussed in the text.
		\\ \\
		{\color{blue} This has been added.} \\ \\
		 
		\item Second paragraph of page 20: It is stated that the agreement between the solution of the non-dispersive shallow water equations and the mean of the numerical solutions on the intermediate level becomes worse as the initial jump height ($h_1/h_0$) increases. This is presumably due to the increasing influence of nonlinearity, so that the mean of the waves on the intermediate level is not near 0.  \\ \\
		{\color{blue} This has been added. We wish to keep these results as other papers have only compared the analytic solution of the Shallow Water Wave equations and the numerical solution of the Serre equations with a plot of the two superimposed. We extend this by demonstrating how these two compare for different aspect ratios, with the results being as expected.} \\ \\
		
		
		\item Second last paragraph on page 20: It is stated that the shock (Rankine-Hugoniot) jump	conditions do not give a good value for the velocity of the lead wave of the bore. This	is well known, for instance see the review article [3]. The lead wave velocity is given by	the velocity of the lead solitary wave of the bore and the conservation of the Riemann invariants of the non-dispersive equations across the bore [3, 4]. This is a different condition to the Rankine-Hugoniot jump conditions. This issue is also discussed in	Chapter 15 of [2], even though its complete resolution was not clear at this time. So the statement in the last sentence of the Conclusions is well studied and well known. \\ \\
		 {\color{blue} We wish to maintain this comment on page 20 and expand it with the references you mention, but we have removed it from the conclusions as it is now only a minor point given your review.} \\ \\
		
		\item Bottom of page 23 and top of page 24: The undular bore solution of the Whitham	modulation equations is not an exact bore solution of the underlying equation, but is an asymptotic solution. So there is no reason to expect that it will agree exactly with numerical solutions. Modulation equations result from assuming a slowly varying periodic wavetrain, as in a multiple scales analysis [2]. It is also well known that these modulation equations give a better result for the velocity of the lead wave of the bore than its amplitude [3]. In addition, the bore solution of Whitham modulation equations	is for an ideal initial step. So, even if it were an exact solution, it could not be captured
		by numerical solutions. So I am not sure that the difference between the ideal bore solution and the numerical solution for a smoothed step shown in Figure 20 is saying much. The figure shows that the agreement is quite good actually given these points.
		 \\ \\
		 {\color{blue} We have added this. I also have included some explanation for why this results should be included. Mainly because the current results comparing the Whitham modulation results and numerical solutions for the leading wave speed and amplitude suggest that if $h_0 / h_1 < 1.94$ then the numerical solutions should converge up to the Whitham modulation results. Our results however, demonstrate this is not necessarily the case.} \\ \\
	\end{enumerate}


\begin{thebibliography}{16}
	\providecommand{\natexlab}[1]{#1}
	\providecommand{\url}[1]{\texttt{#1}}
	\providecommand{\urlprefix}{URL }
	\expandafter\ifx\csname urlstyle\endcsname\relax
	\providecommand{\doi}[1]{doi:\discretionary{}{}{}#1}\else
	\providecommand{\doi}[1]{doi:\discretionary{}{}{}\begingroup
		\urlstyle{rm}\url{#1}\endgroup}\fi
	\providecommand{\bibinfo}[2]{#2}
	
	
	\bibitem[1]{JALEACH}
	\bibinfo{author}{J. A. Leach}, \bibinfo{author}{D. J. Needham}, \bibinfo{title}{The large-time development of the solution to an initial-value problem for the Korteweg-de Vries equation: I. Initial data has a discontinuous expansive step}, \bibinfo{journal}{Nonlinearity},
	\bibinfo{volume}{21}~(\bibinfo{number}{10}) (\bibinfo{year}{2008})
	\bibinfo{pages}{2391--2408}.
	
	\bibitem[2]{Whitham}
	\bibinfo{author}{G.B. Whitham}, \bibinfo{title}{Linear and Nonlinear Waves}, \bibinfo{publisher}{J. Wiley and Sons}, \bibinfo{location}{New York}
	(\bibinfo{year}{1974}).

	\bibitem[3]{Disp}
	\bibinfo{author}{G.A. El}, \bibinfo{author}{M.A. Hoefer}, \bibinfo{title}{Dispersive shock waves and modulation theory}, \bibinfo{journal}{Physica D},
	\bibinfo{volume}{333} (\bibinfo{year}{2016})
	\bibinfo{pages}{11--65}.	

	\bibitem[4]{Res}
	\bibinfo{author}{G.A. El}, \bibinfo{title}{Resolution of a shock in hyperbolic systems modified by weak dispersion}, \bibinfo{journal}{Chaos}
	\bibinfo{volume}{15} \bibinfo{pages}{37103} (\bibinfo{year}{2005}).	
	
\end{thebibliography}
\end{document}